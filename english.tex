\documentclass[10pt,a4paper]{article}
\usepackage[utf8]{inputenc}
\usepackage[english]{babel}
\usepackage{amsmath}
\usepackage{amsfonts}
\usepackage{amssymb}
\usepackage{soul}
\usepackage[margin=1in]{geometry}
\usepackage{enumitem}
\usepackage{adjustbox}

\newlength{\drop}

\usepackage{tikz}

\newcommand{\inline}[2]{%
    \begin{tikzpicture}[baseline=(word.base), txt/.style={shape=rectangle, inner sep=0pt}]% the baseline key ensures that nodes won't shift up if there's text with descenders, and the txt style removes extra spacing so you can use this inline
    \node[txt] (word) {#1};% the first argument is the contents of the main node
    \node[above] at (word.north) {\footnotesize{#2}};% the second argument is the tag; you can play with the positioning as necessary
    \end{tikzpicture}%
    }


\begin{document}

\begin{titlepage}

\drop=0.1\textheight
    \centering
    \vspace*{\baselineskip}
    \rule{\textwidth}{1.6pt}\vspace*{-\baselineskip}\vspace*{2pt}
    \rule{\textwidth}{0.6pt}\\[\baselineskip]
    {\LARGE ENGLISH\\[0.2\baselineskip] CHEAT SHEET}\\[0.2\baselineskip]
    \rule{\textwidth}{0.4pt}\vspace*{-\baselineskip}\vspace{3.2pt}
    \rule{\textwidth}{1.6pt}\\[\baselineskip]
    \scshape
    From A1 to C2 \\
    \vspace*{2\baselineskip}
    Edited by \\[\baselineskip]
    {\Large MAXIMILIANO PONCE\par}
    {\itshape From Langpill\\ English Grammar Course \\\par}
    \vfill
    {\scshape 2020} \\
    {\large MAXIMILIANO PONCE}\par

\end{titlepage}

\tableofcontents
\newpage

\section{Introduction}
\indent
This document was written to understand the English grammar from A1 to C2 levels. In order to let you to study on you own schedule this document is divided in the main topics of English grammar, so that you can go directly to the topic that you want to learn.\\
\indent
Remember that is important to practice on your own to master the English language, another advice for you is to speak with a friend of you who wants to learn English or already knows it, or even to your self, about any topic you like and if you are stuck to explain something then you can search on internet or ask to your friend to get feedback.
\newpage

\section{Nouns}
A \textbf{noun} is a word used to identify people, places, or things (\textbf{common nouns}) or to name a particular one of these (\textbf{proper noun}).
\begin{center}
		\textit{ You can buy} \inline{ \textit{\textbf{coffee}}}{Common noun}
		\textit{eat} \inline{\textit{\textbf{Starbucks}}}{Proper noun}.
\end{center}
\subsection{Common nouns}
A \textbf{common noun} is a noun showing a class of objects or a concept as opposed to a particular individual\\
\begin{center}
\textit{
There was a \textbf{sofa}, two \textbf{chairs}, and a \textbf{wardrobe} in the \textbf{room}}
\end{center}
\indent
Note that \textbf{common nouns} are general names. \textbf{They are not capitalized unless they begin a sentence or are part of a title}.
\begin{center}
		\textit{
		\textbf{Capitals} of the countries are usually very large cities\\
London is the \textbf{capital} of Great Britain}
\end{center}
\subsubsection{Plural form}
\begin{itemize}
		\item Noun $+$ S $\Rightarrow$ Plurar form \\e.g.\textit{flower-flowers, dog-dogs}
		\item Noun ending in $s$, $ss$, $sh$, $ch$, $x$, $z$, $o$ $+$ $es$ $\Rightarrow$ Plural form \\ e.g. \textit{ bus-buses, watch-watches, box-boxes}
		\item Noun ending in $f$, $fe$ $\Rightarrow$ Change $f$ into $ve$ $+$ $s$ \\e.g. \textit{life-lives, wolf-wolves} \quad (but: \textit{belief-beliefs, chef-chefs})
\end{itemize}
Remember that some nouns are \textbf{irregular}\\
\begin{table}[h]
\begin{center}
\begin{tabular}{|c|c|}
\hline
\textbf{Singular} & \textbf{Plural}\\
\hline
man & men\\ \hline
woman & women\\\hline
person & people\\\hline
child & children\\\hline
tooth & teeth\\\hline
foot & feet\\\hline
mouse & mice\\
\hline
\end{tabular}
\end{center}
\caption{\label{tab:nouns1}Some irregular nouns}
\end{table}

\subsection{Proper nouns}
A \textbf{proper noun} is a noun that refers to a unique thing, such as \textit{names, names of cities, planets, corporations, etc.} but a common noun usually refers to a class of things.
\begin{center}
		\textit{\textbf{London} is the capital of \textbf{Great Britain}.}
\end{center}
Note that proper nouns are unique names. \textbf{They are capitalized}
\begin{center}
\textit{\textbf{Olivia} wants to travel around \textbf{Europe} next year.}
\end{center}
We should also capitalize:
\begin{enumerate}[label=\alph*)]
		\item Festiavals\\
				e.g. \textit{\textbf{Christmas} and \textbf{Thanksgiving} are my two favourite holidays!}
		\item People's titles\\
				e.g. \textit{Everything depends on \textbf{President} Trump and his decisions.}
		\item The names of books, fils, plays, paintings. We use capital letters for the nouns, adjectives, and verbs in the title.\\
				e.g. \textit{I've just finished reading \textbf{'The Old Man and the Sea'}}
\end{enumerate}
Sometimes we use a person's name to refer to something ther have created.
\begin{center}
\textit{We were listening to \textbf{Mozart} the other day.}\\
\textit{I'm reading \textbf{an Iris Murdoch} now.}
\end{center}

When you use a word about a family member (e.g. \textit{mom, dad, uncle}), capitalize it only if the word is being used exactly as you would use a name, i.e. if you were addressing the person directly. If the word is not being used as a name, it is not capitalized.
\begin{center}
		\textit{Please ask \textbf{Dad} if he can buy wine on his way home.}\\
		\textit{Is your \textbf{dad} coming over for dinner?}
\end{center}
Whenever you see a capitalized word, question whether or not it is a proper noun. Make sure that the capitalized word is in fact a noun as there are also proper adjectives.
\begin{center}
		\textit{ \textbf{Asia} is one of the continents of the wold.} (proper noun)\\
		\textit{I don't like \textbf{Asian} food.} (proper adjective).
\end{center}

\subsection{Pronouns}
\subsubsection{Subject Pronouns}
A \textbf{subject} is the person or thing that performs the action in the clause or sentence.

A \textbf{subject pronoun} is a pronoun that takes the place of a noun as the subject of a sentence

\begin{center}
\textit{
\textbf{She} told me about her worries.}
\end{center}
Subject pronouns replace nouns that are the subject of their clause.
\begin{table}[h]
\begin{center}
		\begin{tabular}{|c|c|c|}
		\hline
		& \textbf{Singular} & \textbf{Plural} \\
		\hline
		$1^{st}$ person & I & we \\ \hline
		$2^{nd}$ person & you & you \\ \hline
		$3^{rd}$ person & he/she/it & they \\ \hline
	\end{tabular}
\end{center}
\caption{\label{tab:nouns2}Singular and plural forms for subject pronouns}
\end{table}

We should replace the subject with a subject pronoun to avoid repetition.
\begin{center}
\textit{
\textst{Mary is a student and Mary is very hard working.}\\
Mary is a studen and \textbf{she} is very hard working.}
\end{center}

We use the subject pronoun \textit{it} when we refer to objects, things, animals, or ideas.
\begin{center}
		\textit{ Love is eternal. \textbf{It} will last forever.}
\end{center}

Sometimes when we don't know the sex of a baby, we can use \textit{it}'.
\begin{center}
\textit{
Their baby is so small. \textbf{It} only weights 2 kilos.}
\end{center}
\hspace{0.4cm} We use \textit{it} when we talk about time, weather, or temperature.
\begin{center}
\textit{
	What time is \textbf{it}? - \textbf{It}'s 7 o'clock.\\
	\textbf{It}'s quite cold today.}
\end{center}

\subsubsection{Object Pronouns}
An \textbf{object} is the person or thing that receives the action in the clause or sentence.\\
An \textbf{object pronoun} is a pronoun that takes the place of a noun as the object of a sentence.
\begin{center}
\textit{
She told \textbf{me} about her worries.}
\end{center}

Object pronouns are used to replace nouns that are the direct or indirect object of a clause.
\begin{table}[h]
\begin{center}
\begin{tabular}{|c|c|}
	\hline
	\textbf{Subject} & \textbf{Object} \\
	\hline
	I & me \\ \hline
	you & you \\ \hline
	he & him \\ \hline
	she & her \\ \hline
	it & it \\ \hline
	we & us \\ \hline
	they & them \\
	\hline
\end{tabular}
\end{center}
\caption{\label{tab:nouns3}Some irregular nouns}
\end{table}

Object pronouns come either after a verb or a preposition.
\begin{center}
\textit{
Ethan asked \textbf{me} to tal to \textbf{them}.}
\end{center}

Note tha the subject pronoun \textit{it} and the object pronoun \textit{it} look the same.

\begin{center}
\textit{
Do you know the movie 'Pertty Lady'? \textit{it} is my favourite!} (subject pronoun)\\
\textit{ I've seen \textit{it} many times.} (object pronoun)
\end{center}
Remember that object nouns are always the recipients of the action in sentence.
\begin{center}
\textit{
		\textst{He and me went to the movies}. \textbf{He and I} went to the movies.\\
\textst{Mrs. Keith called her and I}. Mrs. Keith called \textbf{her and me}.}
\end{center}

We should replace the object with an object pronoun to avoid repetition.
\begin{center}
\textit{
I can't stop thinking about Amy. \textst{I can't stop imagining my future with Amy}. I can't stop imagining my future with \textbf{her}.}
\end{center}

\subsection{Material Nouns}
\textbf{Material nouns} denote a material or substance from which things are made of.
\begin{center}
		a \textbf{plastic} bottle, a \textbf{diamond} ring, etc.
\end{center}
Material nouns are uncountable, thus they do not have a plural form. Generally, articles are not used with material nouns as they are uncountable.
\begin{center}
\textit{
		\textst{I really want to buy these cottons pants.}\\
I really want to buy these \textbf{cotton} pants.}
\end{center}

Material nouns fall into several categories:
\begin{center}
		\begin{enumerate}[label=\alph*)]
		\item Related to nature\\
				e.g. air, salt, coal, silver, gold, etc.
		\item Related to animals\\
				e.g. meat, milk, egg, wool, etc.
		\item Related to plants\\
				e.g. cotton, coffee, tea, wood, etc.
		\item Arificial or man-made materiales\\
				e.g. alcohol, cheese, brick, steel, etc.
		\end{enumerate}
\end{center}

\subsection{Compund nouns}
A \textbf{compund noun} contains two or more words which are joined together and form a single noun. Compund nouns can be words written together, words that are hyphenated, or separate words.\\
The first word usually describes or modifies the second word, denoting its type or purpose, Consequently, the second word identifies the item itself.
\begin{center}
\textit{
I need to buy a new \textbf{toothbrush}.} ( a brush used for cleaning one's teeth)
\end{center}

There is no exact rule as to when we should write compund nouns together, hyphenated, or as separate words. If you are not sure how to write a compund nound, \textbf{consult a dictionary}.
\begin{center}
\textit{
		Could you go with me to the \textbf{bus stop}?\\
		My \textbf{in-laws} are incredible people.\\
I love your new \textbf{haircut}! You look fantastic!}
\end{center}
Note that the stress usually falls on the first syllable in compund nouns. As a result, the word stress helps to differentiate between a compund noun and an adjective + noun.
\begin{center}
\textit{
A \textbf{greenhouse} is a glass building used for growing plants that need warmth, light, and protection.} (compund noun)\\
\textit{
A \textbf{green house} is a building that someone lives in. This building is painted green.} (adjective + noun)
\end{center}

\subsection{Countable vs Uncountable Nouns}
\vspace{0.5cm}
\begin{tabular}{|c|c|}
		\hline
		\textbf{Countable Nouns (e.g. apple, song, house, etc.)} & \textbf{Uncountable Nouns (e,g, tea, money, love, etc.)}\\
		\hline
		\parbox[t]{8cm}{\vspace{0.08cm} Things that \textbf{can be counted}, even if the number\\ might be extremely high (e.g. all the people in\\ the world).\vspace{0.08cm}} & \parbox[t]{8cm}{\vspace{0.08cm}Things that we \textbf{cannot count} with numbers.\\ They may be the names for abstract ideas or\\ qualities or for physical objects that are too\\ small to count or shapeless (e.g. liquids, gases, etc.).\vspace{0.4cm}}\\
		\hline
		\parbox[t]{8cm}{\vspace{0.08cm}Can be singular or plural.\\ \textit{I have an \textbf{apple} and you have three \textbf{apples}.}\vspace{0.08cm}} & \parbox[t]{8cm}{ \vspace{0.08cm}No plural form.\\\textit{We're goint to have \textbf{rice} for lunch.}\vspace{0.4cm}}\\
		\hline
		\parbox[t]{8cm}{\vspace{0.08cm}You can use \textit{a/an} with singular countable nouns.\\ \textit{There is \textbf{a girl} outside. She is wearing \textbf{a beautiful dress}.}\vspace{0.08cm}} & \parbox[t]{8cm}{\vspace{0.08cm}You can't use \textit{a/an} wih uncountable nouns. But \\ you can often use the phrase \textit{a (bag, cup, etc.) of}.\\ \textit{There is \textbf{a bowl of rice} and \textbf{a bottle of juice} on the table.} \vspace{0.4cm}}\\
		\hline
\parbox[t]{8cm}{\vspace{0.08cm}If you want to ask about the quantity of a\\ countable noun, you ask \textit{'How many?'} combined \\ with the plural countable noun.\\ \textit{ \textbf{How many dogs} are there? - There are \textbf{five dogs.}}\vspace{0.08cm}} & \parbox[t]{8cm}{\vspace{0.08cm}If you want to ask about the quantity of an \\ uncountable noun, you ask \textit{'How much?'}\\ combined with the uncountable noun.\\ \textit{ \textbf{How much coffee} do we have left? - We don't have \textbf{much coffee} left.}\vspace{0.4cm}} \\
\hline
\parbox[t]{8cm}{\vspace{0.08cm}You can use \textit{many, a few, few} with plural\\ countable nouns.\\ \textit{Sorry, but I didn't take \textbf{many pictures.}\\I've got \textbf{a few relatives} leaving here.}\vspace{0.08cm}} & \parbox[t]{8cm}{{\vspace{0.08cm}You can use \textit{much, a little, little} with uncountable\\ nouns.\\ \textit{We didn't do \textbf{much shopping} there. \\ We have \textbf{a little sugar} left.} }\vspace{0.4cm}} \\
\hline
\multicolumn{2}{|c|}{ \parbox[t]{16cm}{\begin{center}You can use \textit{some, any, a lot of, both} with plural countable nouns and uncountable nouns. \end{center}}} \\
\hline
\textit{We like singing \textbf{some crazy songs} at karaoke.} & \textit{We listened to \textbf{some music} there.}\\
\hline
\textit{Did you buy \textbf{any oranges}?} & \textit{I didn't buy \textbf{any orange juice}.}\\
\hline
\textit{She showed \textbf{a lot of signs} of affection.} & \textit{There is \textbf{a lot of love} in the air.}\\ \hline

\end{tabular}






















\end{document}
