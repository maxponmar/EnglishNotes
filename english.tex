\documentclass[10pt,a4paper]{article}
\usepackage[utf8]{inputenc}
\usepackage[english]{babel}
\usepackage{amsmath}
\usepackage{amsfonts}
\usepackage{amssymb}
\usepackage{soul}
\usepackage[margin=1in]{geometry}
\usepackage{enumitem}
\usepackage{adjustbox}

\newlength{\drop}

\usepackage{tikz}

\newcommand{\inline}[2]{%
    \begin{tikzpicture}[baseline=(word.base), txt/.style={shape=rectangle, inner sep=0pt}]% the baseline key ensures that nodes won't shift up if there's text with descenders, and the txt style removes extra spacing so you can use this inline
    \node[txt] (word) {#1};% the first argument is the contents of the main node
    \node[above] at (word.north) {\footnotesize{#2}};% the second argument is the tag; you can play with the positioning as necessary
    \end{tikzpicture}%
    }

\begin{document}

\begin{titlepage}

\drop=0.1\textheight
    \centering
    \vspace*{\baselineskip}
    \rule{\textwidth}{1.6pt}\vspace*{-\baselineskip}\vspace*{2pt}
    \rule{\textwidth}{0.6pt}\\[\baselineskip]
    {\LARGE ENGLISH\\[0.2\baselineskip] NOTES}\\[0.2\baselineskip]
    \rule{\textwidth}{0.4pt}\vspace*{-\baselineskip}\vspace{3.2pt}
    \rule{\textwidth}{1.6pt}\\[\baselineskip]
    \scshape
    From A1 to C2 \\
    \vspace*{2\baselineskip}
    Edited by \\[\baselineskip]
    {\Large MAXIMILIANO PONCE\par}
    {\itshape My notes from \\Langpill english grammar course from Udemy, and Grammarly\\\par}
    \vfill
    {\scshape April 2020} \\
    {\large MAXIMILIANO PONCE}\par

\end{titlepage}

\tableofcontents
\newpage

\section{Introduction}
\indent
This document was written for me to understand the english grammar from a beginner to an advanced level. Because I find that I can unsertand better if I take notes for myself while I'm watching english lessons, and I decided to make my notes with \LaTeX  beacause it is a usefull and cool tool for writing awesome docuements.\\

\indent I know that I a need to improve my english writing and as I improve it I will update this document. I hope my note become useful to other people so this document is divided in the main topics of english grammar, so that you can go directly to the topic that you want to learn.\\

\indent
Remember that is important to practice on your own to master the English language, another advice for you is to speak with a friend of you who wants to learn English or already knows it, or even to your self, about any topic you like and if you are stuck to explain something then you can search on internet or ask to your friend to get feedback.\\

\newpage

\section{Nouns}
A \textbf{noun} is a word that is used to name a person, animal, place, action or thing either generally (\textbf{common noun}) or specifically ( \textbf{proper noun)}.
\begin{center}
		\textit{ You can buy a} \inline{ \textit{\textbf{pencil}}}{Common noun}
		\textit{at} \inline{\textit{\textbf{Office Depot}}}{Proper noun}.
\end{center}

\subsection{Plural form}
There are rules to form plural forms, but remember that some nouns are irregular so that you can't use those rules to form the plural form.

\begin{itemize}
		\item Generally we can use $\Rightarrow$ \underline{Noun $+$ S}, to get the plural form \\e.g.\textit{cat-cats, dog-dogs, pencil-pencils}
		\item Singular noun ending in \underline{$s$, $ss$, $sh$, $ch$, $x$, $z$, $o$ $+$ $es$}, $\Rightarrow$ Plural form \\ e.g. \textit{ tax-taxes, bus-buses, box-boxes}
		\item In some cases, singular nouns ending in \underline{$s$ or $z$} $\Rightarrow$ Double $s$ or $z$ \\
			e.g \textit{fez-fezzes, gas-gasses}
		\item Noun ending in \underline{$f$, $fe$} $\Rightarrow$ Change $f$ into $ve$ $+$ $s$ \\e.g. \textit{life-lives, wolf-wolves, wife-wives} \quad (Exceptions: \textit{roof-roofs, belief-beliefs, chef-chefs})
		\item Noun ending in $y$ and the letter before is a \underline{consonant} $\Rightarrow$ Change the ending to $ies$ \\
			e.g. \textit{city-cities, puppy-puppies}
		\item Noun ending in $y$ and the letter begfore is a \underline{vowel} $\Rightarrow$ Add $s$ \\
			e.g. \textit{ray-rays, boy-boys}
		\item Noun ending in $o$ $\Rightarrow$ Add $es$ \\
			e.g. \textit{potato-potatoes, tomato-tomatoes} \quad (Exceptions: \textit{photo-photos, piano-pianos, halo-halos}
		\item Nound ending in $us$ $\Rightarrow$ Frequently change $us$ to $i$ \\
			e.g. \textit{cactus-cacti, focus-foci}
		\item Noun ending in $is$ $\Rightarrow$ Change $is$ to $es$ \\
			e.g. \textit{analysis-analyses, ellipsis-ellipses}
		\item Noun ending in $on$ $\Rightarrow$ Change $on$ to $a$ \\
			e.g. \textit{phenomenon-phenomena, criterion-criteria}
		\item Some nouns \underline{don't change} \\
			e.g. \textit{sheep-sheeps, series-series, species-species}
\end{itemize}
Here we have some \textbf{irregular nouns}, they don't follow specific rules\\
\begin{table}[h]
\begin{center}
\begin{tabular}{|c|c|}
\hline
\textbf{Singular} & \textbf{Plural}\\
\hline
man & men\\ \hline
woman & women\\\hline
person & people\\\hline
child & children\\\hline
tooth & teeth\\\hline
foot & feet\\\hline
mouse & mice\\
\hline
\end{tabular}
\end{center}
\caption{\label{tab:nouns1}Some irregular nouns}
\end{table}


\subsection{Common nouns}
\textbf{Common nouns} refer to classes or categories of people, animals, places, things, or a concept, as opposed to a particular individual.
\begin{center}
	\textit{I have a \textbf{computer}, a \textbf{keyboard}, a \textbf{mouse} and many \textbf{books}. }
\end{center}
\indent
\textbf{Common nouns} are \textbf{not capitalized} unless they begin a sentence or are part of a title.
\begin{center}
		\textit{
		\textbf{Capitals} of the countries are usually very large cities\\
London is the \textbf{capital} of Great Britain}
\end{center}

\subsection{Proper nouns}
A \textbf{proper noun} is a noun that refers to a unique thing, such as \textit{names, names of cities, planets, corporations, etc.} but a common noun usually refers to a class of things.
\begin{center}
		\textit{\textbf{London} is the capital of \textbf{Great Britain}.}
\end{center}
Note that proper nouns are unique names. \textbf{They are capitalized}
\begin{center}
\textit{\textbf{Olivia} wants to travel around \textbf{Europe} next year.}
\end{center}
We should also capitalize:
\begin{enumerate}[label=\alph*)]
		\item Festiavals\\
				e.g. \textit{\textbf{Christmas} and \textbf{Thanksgiving} are my two favourite holidays!}
		\item People's titles\\
				e.g. \textit{Everything depends on \textbf{President} Trump and his decisions.}
		\item The names of books, films, plays, paintings. We use capital letters for the nouns, adjectives, and verbs in the title.\\
				e.g. \textit{I've just finished reading \textbf{'The Old Man and the Sea'}}
\end{enumerate}
Sometimes we use a person's name to refer to something they have created.
\begin{center}
\textit{We were listening to \textbf{Mozart} the other day.}\\
\textit{I'm reading \textbf{an Iris Murdoch} now.}
\end{center}

When you use a word about a family member (e.g. \textit{mom, dad, uncle}), capitalize it only if the word is being used exactly as you would use a name, i.e. if you were addressing the person directly. If the word is not being used as a name, it is not capitalized.
\begin{center}
		\textit{Please ask \textbf{Dad} if he can buy wine on his way home.}\\
		\textit{Is your \textbf{dad} coming over for dinner?}
\end{center}
Whenever you see a capitalized word, question whether or not it is a proper noun. Make sure that the capitalized word is in fact a noun as there are also proper adjectives.
\begin{center}
		\textit{ \textbf{Asia} is one of the continents of the wold.} (proper noun)\\
		\textit{I don't like \textbf{Asian} food.} (proper adjective).
\end{center}

\subsection{Material Nouns}
\textbf{Material nouns} denote a material or substance from which things are made of.
\begin{center}
		\textit{		a \textbf{plastic} bottle, a \textbf{diamond} ring, etc.}
\end{center}
Material nouns are uncountable, thus they do not have a plural form. Generally, articles are not used with material nouns as they are uncountable.
\begin{center}
\textit{
		\textst{I really want to buy these cottons pants.}\\
I really want to buy these \textbf{cotton} pants.}
\end{center}

Material nouns fall into several categories:
\begin{center}
		\begin{enumerate}[label=\alph*)]
		\item Related to nature\\
				e.g. air, salt, coal, silver, gold, etc.
		\item Related to animals\\
				e.g. meat, milk, egg, wool, etc.
		\item Related to plants\\
				e.g. cotton, coffee, tea, wood, etc.
		\item Arificial or man-made materiales\\
				e.g. alcohol, cheese, brick, steel, etc.
		\end{enumerate}
\end{center}

\subsection{Compund nouns}
A \textbf{compund noun} contains two or more words which are joined together and form a single noun. Compund nouns can be words written together, words that are hyphenated, or separate words.\\
The first word usually describes or modifies the second word, denoting its type or purpose, Consequently, the second word identifies the item itself.
\begin{center}
\textit{
I need to buy a new \textbf{toothbrush}.} ( a brush used for cleaning one's teeth)
\end{center}

There is no exact rule as to when we should write compund nouns together, hyphenated, or as separate words. If you are not sure how to write a compund nound, \textbf{consult a dictionary}.
\begin{center}
\textit{
		Could you go with me to the \textbf{bus stop}?\\
		My \textbf{in-laws} are incredible people.\\
I love your new \textbf{haircut}! You look fantastic!}
\end{center}
Note that the stress usually falls on the first syllable in compund nouns. As a result, the word stress helps to differentiate between a compund noun and an adjective + noun.
\begin{center}
\textit{
A \textbf{greenhouse} is a glass building used for growing plants that need warmth, light, and protection.} (compund noun)\\
\textit{
A \textbf{green house} is a building that someone lives in. This building is painted green.} (adjective + noun)
\end{center}

\subsection{Countable vs Uncountable Nouns}
\begin{table}[h]
\begin{center}
\begin{tabular}{|c|c|}
		\hline
		\textbf{Countable Nouns (e.g. apple, song, house, etc.)} & \textbf{Uncountable Nouns (e,g, tea, money, love, etc.)}\\
		\hline
		\parbox[t]{8cm}{\vspace{0.08cm} Things that \textbf{can be counted}, even if the number\\ might be extremely high ( \textit{e.g. all the people in\\ the world}).\vspace{0.08cm}} & \parbox[t]{8cm}{\vspace{0.08cm}Things that we \textbf{cannot count} with numbers.\\ They may be the names for abstract ideas or\\ qualities or for physical objects that are too\\ small to count or shapeless ( \textit{e.g. liquids, gases, etc.}).\vspace{0.4cm}}\\
		\hline
		\parbox[t]{8cm}{\vspace{0.08cm}Can be singular or plural.\\ \textit{I have an \textbf{apple} and you have three \textbf{apples}.}\vspace{0.08cm}} & \parbox[t]{8cm}{ \vspace{0.08cm}No plural form.\\\textit{We're goint to have \textbf{rice} for lunch.}\vspace{0.4cm}}\\
		\hline
		\parbox[t]{8cm}{\vspace{0.08cm}You can use \textit{a/an} with singular countable nouns.\\ \textit{There is \textbf{a girl} outside. She is wearing \textbf{a beautiful dress}.}\vspace{0.08cm}} & \parbox[t]{8cm}{\vspace{0.08cm}You can't use \textit{a/an} wih uncountable nouns. But \\ you can often use the phrase \textit{a (bag, cup, etc.) of}.\\ \textit{There is \textbf{a bowl of rice} and \textbf{a bottle of juice} on the table.} \vspace{0.4cm}}\\
		\hline
\parbox[t]{8cm}{\vspace{0.08cm}If you want to ask about the quantity of a\\ countable noun, you ask \textit{'How many?'} combined \\ with the plural countable noun.\\ \textit{ \textbf{How many dogs} are there? - There are \textbf{five dogs.}}\vspace{0.08cm}} & \parbox[t]{8cm}{\vspace{0.08cm}If you want to ask about the quantity of an \\ uncountable noun, you ask \textit{'How much?'}\\ combined with the uncountable noun.\\ \textit{ \textbf{How much coffee} do we have left? - We don't have \textbf{much coffee} left.}\vspace{0.4cm}} \\
\hline
\parbox[t]{8cm}{\vspace{0.08cm}You can use \textit{many, a few, few} with plural\\ countable nouns.\\ \textit{Sorry, but I didn't take \textbf{many pictures.}\\I've got \textbf{a few relatives} leaving here.}\vspace{0.08cm}} & \parbox[t]{8cm}{{\vspace{0.08cm}You can use \textit{much, a little, little} with uncountable\\ nouns.\\ \textit{We didn't do \textbf{much shopping} there. \\ We have \textbf{a little sugar} left.} }\vspace{0.4cm}} \\
\hline
\multicolumn{2}{|c|}{ \parbox[t]{16cm}{\begin{center}You can use \textit{some, any, a lot of, both} with plural countable nouns and uncountable nouns. \end{center}}} \\
\hline
\textit{We like singing \textbf{some crazy songs} at karaoke.} & \textit{We listened to \textbf{some music} there.}\\
\hline
\textit{Did you buy \textbf{any oranges}?} & \textit{I didn't buy \textbf{any orange juice}.}\\
\hline
\textit{She showed \textbf{a lot of signs} of affection.} & \textit{There is \textbf{a lot of love} in the air.}\\ \hline

\end{tabular}
\end{center}
\caption{Countable vs Uncountable Nouns} \label{tab:nouns2}
\end{table}

\subsection{Collective Nouns}
A collective noun is used to refer to an entire group of people, animals, or things.\\
Therefore it includes more than one member.

\begin{center}
		\textit{My \textbf{family} is very big.}
\end{center}
\newpage
Collective nouns can refer to:
\begin{enumerate}[label=\alph*)]
		\item People\\
				e.g. \textit{family, class, committee, staff, etc.}
		\item Animals\\
				e.g. \textit{a pack of dogs, a swarm of flies, a herd of horses, a litter of puppies, etc.}
		\item Things\\
				e.g. \textit{pack, set, bunch, stack, etc,}
\end{enumerate}
When the members within one group behave in the same manner, they are part of a collective noun, thus this noun becomes singular and requires a singular verb.
\begin{center}
		\textit{Every day \textbf{the football team} follows its coach out to the field for practice.}
\end{center}

When the members are acting as individuals, the collective noun is plural and requiresa plural verb.\\
In many cases, it may sound more natural to make the subject plural in form by adding words like \textit{members, mates, etc.}
\begin{center}
		\textit{After the practice \textbf{the team(mates)} shower, change into their casual clothes, and head to their homes.}
\end{center}

\subsection{Concrete and Abstract Nouns}
\hspace{0.8cm} Nouns can be concrete or abstract. \\
\textbf{Concrete nouns} are tangible and you can experience them with your five senses.\\
\textbf{Abstract nouns} refer to intangible things, like \textit{actions, feelings, ideals, concepts, and qualities}.
\begin{center}
		\textit{ \textbf{Food} is great. But \textbf{love} is even greater.}
\end{center}
\subsubsection{Concrete nouns}
A \textbf{concrete noun} is a noun that can be identified through one of the five senses: \textit{touch, sight, hearing, smell, or taste}.
\begin{center}
		\textit{Who turned off the \textbf{TV}?} (The noun \textit{TV} is a concrete noun)\\
		\textit{What is that \textbf{noise}?} (Even though \textit{nose} can't be touched, you can hear it, so it's a concrete noun)
\end{center}
Concrete nouns fall into several categories:
\begin{enumerate}[label=\alph*)]
		\item People\\
				e.g. \textit{mother, friend, teacher, stranger, etc}.
		\item Places\\
				e.g. \textit{school, McDonald's, Las Vegas, India, etc}.
		\item Things you can touch and see\\
				e.g. \textit{plane, cup, lamp, book, etc}.
		\item Things you can hear\\
				e.g. \textit{music, noise, someone's voice, song, etc.}
		\item Things you can smell and taste\\
				e.g. \textit{herbs, cookies, bread, wine, etc.}
\end{enumerate}

\subsubsection{Abstract Nouns}
Remember that \textbf{abstract nouns} refer to a intangible things, like \textit{actions, feelings, ideals, concepts, and qualities}.\\
Abstract nouns fall into several categories:
\begin{enumerate}[label=\alph*)]
		\item Emotions and feelings\\
				e.g. \textit{anger, sadness, love, grief, etc.}
		\item Human qualities and characteristics\\
				e.g. \textit{beauty, maturity, humour, patience, etc.}
		\item Ideas and concepts\\
				e.g. \textit{knowledge, freedom, luxury, comfort, etc.}
		\item Events\\
				e.g. \textit{marriage, birthday, career, adventure, etc.}
\end{enumerate}

Many abstract nouns are formed from adjectives, verbs, or nouns. Sometimes you can add a suffix to the concrete noun or alter the word root to form abstract nouns.
\begin{center}
		\inline{ \textit{child}}{(Concrete noun)} \inline{ \textit{child\textbf{hood}}}{(abstract noun)}
\end{center}
Nouns with the following suffixes are often abstract:
\begin{table}[h]
\begin{center}
\begin{tabular}{|c|c|c|c|c|c|}
		\hline
		\multicolumn{2}{|c|}{\textbf{-tion} e.g. \textit{devotion}} & \multicolumn{2}{|c|}{\textbf{-ism} e.g. \textit{pessimism}} & \multicolumn{2}{|c|}{\textbf{-ity} e.g. \textit{hospitality}}\\ \hline
		\multicolumn{2}{|c|}{\textbf{-ment} e.g. \textit{movement}} & \multicolumn{2}{|c|}{\textbf{-ness} e.g. \textit{restlesness}} & \multicolumn{2}{|c|}{\textbf{-age} e.g. \textit{marriage}} \\ \hline
		\multicolumn{2}{|c|}{\textbf{-ance} e.g. \textit{brilliance}} & \multicolumn{2}{|c|}{\textbf{-ence} e.g. \textit{indifference}} & \multicolumn{2}{|c|}{\textbf{-ship} e.g. \textit{relationship}} \\ \hline
		\multicolumn{3}{|c|}{\textbf{-ability} e.g. \textit{vailability}} & \multicolumn{3}{|c|}{\textbf{-acy} e.g. \textit{bureacracy}}
\end{tabular}
\end{center}
\caption{Common suffixes for abstract nouns} \label{tab:nouns3}
\end{table}

\subsection{Possessive Nouns}
The \textbf{Possessive} form is used with \textbf{nouns} referring to people, groups of people, countries, and animals.\\
It shows a relationship of belonging between one thing and another.
\begin{center}
		\textit{ \textbf{Lelie's} aunt is a doctor.}
\end{center}
To form the possessive, add an \textbf{apostrophe + -s} to the noun.
\begin{center}
\textit{My \textbf{brother's} computer was stolen a week ago.\\
		\textbf{Children's} toys were on the ground.\\}
\end{center}
If the noun already \textbf{ends in -s}, just add an \textbf{apostrophe}.
\begin{center}
\textit{ \textbf{Student's} homework will be assessed later.}
\end{center}
For names \textbf{ending in -s}, you can either add an \textbf{apostrophe + -s}, or just an \textbf{apostrophe}. The first option is more common.
\begin{center}
\textit{They want to sell \textbf{Jame's }car.}
\end{center}
Study some of the fixed expressions where the possessive form is used.
\begin{center}
\textit{a day's work, a month's pay, in a year's time, for God's sake}
\end{center}
Note that the possessive is also used to refer to \textit{shops, restaurants, churches, universities, etc.}, using the name or job title of the owner.
\begin{center}
\textit{
		I want to go to \textbf{Luigi's} for dinner.\\
		Peter has an appointment \textbf{at the dentist's} at 10 a.m.
}
\end{center}
\newpage
\section{Pronouns}
A \textbf{pronoun} is a word that replaces a noun in a sentence, making the subject a person or a thing
\subsection{Subject Pronouns}
A \textbf{subject} is the person or thing that performs the action in the clause or sentence.

A \textbf{subject pronoun} is a pronoun that takes the place of a noun as the subject of a sentence

\begin{center}
\textit{
\textbf{She} told me about her worries.}
\end{center}
Subject pronouns replace nouns that are the subject of their clause.
\begin{table}[h]
\begin{center}
		\begin{tabular}{|c|c|c|}
		\hline
		& \textbf{Singular} & \textbf{Plural} \\
		\hline
		$1^{st}$ person & I & we \\ \hline
		$2^{nd}$ person & you & you \\ \hline
		$3^{rd}$ person & he/she/it & they \\ \hline
	\end{tabular}
\end{center}
\caption{\label{tab:nouns4}Singular and plural forms for subject pronouns}
\end{table}

We should replace the subject with a subject pronoun to avoid repetition.
\begin{center}
\textit{
\textst{Mary is a student and Mary is very hard working.}\\
Mary is a student and \textbf{she} is very hard working.}
\end{center}

We use the subject pronoun \textit{it} when we refer to objects, things, animals, or ideas.
\begin{center}
		\textit{ Love is eternal. \textbf{It} will last forever.}
\end{center}

Sometimes when we don't know the sex of a baby, we can use \textit{it}'.
\begin{center}
\textit{
Their baby is so small. \textbf{It} only weights 2 kilos.}
\end{center}
\hspace{0.4cm} We use \textit{it} when we talk about \textit{time, weather, or temperature}.
\begin{center}
\textit{
	What time is \textbf{it}? - \textbf{It}'s 7 o'clock.\\
	\textbf{It}'s quite cold today.}
\end{center}

\subsection{Object Pronouns}
An \textbf{object} is the person or thing that receives the action in the clause or sentence.\\
An \textbf{object pronoun} is a pronoun that takes the place of a noun as the object of a sentence.
\begin{center}
\textit{
She told \textbf{me} about her worries.}
\end{center}

Object pronouns are used to replace nouns that are the direct or indirect object of a clause.
\begin{table}[h]
\begin{center}
\begin{tabular}{|c|c|}
	\hline
	\textbf{Subject} & \textbf{Object} \\
	\hline
	I & me \\ \hline
	you & you \\ \hline
	he & him \\ \hline
	she & her \\ \hline
	it & it \\ \hline
	we & us \\ \hline
	they & them \\
	\hline
\end{tabular}
\end{center}
\caption{\label{tab:nouns5}Subject and Object Pronouns}
\end{table}

Object pronouns come either after a verb or a preposition.
\begin{center}
\textit{
Ethan asked \textbf{me} to talk to \textbf{them}.}
\end{center}

Note that the subject pronoun \textit{it} and the object pronoun \textit{it} look the same.

\begin{center}
\textit{
Do you know the movie 'Pertty Lady'? \textit{it} is my favourite!} (subject pronoun)\\
\textit{ I've seen \textit{it} many times.} (object pronoun)
\end{center}
Remember that object nouns are always the recipients of the action in sentence.
\begin{center}
\textit{
		\textst{He and me went to the movies}. \textbf{He and I} went to the movies.\\
\textst{Mrs. Keith called her and I}. Mrs. Keith called \textbf{her and me}.}
\end{center}

We should replace the object with an object pronoun to avoid repetition.
\begin{center}
\textit{
I can't stop thinking about Amy. \textst{I can't stop imagining my future with Amy}. I can't stop imagining my future with \textbf{her}.}
\end{center}

\subsection{Possessive Pronouns}
\textbf{Possessive pronouns} are pronouns that demostrate ownership.
\begin{center}
\textit{This car is \textbf{mine}.}
\end{center}
Possessive pronouns are used instead of a possessive adjective and noun. Study the following table:
\begin{table}[h]
\begin{center}
\begin{tabular}{|c|c|c|c|}
		\hline
		\textbf{Subject} & \textbf{Object} & \textbf{Possessive Adjective} & \textbf{Possessive Pronoun}\\ \hline
		I & me & my & mine \\ \hline
		you & you & your & yours \\ \hline
		he & him & his & his \\ \hline
		she & her & her & hers \\ \hline
		it & it & its & its \\ \hline
		they & them & their & theirs\\ \hline
\end{tabular}
\end{center}
\caption{Possessive Adjectives \& Possessive Pronouns} \label{tab:nouns6}
\end{table}

\section{Articles}
\textbf{Articles} are words that define a noun as specific or unspecific.\\
English has two types of articles:
\begin{itemize}
		\item Indefinite: \textit{a/an}
		\item Definite: \textit{the}
\end{itemize}
\begin{center}
\textit{I'm \textbf{a} nurse. \textbf{The} hospital I'm working in is huge.}
\end{center}

\subsection{Indefinite Article}
The \textbf{indefinite article} takes two forms: \textbf{a/an}. Use the indefinite article \textbf{a} when it precedes a word that \textbf{begin} with a \textbf{consonant}. Use the indefinite article \textbf{an} when it precedes a word that \textbf{begins} with a \textbf{vowel}.
\begin{center}
		\textit{ \textbf{a} table, \textbf{an} umbrella, \textbf{a} university, \textbf{an} honest person.}
\end{center}
The indefinite article \textbf{a/an} indicates that a noun refers to a general idea rather than a particular thing.
\begin{center}
\textit{What does \textbf{a} fox say?}
\end{center}
We use \textbf{a/an} when the listener does not know which person or thing we are talking about.
\begin{center}
		\textit{Helen's brother works in \textbf{a} factory. I don't know which factory exactly.}
\end{center}
\newpage
If we refer to something for the first time, it will be new information for the listener so we use \textbf{a/an}.\\
When referencing to the same thing again use \textbf{the} because now the listener knows what we are talking about.
\begin{center}
\textit{I bought \textbf{a} new computer. It's really great! \textbf{The} computer is much better than my previous one.}
\end{center}

\subsection{Definite Article}
The \textbf{definite article} is the word \textbf{the}. It limits the meaning of a noun to one particular thing. We use \textbf{the} when it is clear which thing or person we are talking about.
\begin{center}
\textit{ \textbf{The} cake is in the fridge. I know that Kate made it.}
\end{center}

We use the definite article \textbf{the} with:
\begin{enumerate}[label=\alph*)]
\item Nationalities and other groups\\
		e.g. \textit{ \textbf{the} French, \textbf{the} Italians, \textbf{the} old, \textbf{the} poor.}
\item Time\\
		e.g. \textit{in \textbf{the} past, in \textbf{the} future (but: \textbf{at present}.)}
\item Superlatives\\
		e.g. \textit{You are \textbf{the} first one!}
\item Musical instruments\\
		e.g. \textit{I played \textbf{the} piano as a kid.}
\item Countries which are a group or plual\\
		e.g. \textit{ \textbf{the} U.S., \textbf{the} U.K., \textbf{the} United Arab Emirates, \textbf{the} Netherlands }
\item Names of ship.\\
		e.g. \textit{We sailed on \textbf{the} Claudia}
\item Oceans\\
		e.g. \textit{ \textbf{the} Pacific, \textbf{the} Atlantic}
\item Rivers\\
		e.g. \textit{ \textbf{the} Amazon, \textbf{the} Nile}
\end{enumerate}

Note that we use \textbf{zero article} with \textbf{plurals} and \textbf{uncountable nouns} when we are generally talking about something.
\begin{center}
		\textit{ \textbf{Dogs} are not allowed in that shop.} (We are talking about dogs in general.)\\
		\textit{ \textbf{The dogs} next door were barking at night.} (W are talking about the particular dogs.)
\end{center}

\section{Demonstratives}

\textbf{Demonstratives} are words that show which person or thing is being referred to.
Demonstratives show where an object, event, or person is in relation to the speaker. They can refer to a physical or a psychological closeness or distance.
\begin{center}
		\textit{\textbf{This} is Hugh, and \textbf{that} is Kevin.}
\end{center}

\begin{table}[h]
\begin{center}
\begin{tabular}{|c|c|c|}
		\hline
		     	 & \textbf{Near the speaker} & \textbf{Far from the speaker} \\ \hline
		Adverbs  & here & there \\ \hline
		Demonstratives with singular and uncountable nouns & this & that \\ \hline
		Demonstratives with plurar countable nouns & these & those\\ \hline
\end{tabular}
\end{center}
\caption{\label{tab:nouns7}Demonstratives}
\end{table}
\newpage
Demonstratives can be placed before the noun or the adjective that modifies the noun.
\begin{center}
\textit{ \textbf{That old man} stole my purse!\\
\textbf{These oranges} are delicious!}
\end{center}
Demonstratives can also appear before a number by itself when the noun is understood from the context.
\begin{center}
\textit{I'll take \textbf{this one}, please.} = \textit{I'll take this watermelon, please.}
\end{center}
Demonstratives can be used by themselves when the noun they modify is understood from the context.
\begin{center}
\textit{ \textbf{Those} aren't yours. Put them back.} = \textit{Those shoes aren't yours. Put them back.}
\end{center}
When talking about events, the \textbf{near demonstratives} are often used to refer to the \textbf{present} while the \textbf{far demonstratives} often refer to the \textbf{past}.
\begin{center}
\textit{ \textbf{This situation} is quite unstable. \\
\textbf{That event} made me realise how important my family is to me.}
\end{center}

\section{Distributives}
\textbf{Distributives determiners} or simply \textbf{distributives} refer to a group of people or things, and to individual members of the group.\\
\indent They show different ways of looking at the individuals within a group, and they express how something is distributed, shared, or divided.
\begin{center}
\textit{ \textbf{All people} want to love and to be loved.\\
\textbf{Each} person is unique. \textbf{Every} person is unique.\\
\textbf{Both of us} like Mexican food.}
\end{center}

\subsection{All}
The distributive determinal \textbf{all} is used to talk about a whole group, with a special emphasis on the fact that nothing has been left out.\\
\indent \textbf{All} can beused with uncountable nouns and plural countable nouns by itself. In this usage, it refers to the group as a concept rather than as individuals.
\begin{center}
\textit{ \textbf{All parents} want the best for their childre.}
\end{center}
\textbf{All} can be used with uncountable nouns and plural countable nouns preceded by \textbf{the} or a \textbf{possessive adjective}. In these uses, the word \textbf{of} can be added just after \textbf{all} with no change in meaning.
\begin{center}
\textit{ Have you eaten \textbf{all the cookies} in the jar?} = \textit{Have you eaten \textbf{all of the cookies} in the jar?.}
\end{center}
\textbf{All} can be used with \textbf{plural pronouns} preceded by \textbf{of}.
\begin{center}
\textit{ \textbf{All of us} are going to be there tonight.}
\end{center}
\textbf{All} can be used in questions and exclamations with \textbf{uncountable nouns} preceded by \textbf{this/that} or with \textbf{countable nouns} preceded by \textbf{hese/those}.  In these uses, the word \textbf{of} can be added just after \textbf{all} with no change in meaning.
\begin{center}
\textit{ Look at \textbf{al this snow} out there!\\
What are \textbf{all these people} doing in our house?}
\end{center}

\subsection{Half}
The distributive determiner \textbf{half} is used to talk about a whole group divided in \textbf{two}. \textbf{Half} can be used as a distributive in several different patterns.\\
\textbf{Half} can refer to measurements if it is followed by an indefinite article \textbf{a/an} and a noun.
\begin{center}
\textit{I'll be back in \textbf{half an hour}.}
\end{center}
\textbf{Half} can be used with plural pronouns preceded by \textbf{of}.
\begin{center}
\textit{ \textbf{Only half of us} are going to be there tonight.}
\end{center}
\textbf{Half} can be used with nouns preceded by \textbf{the, a/b, a demonstrative, or a possessive adjective}. In this case, the meaning refers to a concrete, physical division.
The word \textbf{of} can be added just after \textbf{half} with no change in meaning.
\begin{center}
\textit{ \textbf{Half the people} have already left the party.\\
Putting \textbf{half a kilo of sugar} into the topping will ruin the cake.\\
I want \textbf{half of that cake}!\\
Sorry, but I used \textbf{half of your eggs} making breakfast today.}
\end{center}

\subsection{Each and Every}
The distributives \textbf{each} and \textbf{every} are both related to describing the members of a group. These distributives can only be used with \textbf{countable nouns} by being placed before the nouns. \\
\indent In many cases, they are interchangable but there is a \textbf{subtle difference} between them.
\subsubsection{Each}
\textbf{Each} is used to describe and highlight an individual member of a group, or multiple individuals. By using \textbf{each} you recognise the item is a part of a group, but that it also needs to be pointed out as a singular item too.
\begin{center}
\textit{ \textbf{Each book} on the shelf had a unique cover.}
\end{center}
\textbf{Each} can be used with plural nouns and pronouns but \textbf{must} be followed by \textbf{of}.
\begin{center}
\textit{ \textbf{Each of the pupils} received a Christmas card.}
\end{center}
\textbf{Each} can be used after the subject or at the end of a sentence.
\begin{center}
\textit{ \textbf{My siblings each} have their own room.\\
My mother gave my sister and I \$20 \textbf{each}.} = (gave \$20 to each of us.)
\end{center}
\subsubsection{Every}
\textbf{Every} by contrast is a way of referring to the group as a collection of individual members. \textbf{Every} cannot be used with plural nouns.
\begin{center}
\textit{ \textst{ \textbf{Every boys} in my class wanted that computer game.} \textbf{Every boy} in my class wanted that computer game.}
\end{center}
\textbf{Every} can express different points in a series, especially with time expressions.
\begin{center}
\textit{ \textbf{Every} morning Phillip goes for a run.\\
And \textbf{every time} Ann would forgive him.}
\end{center}

\subsection{Both}
\textbf{Both} refers to the whole pair and is equivalent to \textit{'one and the other'}. \textbf{Both} can be used with plural nouns on its own, or it can be followed by \textbf{of}, with \textbf{of} without an article. When followed by a plural pronoun, \textbf{both} must be separated from the pronoun by \textbf{of}.
\begin{center}
		\textit{ \textbf{Both (of) my parents} approve of me going to college.\\
		I told \textbf{both of them} to give me a call.}
\end{center}
\textbf{Both} cannot be used with singular nouns, because it refer to two thigns.
\begin{center}
\textit{ \textst{Both my sister likes travelling.} \textbf{Both my sisters like travelling.}}
\end{center}

\newpage
\subsection{Either}
\textbf{Either} is positive and when used alone refers to one of the two members of the pair. It is equivalent to \textit{'one or the other'}. Because it refers to just one member of a pair, \textbf{either} must be used before a singular noun. It can also be used with a plural noun or pronoun if followed by \textbf{of}.
\begin{center}
		\textit{ \textbf{Either day} is fine.\\
		We could stay at \textbf{either of the hotels}.}
\end{center}
\textbf{Either} can also be used with \textbf{or} in a construction that talks avout each member of the par in turn. The meaning reamins the same, but in this case \textbf{either} is not functioning as a distributive. It is functioning as a \textbf{conjunction}.
\begin{center}
\textit{ You can have \textbf{either} ice cream \textbf{or} cake.}
\end{center}

\subsection{Neither}
\textbf{Neither} is negative and when used alone refer to the whole pair. It is equivalent to \textit{'not one or the other'}. Because it refers to just one member of a pair, \textbf{neither} must be used before a singular noun. It cal also be used with a plural noun or pronoun if followed by \textbf{of}.
\begin{center}
\textit{ \textbf{Neither date} is convenient for me.\\
\textbf{Neither of these dresses} suits her.}
\end{center}
\textbf{Neither} can also be used with \textbf{nor} in a construction that talks about each member of the pair it turn. The meaning reamins the same, but in this case \textbf{neither} is not functioning as a distributive. It is functioning as a \textbf{conjunction}.
\begin{center}
\textit{ It is \textbf{neither} snowing \textbf{nor} raining.}
\end{center}

\section{Quantifiers}





\end{document}
