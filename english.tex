\documentclass[hidelinks,10pt,a4paper]{article}
\usepackage[utf8]{inputenc}
\usepackage[english]{babel}
\usepackage{amsmath}
\usepackage{amsfonts}
\usepackage{amssymb}
\usepackage{soul}
\usepackage[margin=1in]{geometry}
\usepackage{enumitem}
\usepackage{adjustbox}
\usepackage[driverfallback=hypertex]{hyperref}
\usepackage{multirow}

\newlength{\drop}

\usepackage{tikz}

\newcommand{\inline}[2]{%
    \begin{tikzpicture}[baseline=(word.base), txt/.style={shape=rectangle, inner sep=0pt}]% the baseline key ensures that nodes won't shift up if there's text with descenders, and the txt style removes extra spacing so you can use this inline
    \node[txt] (word) {#1};% the first argument is the contents of the main node
    \node[above] at (word.north) {\footnotesize{#2}};% the second argument is the tag; you can play with the positioning as necessary
    \end{tikzpicture}%
    }

\begin{document}

\begin{titlepage}

\drop=0.1\textheight
    \centering
    \vspace*{\baselineskip}
    \rule{\textwidth}{1.6pt}\vspace*{-\baselineskip}\vspace*{2pt}
    \rule{\textwidth}{0.6pt}\\[\baselineskip]
    {\LARGE ENGLISH\\[0.2\baselineskip] NOTES}\\[0.2\baselineskip]
    \rule{\textwidth}{0.4pt}\vspace*{-\baselineskip}\vspace{3.2pt}
    \rule{\textwidth}{1.6pt}\\[\baselineskip]
    \scshape
    From A1 to C2 \\
    \vspace*{2\baselineskip}
    Edited by \\[\baselineskip]
    {\Large MAXIMILIANO PONCE\par}
    {\itshape My notes from \\Langpill english grammar course from Udemy, and internet\\\par}
    \vfill
    {\scshape April 2020} \\
    {\large MAXIMILIANO PONCE}\par

\end{titlepage}

\tableofcontents
\newpage

\section{Introduction}
\indent
This document was written for me to understand the English grammar from a beginner to an advanced level. Because I find that I can understand better if I take notes for myself while I'm watching English lessons, and I decided to make my notes with \LaTeX \hspace{0.05cm} because it is a useful and cool tool for writing awesome documents.\\

\indent I know that I need to improve my English writing and as I improve it I will update this document. I hope my notes become useful to other people so this document is divided in the main topics of English grammar, so that you can go directly to the topic that you want to learn.\\

\indent
Remember that is important to practice on your own to master the English language, another advice for you is to speak with a friend of you who wants to learn English or already knows it, or even to your self, about any topic you like and if you are stuck to explain something then you can search on Internet or ask to your friend to get feedback.\\

\newpage

\section{Nouns}
A \textbf{noun} is a word that is used to name a person, animal, place, action or thing either generally (\textbf{common noun}) or specifically ( \textbf{proper noun)}.
\begin{center}
		\textit{ You can buy a} \inline{ \textit{\textbf{pencil}}}{Common noun}
		\textit{at} \inline{\textit{\textbf{Office Depot}}}{Proper noun}.
\end{center}

\subsection{Plural Form}
There are rules to form plural forms, but remember that some nouns are irregular so that you can't use those rules to form the plural form.

\begin{itemize}
		\item Generally we can use $\Rightarrow$ \underline{Noun $+$ S}, to get the plural form \\e.g.\textit{cat-cats, dog-dogs, pencil-pencils}
		\item Singular noun ending in \underline{$s$, $ss$, $sh$, $ch$, $x$, $z$, $o$ $+$ $es$}, $\Rightarrow$ Plural form \\ e.g. \textit{ tax-taxes, bus-buses, box-boxes}
		\item In some cases, singular nouns ending in \underline{$s$ or $z$} $\Rightarrow$ Double $s$ or $z$ \\
			e.g \textit{fez-fezzes, gas-gasses}
		\item Noun ending in \underline{$f$, $fe$} $\Rightarrow$ Change $f$ into $ve$ $+$ $s$ \\e.g. \textit{life-lives, wolf-wolves, wife-wives} \quad (Exceptions: \textit{roof-roofs, belief-beliefs, chef-chefs})
		\item Noun ending in $y$ and the letter before is a \underline{consonant} $\Rightarrow$ Change the ending to $ies$ \\
			e.g. \textit{city-cities, puppy-puppies}
		\item Noun ending in $y$ and the letter before is a \underline{vowel} $\Rightarrow$ Add $s$ \\
			e.g. \textit{ray-rays, boy-boys}
		\item Noun ending in $o$ $\Rightarrow$ Add $es$ \\
			e.g. \textit{potato-potatoes, tomato-tomatoes} \quad (Exceptions: \textit{photo-photos, piano-pianos, halo-halos}
		\item Noun ending in $us$ $\Rightarrow$ Frequently change $us$ to $i$ \\
			e.g. \textit{cactus-cacti, focus-foci}
		\item Noun ending in $is$ $\Rightarrow$ Change $is$ to $es$ \\
			e.g. \textit{analysis-analyses, ellipsis-ellipses}
		\item Noun ending in $on$ $\Rightarrow$ Change $on$ to $a$ \\
			e.g. \textit{phenomenon-phenomena, criterion-criteria}
		\item Some nouns \underline{don't change} \\
			e.g. \textit{sheep-sheep, series-series, species-species}
\end{itemize}
Here we have some \textbf{irregular nouns}, they don't follow specific rules\\
\begin{table}[h]
\begin{center}
\begin{tabular}{|c|c|}
\hline
\textbf{Singular} & \textbf{Plural}\\
\hline
man & men\\ \hline
woman & women\\\hline
person & people\\\hline
child & children\\\hline
tooth & teeth\\\hline
foot & feet\\\hline
mouse & mice\\
\hline
\end{tabular}
\end{center}
\caption{\label{tab:nouns1}Some irregular nouns}
\end{table}


\subsection{Common Nouns}
\textbf{Common nouns} refer to classes or categories of people, animals, places, things, or a concept, as opposed to a particular individual.
\begin{center}
	\textit{I have a \textbf{computer}, a \textbf{keyboard}, a \textbf{mouse} and many \textbf{books}. }
\end{center}
\indent
\textbf{Common nouns} are \textbf{not capitalized} unless they begin a sentence or are part of a title.
\begin{center}
		\textit{
		\textbf{Apples} are delicious fruits\\
				I don't like \textbf{apples} }
\end{center}

\subsection{Proper Nouns}
\textbf{Proper nouns} are used to name to specific items rather than refer to a category or a class, such as names, names of cities, countries, etc.
\begin{center}
		\textit{I'm from \textbf{Mexico} }
\end{center}
Note that proper nouns are unique names. \textbf{They are capitalized}
\begin{center}
	\textit{My friend \textbf{George} is from \textbf{Brazil}  }
\end{center}
We should also capitalize:
\begin{enumerate}[label=\alph*)]
		\item Festivals\\
				e.g. \textit{\textbf{Christmas} and \textbf{Thanksgiving} are my two favourite holidays!}
		\item People's titles\\
				e.g. \textit{Everything depends on \textbf{President} Trump and his decisions.}
		\item The names of books, films, plays, paintings. We use capital letters for the nouns, adjectives, and verbs in the title.\\
				e.g. \textit{I've just finished reading \textbf{'The Old Man and the Sea'}}
\end{enumerate}
Sometimes we use a person's name to refer to something they have created.
\begin{center}
\textit{We were listening to \textbf{Mozart} the other day.}\\
\textit{I'm reading \textbf{an Iris Murdoch} now.}
\end{center}

When you use a word about a family member (e.g. \textit{mom, dad, uncle}), capitalize it only if the word is being used exactly as you would use a name, i.e. if you were addressing the person directly. If the word is not being used as a name, it is not capitalized.
\begin{center}
		\textit{Please ask \textbf{Dad} if he can buy wine on his way home.}\\
		\textit{Is your \textbf{dad} coming over for dinner?}
\end{center}
Whenever you see a capitalized word, question whether or not it is a proper noun. Make sure that the capitalized word is in fact a noun as there are also proper adjectives.
\begin{center}
		\textit{ \textbf{Asia} is one of the continents of the wold.} (proper noun)\\
		\textit{I don't like \textbf{Asian} food.} (proper adjective).
\end{center}

\subsection{Material Nouns}
\textbf{Material nouns} denote a material or substance from which things are made of.
\begin{center}
		\textit{		a \textbf{plastic} bottle, a \textbf{diamond} ring, etc.}
\end{center}
Material nouns are uncountable, thus they do not have a plural form. Generally, articles are not used with material nouns as they are uncountable.
\begin{center}
\textit{
		\textst{I really want to buy these cottons pants.}\\
I really want to buy these \textbf{cotton} pants.}
\end{center}

\newpage
Material nouns fall into several categories:
\begin{center}
		\begin{enumerate}[label=\alph*)]
		\item Related to nature\\
				e.g. air, salt, coal, silver, gold, etc.
		\item Related to animals\\
				e.g. meat, milk, egg, wool, etc.
		\item Related to plants\\
				e.g. cotton, coffee, tea, wood, etc.
		\item Artificial or man-made materials\\
				e.g. alcohol, cheese, brick, steel, etc.
		\end{enumerate}
\end{center}

\subsection{Compound Nouns}
A \textbf{compound noun} contains two or more words which are joined together and form a single noun. Compound nouns can be words written together, words that are hyphenated, or separate words.\\
The first word usually describes or modifies the second word, denoting its type or purpose, Consequently, the second word identifies the item itself.
\begin{center}
\textit{
I need to buy a new \textbf{toothbrush}.} ( a brush used for cleaning one's teeth)
\end{center}

There is no exact rule as to when we should write compound nouns together, hyphenated, or as separate words. If you are not sure how to write a compound noun, \textbf{consult a dictionary}.
\begin{center}
\textit{
		Could you go with me to the \textbf{bus stop}?\\
		My \textbf{in-laws} are incredible people.\\
I love your new \textbf{haircut}! You look fantastic!}
\end{center}
Note that the stress usually falls on the first syllable in compound nouns. As a result, the word stress helps to differentiate between a compound noun and an adjective + noun.
\begin{center}
\textit{
A \textbf{greenhouse} is a glass building used for growing plants that need warmth, light, and protection.} (compound noun)\\
\textit{
A \textbf{green house} is a building that someone lives in. This building is painted green.} (adjective + noun)
\end{center}

\newpage
\subsection{Countable vs Uncountable Nouns}
\begin{table}[h]
\begin{center}
\begin{tabular}{|c|c|}
		\hline
		\textbf{Countable Nouns (e.g. apple, song, house, etc.)} & \textbf{Uncountable Nouns (e,g, tea, money, love, etc.)}\\
		\hline
		\parbox[t]{8cm}{\vspace{0.08cm} Things that \textbf{can be counted}, even if the number\\ might be extremely high ( \textit{e.g. all the people in\\ the world}).\vspace{0.08cm}} & \parbox[t]{8cm}{\vspace{0.08cm}Things that we \textbf{cannot count} with numbers.\\ They may be the names for abstract ideas or\\ qualities or for physical objects that are too\\ small to count or shapeless ( \textit{e.g. liquids, gases, etc.}).\vspace{0.4cm}}\\
		\hline
		\parbox[t]{8cm}{\vspace{0.08cm}Can be singular or plural.\\ \textit{I have an \textbf{apple} and you have three \textbf{apples}.}\vspace{0.08cm}} & \parbox[t]{8cm}{ \vspace{0.08cm}No plural form.\\\textit{We're goint to have \textbf{rice} for lunch.}\vspace{0.4cm}}\\
		\hline
		\parbox[t]{8cm}{\vspace{0.08cm}You can use \textit{a/an} with singular countable nouns.\\ \textit{There is \textbf{a girl} outside. She is wearing \textbf{a beautiful dress}.}\vspace{0.08cm}} & \parbox[t]{8cm}{\vspace{0.08cm}You can't use \textit{a/an} with uncountable nouns. But \\ you can often use the phrase \textit{a (bag, cup, etc.) of}.\\ \textit{There is \textbf{a bowl of rice} and \textbf{a bottle of juice} on the table.} \vspace{0.4cm}}\\
		\hline
\parbox[t]{8cm}{\vspace{0.08cm}If you want to ask about the quantity of a\\ countable noun, you ask \textit{'How many?'} combined \\ with the plural countable noun.\\ \textit{ \textbf{How many dogs} are there? - There are \textbf{five dogs.}}\vspace{0.08cm}} & \parbox[t]{8cm}{\vspace{0.08cm}If you want to ask about the quantity of an \\ uncountable noun, you ask \textit{'How much?'}\\ combined with the uncountable noun.\\ \textit{ \textbf{How much coffee} do we have left? - We don't have \textbf{much coffee} left.}\vspace{0.4cm}} \\
\hline
\parbox[t]{8cm}{\vspace{0.08cm}You can use \textit{many, a few, few} with plural\\ countable nouns.\\ \textit{Sorry, but I didn't take \textbf{many pictures.}\\I've got \textbf{a few relatives} leaving here.}\vspace{0.08cm}} & \parbox[t]{8cm}{{\vspace{0.08cm}You can use \textit{much, a little, little} with uncountable\\ nouns.\\ \textit{We didn't do \textbf{much shopping} there. \\ We have \textbf{a little sugar} left.} }\vspace{0.4cm}} \\
\hline
\multicolumn{2}{|c|}{ \parbox[t]{16cm}{\begin{center}You can use \textit{some, any, a lot of, both} with plural countable nouns and uncountable nouns. \end{center}}} \\
\hline
\textit{We like singing \textbf{some crazy songs} at karaoke.} & \textit{We listened to \textbf{some music} there.}\\
\hline
\textit{Did you buy \textbf{any oranges}?} & \textit{I didn't buy \textbf{any orange juice}.}\\
\hline
\textit{She showed \textbf{a lot of signs} of affection.} & \textit{There is \textbf{a lot of love} in the air.}\\ \hline

\end{tabular}
\end{center}
\caption{Countable vs Uncountable Nouns} \label{tab:nouns2}
\end{table}

\subsection{Collective Nouns}
A collective noun is used to refer to an entire group of people, animals, or things.\\
Therefore it includes more than one member.

\begin{center}
		\textit{My \textbf{family} is very big.}
\end{center}

Collective nouns can refer to:
\begin{enumerate}[label=\alph*)]
		\item People\\
				e.g. \textit{family, class, committee, staff, etc.}
		\item Animals\\
				e.g. \textit{a pack of dogs, a swarm of flies, a herd of horses, a litter of puppies, etc.}
		\item Things\\
				e.g. \textit{pack, set, bunch, stack, etc.}
\end{enumerate}
\newpage
When the members within one group behave in the same manner, they are part of a collective noun, thus this noun becomes singular and requires a singular verb.
\begin{center}
		\textit{Every day \textbf{the football team} follows its coach out to the field for practice.}
\end{center}

When the members are acting as individuals, the collective noun is plural and requires a plural verb.\\
In many cases, it may sound more natural to make the subject plural in form by adding words like \textit{members, mates, etc.}
\begin{center}
		\textit{After the practice \textbf{the team(mates)} shower, change into their casual clothes, and head to their homes.}
\end{center}

\subsection{Concrete and Abstract Nouns}
\hspace{0.8cm} Nouns can be concrete or abstract. \\
\textbf{Concrete Nouns} are tangible and you can experience them with your five senses.\\
\textbf{Abstract nouns} refer to intangible things, like \textit{actions, feelings, ideals, concepts, and qualities}.
\begin{center}
		\textit{ \textbf{Food} is great. But \textbf{love} is even greater.}
\end{center}
\subsubsection{Concrete nouns}
A \textbf{concrete noun} is a noun that can be identified through one of the five senses: \textit{touch, sight, hearing, smell, or taste}.
\begin{center}
		\textit{Who turned off the \textbf{TV}?} (The noun \textit{TV} is a concrete noun)\\
		\textit{What is that \textbf{noise}?} (Even though \textit{nose} can't be touched, you can hear it, so it's a concrete noun)
\end{center}
Concrete nouns fall into several categories:
\begin{enumerate}[label=\alph*)]
		\item People\\
				e.g. \textit{mother, friend, teacher, stranger, etc.}
		\item Places\\
				e.g. \textit{school, McDonald's, Las Vegas, India, etc.}
		\item Things you can touch and see\\
				e.g. \textit{plane, cup, lamp, book, etc.}
		\item Things you can hear\\
				e.g. \textit{music, noise, someone's voice, song, etc.}
		\item Things you can smell and taste\\
				e.g. \textit{herbs, cookies, bread, wine, etc.}
\end{enumerate}

\subsubsection{Abstract Nouns}
Remember that \textbf{abstract nouns} refer to a intangible things, like \textit{actions, feelings, ideals, concepts, and qualities}.\\
Abstract nouns fall into several categories:
\begin{enumerate}[label=\alph*)]
		\item Emotions and feelings\\
				e.g. \textit{anger, sadness, love, grief, etc.}
		\item Human qualities and characteristics\\
				e.g. \textit{beauty, maturity, humour, patience, etc.}
		\item Ideas and concepts\\
				e.g. \textit{knowledge, freedom, luxury, comfort, etc.}
		\item Events\\
				e.g. \textit{marriage, birthday, career, adventure, etc.}
\end{enumerate}

Many abstract nouns are formed from adjectives, verbs, or nouns. Sometimes you can add a suffix to the concrete noun or alter the word root to form abstract nouns.
\begin{center}
		\inline{ \textit{child}}{(Concrete noun)} \inline{ \textit{child\textbf{hood}}}{(abstract noun)}
\end{center}
\newpage
Nouns with the following suffixes are often abstract:
\begin{table}[h]
\begin{center}
\begin{tabular}{|c|c|c|c|c|c|}
		\hline
		\multicolumn{2}{|c|}{\textbf{-tion} e.g. \textit{devotion}} & \multicolumn{2}{|c|}{\textbf{-ism} e.g. \textit{pessimism}} & \multicolumn{2}{|c|}{\textbf{-ity} e.g. \textit{hospitality}}\\ \hline
		\multicolumn{2}{|c|}{\textbf{-ment} e.g. \textit{movement}} & \multicolumn{2}{|c|}{\textbf{-ness} e.g. \textit{restlessness}} & \multicolumn{2}{|c|}{\textbf{-age} e.g. \textit{marriage}} \\ \hline
		\multicolumn{2}{|c|}{\textbf{-ance} e.g. \textit{brilliance}} & \multicolumn{2}{|c|}{\textbf{-ence} e.g. \textit{indifference}} & \multicolumn{2}{|c|}{\textbf{-ship} e.g. \textit{relationship}} \\ \hline
		\multicolumn{3}{|c|}{\textbf{-ability} e.g. \textit{availability}} & \multicolumn{3}{|c|}{\textbf{-acy} e.g. \textit{bureaucracy}} \\ \hline
\end{tabular}
\end{center}
\caption{Common suffixes for abstract nouns} \label{tab:nouns3}
\end{table}

\subsection{Possessive Nouns}
The \textbf{Possessive} form is used with \textbf{nouns} referring to people, groups of people, countries, and animals.\\
It shows a relationship of belonging between one thing and another.
\begin{center}
		\textit{ \textbf{Leslie's} aunt is a doctor.}
\end{center}
To form the possessive, add an \textbf{apostrophe + -s} to the noun.
\begin{center}
\textit{My \textbf{brother's} computer was stolen a week ago.\\
		\textbf{Children's} toys were on the ground.\\}
\end{center}
If the noun already \textbf{ends in -s}, just add an \textbf{apostrophe}.
\begin{center}
\textit{ \textbf{Student's} homework will be assessed later.}
\end{center}
For names \textbf{ending in -s}, you can either add an \textbf{apostrophe + -s}, or just an \textbf{apostrophe}. The first option is more common.
\begin{center}
\textit{They want to sell \textbf{Jame's }car.}
\end{center}
Study some of the fixed expressions where the possessive form is used.
\begin{center}
\textit{a day's work, a month's pay, in a year's time, for God's sake}
\end{center}
Note that the possessive is also used to refer to \textit{shops, restaurants, churches, universities, etc.}, using the name or job title of the owner.
\begin{center}
\textit{
		I want to go to \textbf{Luigi's} for dinner.\\
		Peter has an appointment \textbf{at the dentist's} at 10 a.m.
}
\end{center}
\section{Pronouns}
A \textbf{pronoun} is a word that replaces a noun in a sentence, making the subject a person or a thing
\subsection{Subject Pronouns}
A \textbf{subject} is the person or thing that performs the action in the clause or sentence.

A \textbf{subject pronoun} is a pronoun that takes the place of a noun as the subject of a sentence

\begin{center}
\textit{
\textbf{She} told me about her worries.}
\end{center}
Subject pronouns replace nouns that are the subject of their clause.
\begin{table}[h]
\begin{center}
		\begin{tabular}{|c|c|c|}
		\hline
		& \textbf{Singular} & \textbf{Plural} \\
		\hline
		$1^{st}$ person & I & we \\ \hline
		$2^{nd}$ person & you & you \\ \hline
		$3^{rd}$ person & he/she/it & they \\ \hline
	\end{tabular}
\end{center}
\caption{\label{tab:nouns4}Singular and plural forms for subject pronouns}
\end{table}

\newpage
We should replace the subject with a subject pronoun to avoid repetition.
\begin{center}
\textit{
\textst{Mary is a student and Mary is very hard working.}\\
Mary is a student and \textbf{she} is very hard working.}
\end{center}

We use the subject pronoun \textit{it} when we refer to objects, things, animals, or ideas.
\begin{center}
		\textit{ Love is eternal. \textbf{It} will last forever.}
\end{center}

Sometimes when we don't know the sex of a baby, we can use \textit{it}'.
\begin{center}
\textit{
Their baby is so small. \textbf{It} only weights 2 kilos.}
\end{center}
\hspace{0.4cm} We use \textit{it} when we talk about \textit{time, weather, or temperature}.
\begin{center}
\textit{
	What time is \textbf{it}? - \textbf{It}'s 7 o'clock.\\
	\textbf{It}'s quite cold today.}
\end{center}

\subsection{Object Pronouns}
An \textbf{object} is the person or thing that receives the action in the clause or sentence.\\
An \textbf{object pronoun} is a pronoun that takes the place of a noun as the object of a sentence.
\begin{center}
\textit{
She told \textbf{me} about her worries.}
\end{center}

Object pronouns are used to replace nouns that are the direct or indirect object of a clause.
\begin{table}[h]
\begin{center}
\begin{tabular}{|c|c|}
	\hline
	\textbf{Subject} & \textbf{Object} \\
	\hline
	I & me \\ \hline
	you & you \\ \hline
	he & him \\ \hline
	she & her \\ \hline
	it & it \\ \hline
	we & us \\ \hline
	they & them \\
	\hline
\end{tabular}
\end{center}
\caption{\label{tab:nouns5}Subject and Object Pronouns}
\end{table}

Object pronouns come either after a verb or a preposition.
\begin{center}
\textit{
Ethan asked \textbf{me} to talk to \textbf{them}.}
\end{center}

Note that the subject pronoun \textit{it} and the object pronoun \textit{it} look the same.

\begin{center}
\textit{
Do you know the movie 'Pretty Lady'? \textit{it} is my favourite!} (subject pronoun)\\
\textit{ I've seen \textit{it} many times.} (object pronoun)
\end{center}
Remember that object nouns are always the recipients of the action in sentence.
\begin{center}
\textit{
		\textst{He and me went to the movies}. \textbf{He and I} went to the movies.\\
\textst{Mrs. Keith called her and I}. Mrs. Keith called \textbf{her and me}.}
\end{center}

We should replace the object with an object pronoun to avoid repetition.
\begin{center}
\textit{
I can't stop thinking about Amy. \textst{I can't stop imagining my future with Amy}. I can't stop imagining my future with \textbf{her}.}
\end{center}

\subsection{Possessive Pronouns}
\textbf{Possessive pronouns} are pronouns that demonstrate ownership.
\begin{center}
\textit{This car is \textbf{mine}.}
\end{center}
\newpage
Possessive pronouns are used instead of a possessive adjective and noun. Study the following table:
\begin{table}[h]
\begin{center}
\begin{tabular}{|c|c|c|c|}
		\hline
		\textbf{Subject} & \textbf{Object} & \textbf{Possessive Adjective} & \textbf{Possessive Pronoun}\\ \hline
		I & me & my & mine \\ \hline
		you & you & your & yours \\ \hline
		he & him & his & his \\ \hline
		she & her & her & hers \\ \hline
		it & it & its & its \\ \hline
		they & them & their & theirs\\ \hline
\end{tabular}
\end{center}
\caption{Possessive Adjectives \& Possessive Pronouns} \label{tab:nouns6}
\end{table}

\section{Articles}
\textbf{Articles} are words that define a noun as specific or unspecific.\\
English has two types of articles:
\begin{itemize}
		\item Indefinite: \textit{a/an}
		\item Definite: \textit{the}
\end{itemize}
\begin{center}
\textit{I'm \textbf{a} nurse. \textbf{The} hospital I'm working in is huge.}
\end{center}

\subsection{Indefinite Article}
The \textbf{indefinite article} takes two forms: \textbf{a/an}. Use the indefinite article \textbf{a} when it precedes a word that \textbf{begin} with a \textbf{consonant}. Use the indefinite article \textbf{an} when it precedes a word that \textbf{begins} with a \textbf{vowel}.
\begin{center}
		\textit{ \textbf{a} table, \textbf{an} umbrella, \textbf{a} university, \textbf{an} honest person.}
\end{center}
The indefinite article \textbf{a/an} indicates that a noun refers to a general idea rather than a particular thing.
\begin{center}
\textit{What does \textbf{a} fox say?}
\end{center}
We use \textbf{a/an} when the listener does not know which person or thing we are talking about.
\begin{center}
		\textit{Helen's brother works in \textbf{a} factory. I don't know which factory exactly.}
\end{center}
If we refer to something for the first time, it will be new information for the listener so we use \textbf{a/an}.\\
When referencing to the same thing again use \textbf{the} because now the listener knows what we are talking about.
\begin{center}
\textit{I bought \textbf{a} new computer. It's really great! \textbf{The} computer is much better than my previous one.}
\end{center}

\subsection{Definite Article}
The \textbf{definite article} is the word \textbf{the}. It limits the meaning of a noun to one particular thing. We use \textbf{the} when it is clear which thing or person we are talking about.
\begin{center}
\textit{ \textbf{The} cake is in the fridge. I know that Kate made it.}
\end{center}

We use the definite article \textbf{the} with:
\begin{enumerate}[label=\alph*)]
\item Nationalities and other groups\\
		e.g. \textit{ \textbf{the} French, \textbf{the} Italians, \textbf{the} old, \textbf{the} poor.}
\item Time\\
		e.g. \textit{in \textbf{the} past, in \textbf{the} future (but: \textbf{at present}.)}
\item Superlatives\\
		e.g. \textit{You are \textbf{the} first one!}
\item Musical instruments\\
		e.g. \textit{I played \textbf{the} piano as a kid.}
\item Countries which are a group or plural\\
		e.g. \textit{ \textbf{the} U.S. , \textbf{the} U.K., \textbf{the} United Arab Emirates, \textbf{the} Netherlands }
\item Names of ship.\\
		e.g. \textit{We sailed on \textbf{the} Claudia}
\item Oceans\\
		e.g. \textit{ \textbf{the} Pacific, \textbf{the} Atlantic}
\item Rivers\\
		e.g. \textit{ \textbf{the} Amazon, \textbf{the} Nile}
\end{enumerate}

Note that we use \textbf{zero article} with \textbf{plurals} and \textbf{uncountable nouns} when we are generally talking about something.
\begin{center}
		\textit{ \textbf{Dogs} are not allowed in that shop.} (We are talking about dogs in general.)\\
		\textit{ \textbf{The dogs} next door were barking at night.} (W are talking about the particular dogs.)
\end{center}

\section{Demonstratives}

\textbf{Demonstratives} are words that show which person or thing is being referred to.
Demonstratives show where an object, event, or person is in relation to the speaker. They can refer to a physical or a psychological closeness or distance.
\begin{center}
		\textit{\textbf{This} is Hugh, and \textbf{that} is Kevin.}
\end{center}

\begin{table}[h]
\begin{center}
\begin{tabular}{|c|c|c|}
		\hline
		     	 & \textbf{Near the speaker} & \textbf{Far from the speaker} \\ \hline
		Adverbs  & here & there \\ \hline
		Demonstratives with singular and uncountable nouns & this & that \\ \hline
		Demonstratives with plural countable nouns & these & those\\ \hline
\end{tabular}
\end{center}
\caption{\label{tab:nouns7}Demonstratives}
\end{table}
Demonstratives can be placed before the noun or the adjective that modifies the noun.
\begin{center}
\textit{ \textbf{That old man} stole my purse!\\
\textbf{These oranges} are delicious!}
\end{center}
Demonstratives can also appear before a number by itself when the noun is understood from the context.
\begin{center}
\textit{I'll take \textbf{this one}, please.} = \textit{I'll take this watermelon, please.}
\end{center}
Demonstratives can be used by themselves when the noun they modify is understood from the context.
\begin{center}
\textit{ \textbf{Those} aren't yours. Put them back.} = \textit{Those shoes aren't yours. Put them back.}
\end{center}
When talking about events, the \textbf{near demonstratives} are often used to refer to the \textbf{present} while the \textbf{far demonstratives} often refer to the \textbf{past}.
\begin{center}
\textit{ \textbf{This situation} is quite unstable. \\
\textbf{That event} made me realise how important my family is to me.}
\end{center}

\section{Distributives}
\textbf{Distributives determiners} or simply \textbf{distributives} refer to a group of people or things, and to individual members of the group.\\
\indent They show different ways of looking at the individuals within a group, and they express how something is distributed, shared, or divided.
\begin{center}
\textit{ \textbf{All people} want to love and to be loved.\\
\textbf{Each} person is unique. \textbf{Every} person is unique.\\
\textbf{Both of us} like Mexican food.}
\end{center}

\subsection{All}
The distributive determiner \textbf{all} is used to talk about a whole group, with a special emphasis on the fact that nothing has been left out.\\
\indent \textbf{All} can be used with uncountable nouns and plural countable nouns by itself. In this usage, it refers to the group as a concept rather than as individuals.
\begin{center}
\textit{ \textbf{All parents} want the best for their children.}
\end{center}
\textbf{All} can be used with uncountable nouns and plural countable nouns preceded by \textbf{the} or a \textbf{possessive adjective}. In these uses, the word \textbf{of} can be added just after \textbf{all} with no change in meaning.
\begin{center}
\textit{ Have you eaten \textbf{all the cookies} in the jar?} = \textit{Have you eaten \textbf{all of the cookies} in the jar?.}
\end{center}
\textbf{All} can be used with \textbf{plural pronouns} preceded by \textbf{of}.
\begin{center}
\textit{ \textbf{All of us} are going to be there tonight.}
\end{center}
\textbf{All} can be used in questions and exclamations with \textbf{uncountable nouns} preceded by \textbf{this/that} or with \textbf{countable nouns} preceded by \textbf{these/those}.  In these uses, the word \textbf{of} can be added just after \textbf{all} with no change in meaning.
\begin{center}
\textit{ Look at \textbf{all this snow} out there!\\
What are \textbf{all these people} doing in our house?}
\end{center}

\subsection{Half}
The distributive determiner \textbf{half} is used to talk about a whole group divided in \textbf{two}. \textbf{Half} can be used as a distributive in several different patterns.\\
\textbf{Half} can refer to measurements if it is followed by an indefinite article \textbf{a/an} and a noun.
\begin{center}
\textit{I'll be back in \textbf{half an hour}.}
\end{center}
\textbf{Half} can be used with plural pronouns preceded by \textbf{of}.
\begin{center}
\textit{ \textbf{Only half of us} are going to be there tonight.}
\end{center}
\textbf{Half} can be used with nouns preceded by \textbf{the, a/b, a demonstrative, or a possessive adjective}. In this case, the meaning refers to a concrete, physical division.
The word \textbf{of} can be added just after \textbf{half} with no change in meaning.
\begin{center}
\textit{ \textbf{Half the people} have already left the party.\\
Putting \textbf{half a kilo of sugar} into the topping will ruin the cake.\\
I want \textbf{half of that cake}!\\
Sorry, but I used \textbf{half of your eggs} making breakfast today.}
\end{center}

\subsection{Each and Every}
The distributives \textbf{each} and \textbf{every} are both related to describing the members of a group. These distributives can only be used with \textbf{countable nouns} by being placed before the nouns. \\
\indent In many cases, they are interchangeable but there is a \textbf{subtle difference} between them.
\subsubsection{Each}
\textbf{Each} is used to describe and highlight an individual member of a group, or multiple individuals. By using \textbf{each} you recognise the item is a part of a group, but that it also needs to be pointed out as a singular item too.
\begin{center}
\textit{ \textbf{Each book} on the shelf had a unique cover.}
\end{center}
\textbf{Each} can be used with plural nouns and pronouns but \textbf{must} be followed by \textbf{of}.
\begin{center}
\textit{ \textbf{Each of the pupils} received a Christmas card.}
\end{center}
\textbf{Each} can be used after the subject or at the end of a sentence.
\begin{center}
\textit{ \textbf{My siblings each} have their own room.\\
My mother gave my sister and I \$20 \textbf{each}.} = (gave \$20 to each of us.)
\end{center}
\subsubsection{Every}
\textbf{Every} by contrast is a way of referring to the group as a collection of individual members. \textbf{Every} cannot be used with plural nouns.
\begin{center}
\textit{ \textst{ \textbf{Every boys} in my class wanted that computer game.} \textbf{Every boy} in my class wanted that computer game.}
\end{center}
\textbf{Every} can express different points in a series, especially with time expressions.
\begin{center}
\textit{ \textbf{Every} morning Phillip goes for a run.\\
And \textbf{every time} Ann would forgive him.}
\end{center}

\subsection{Both}
\textbf{Both} refers to the whole pair and is equivalent to \textit{'one and the other'}. \textbf{Both} can be used with plural nouns on its own, or it can be followed by \textbf{of}, with \textbf{of} without an article. When followed by a plural pronoun, \textbf{both} must be separated from the pronoun by \textbf{of}.
\begin{center}
		\textit{ \textbf{Both (of) my parents} approve of me going to college.\\
		I told \textbf{both of them} to give me a call.}
\end{center}
\textbf{Both} cannot be used with singular nouns, because it refers to two things.
\begin{center}
\textit{ \textst{Both my sister likes traveling.} \textbf{Both my sisters like traveling.}}
\end{center}

\subsection{Either}
\textbf{Either} is positive and when used alone refers to one of the two members of the pair. It is equivalent to \textit{'one or the other'}. Because it refers to just one member of a pair, \textbf{either} must be used before a singular noun. It can also be used with a plural noun or pronoun if followed by \textbf{of}.
\begin{center}
		\textit{ \textbf{Either day} is fine.\\
		We could stay at \textbf{either of the hotels}.}
\end{center}
\textbf{Either} can also be used with \textbf{or} in a construction that talks about each member of the par in turn. The meaning remains the same, but in this case \textbf{either} is not functioning as a distributive. It is functioning as a \textbf{conjunction}.
\begin{center}
\textit{ You can have \textbf{either} ice cream \textbf{or} cake.}
\end{center}

\subsection{Neither}
\textbf{Neither} is negative and when used alone refer to the whole pair. It is equivalent to \textit{'not one or the other'}. Because it refers to just one member of a pair, \textbf{neither} must be used before a singular noun. It cal also be used with a plural noun or pronoun if followed by \textbf{of}.
\begin{center}
\textit{ \textbf{Neither date} is convenient for me.\\
\textbf{Neither of these dresses} suits her.}
\end{center}
\textbf{Neither} can also be used with \textbf{nor} in a construction that talks about each member of the pair it turn. The meaning remains the same, but in this case \textbf{neither} is not functioning as a distributive. It is functioning as a \textbf{conjunction}.
\begin{center}
\textit{ It is \textbf{neither} snowing \textbf{nor} raining.}
\end{center}

\newpage
\section{Quantifiers}
We use \textbf{quantifiers} when we want to give someone information about the number of something, the are adjectives and adjectival phrases that give approximate or specific answers to the questions \textit{'How much?'} and \textit{'How many?'}

\begin{center}
	\textit{ \textbf{Most} children start school at the age of five. \\
		I ate some \textbf{rice}. \\
		There are \textbf{a lot of} dogs. }
\end{center}

We can use \textbf{quantifiers} with both \textbf{count} and \textbf{uncountable} nouns:
 \begin{center}
 	\textit{How \textbf{much} coffee do we have left. \\
	 		How many \textbf{cookies}  do you have?}
 \end{center}

\textbf{How much} is used to ask about uncountable nouns and when we want to know the price of something.
\begin{center}
	\textit{ \textbf{How much} this computer cost? }
\end{center}

\subsection{A Few,  Little}
A (very) few, (very) little are generally used in affirmative statements, not negatives or questions.
\begin{table}[h]
	\begin{center}
	\begin{tabular}{|c|c|}
		\hline
		\textbf{With countable nouns} & \textbf{With uncountable nouns} \\ \hline
		\textbf{(very) few} = hardly any or not enough & \textbf{(very) little} = hardly any or not enough \\ \hline
		\textit{I have \textbf{(very) few} toys.} & \textit{We have \textbf{(very) little} coffee left. } \\ \hline
		\textbf{a few} = some or enough & \textbf{a little} = some or enough \\ \hline
		\textit{I have \textbf{a few} examples to show } & \textit{I have \textbf{a little} coffee left but I can make me a cup of coffee } \\ \hline
	\end{tabular}
\end{center}
\caption{\label{tab:quantifiers1}A few vs A little}
\end{table}

\subsection{Much and Many}
Normally, we use \textbf{much} and \textbf{many} only in questions and negative clauses. But can be used in affirmative sentences in combination with \textit{too} and \textit{so}. In this case, they denote the excessive amount of something.

\begin{center}
	\textit{How \textbf{much} money do you have left? \\
	There are \textbf{too many} people. \\
	You put a \textbf{lot of sugar} on my coffee! }
\end{center}

We use \textbf{much} to talk about the quantity of uncountable nouns or the price of something, while we use \textbf{many} when we talk about the quantity of countable nouns.
\begin{center}
	\textit{I have \textbf{many} friends. \\
	She has too \textbf{much}  money }
\end{center}

\subsection{A Lot, Most}
Note that in spoken English and informal writing when we want to indicate a large quantity of something we tend to use \textbf{ a lot, a lot of, lots of}.\\

\textbf{A lot} means very often or very much. It is used as an adverb. It often comes at the end of a sentence and \textbf{never} before a noun.

\begin{center}
	\textit{My brother plays video games \textbf{a lot}. \\
	She's \textbf{a lot} happier after quitting her job. }
\end{center}

We use the quantifier \textbf{most} to talk about quantities, amounts and degree. We can use it with a noun (as a determiner) or without a noun (as a pronoun). \\
\newpage
We use \textbf{most} with nouns in the meaning \textbf{the majority of}. If there is no article, demonstrative or possessive pronoun, we use \textbf{most} right before the noun.
 \begin{center}
 	\textit{ \textbf{Most tap water} is drinkable. }
 \end{center}

When we are talking about the majority of a specific set of something, we use \textbf{most of the + noun}.

\begin{center}
	\textit{ \textbf{Most cakes} are sweet. } (cakes in general) \\
	\textit{ The party was amazing. Kate made \textbf{most of the cakes} herself. } (a specific set of cakes at the party)
\end{center}

We can leave out the noun with \textbf{most} when the noun is obvious from the context.

\begin{center}
	\textit{Students can eat in the cafeteria but \textbf{most} bring food from home.} (=most students)
\end{center}

\subsection{Some, Any and Enough}

We use \textbf{some, any} when we are talking about limited but rather indefinite number of quantities.

In general, we use \textbf{some} for affirmative sentences, and \textbf{any} for negatives and questions. Both can be used with countable and uncountable nouns.
\begin{center}
	\textit{Jane bought \textbf{some} flowers. \\
	Did Jane buy \textbf{any} flowers? - No, she didn't buy \textbf{any}. }
\end{center}

\textbf{Some} can be used for questions, typically offers and requests, if we think the answer will be positive.

\begin{center}
	\textit{Would you like \textbf{some} tea.}
\end{center}

\textbf{Any} can be used in the meaning \textit{'it doesn't matter which'}.

\begin{center}
	\textit{You cant take \textbf{any} bus. They all go to the centre.} (=it doesn't matter which bus you take=
\end{center}

We use \textbf{enough} to indicate sufficiency, while in negative sentences it means less than sufficient or less than necessary.
\begin{center}
	\textit{I'll take your t-shirt. It's \textbf{big enough} to fit me. \\
	Sorry, but I can't go with you. I don't have \textbf{enough money} for that. }
\end{center}

\section{Verb Conjugation}

\textbf{Verb conjugation} refer to how a verb changes to indicate a different person, number, tense, or mood. In other words, \textbf{conjugation} is the changing of a verb's form to express a different person, number, tense, aspect, or gender. In order to communicate in more than one tone, verbs must be conjugates. To conjugate something is to change a verb's form to express a different meaning.

\begin{center}
	\textit{I'm a student.} ($1^{st}$ person, singular, present simple, indicative mood)
\end{center}

\subsection{First, Second and Third Person}
Verbs should be conjugated with regard to person. Depending on the subject, a verb can stand in the first, second, or third person.

\begin{table}[h]
	\begin{center}
	\begin{tabular}{|c|c|c|}
		\hline
		  & \textbf{Singular} & \textbf{Plural} \\ \hline
		\textbf{$1^{st}$ person} & I & we \\ \hline
		\textbf{$2^{nd}$ person} & you & you \\ \hline
		\textbf{$3^{rd}$ person} & he, she, it & they \\ \hline
	\end{tabular}
\end{center}
	\caption{\label{tab:vrbcjg1}First, Second and Third Person (Singular and Plural).}
\end{table}

\newpage
As you can see, the pronouns \textbf{I, were} refer to the first person; \textbf{you}, to the second person; \textbf{he, she, it, they}, to the third person.

\begin{center}
	\textit{We work on Saturdays.} (first person) \\
	\textit{You need to take a break.} (second person) \\
	\textit{It is snowing outside.} (third person)
\end{center}

Usually we assume the person of the verb in the sentence automatically as we almost always state a subject explicitly.

\begin{center}
	\textit{Sarah has signed up for a yoga class.} ( \textbf{Sarah} can be substituted with the pronoun \textbf{she}; the verb is in the third person)
 \end{center}

Note that the verb \textbf{to be} is irregular and has three forms in present tenses and two forms in past tenses. These forms depend on the person expressed by the subject.

\begin{table}[h]
	\begin{center}
	\begin{tabular}{|c|c|c|c|c|c}
		\hline
		 & \multicolumn{2}{|c|}{ \textbf{Present}} & \multicolumn{2}{|c|}{ \textbf{Past}} \\ \hline
		\textbf{$1^{st}$ person} & I am & we are & I was & we were \\ \hline
		\textbf{$2^{nd}$ person} & you are & you are & you were & you were \\ \hline
		\textbf{$3^{rd}$ person} & he/she/it is & they are & he/she/it was & they were \\ \hline
	\end{tabular}
\end{center}
\caption{\label{tab:vrbcjg2}Verb To Be forms.}
\end{table}

\section{Simple Tense}
\subsection{Past Simple}
The \textbf{past simple} is used to write and talk about completed actions that happened in a time before the present. It is the basic form of the past tense in English. We use the \textbf{past simple} when we talk about an action which happened at a definite time in the past.

This tense emphasizes that the action is finished.

We can also use this tense to talk about how someone felt about something.

\begin{center}
	\textit{I \textbf{solved} the puzzle. \\
	I \textbf{was} happy for your success.}
\end{center}

\subsubsection{How to form the past simple tense}
\begin{itemize}
	\item infinitive + (e)d \\
		e.g. \textit{He \textbf{worked} part-time as a waiter. \\
		We \textbf{liked} our stay at the hotel.} \\
	Note that \underline{all persons}  have the same form.
	\item cons + -y $\Rightarrow$ cons + -ied \\
		e.g. \textit{cry-cr\textbf{ied}, try,tr\textbf{ied}}
	\item vowel + const $\Rightarrow$ vowel + double const + ed \\
		e.g. \textit{stop-sto\textbf{pp}ed, regret-regre\textbf{tt}ed}
\end{itemize}

Remember that irregular verbs don't follow the rules above, use \underline{the past tense form of the irregular verbs} to make sentences in the past simple.
\begin{center}
	\textit{be-\textbf{was/were}, eat-\textbf{ate}, drink-\textbf{drank}}
\end{center}

The \textbf{past tense} of the verb \textbf{to be} depends on the person of the subject. (Table - \ref{tab:vrbcjg2})

\begin{table}[h]
	\begin{center}
	\begin{tabular}{|c|c|}
		\hline
		I was & We were \\ \hline
		You were & You were \\ \hline
		he/she/it was & They were \\ \hline
	\end{tabular}
\end{center}
\caption{\label{tab:pastsimple1}Past forms of verb To Be}
\end{table}


\subsubsection{Positive, negative, and questions forms}

\inline{ \textbf{did}}{Positive}/\inline{\textbf{did not}}{Negative} + Verb

\begin{itemize}
	\item (+) \textit{His sister \textbf{lived} in Sutton, London.}
	\item (-) \textit{His sister \textbf{did not live} in Sutton. She \textbf{lived} in Harrow. }
	\item (?) \textit{ \textbf{Did} his sister \textbf{live} in Sutton?  }
	\item (?) \textit{Where \textbf{did} his sister \textbf{live} in London?}
\end{itemize}

\subsubsection{Using time markers}

Yesterday, last night, (not) a long time ago, two years ago, etc.

\begin{center}
	\textit{Shakespeare died \textbf{in 1616}.\\
	Ryan did not go to work \textbf{yesterday}. He got sick. \\
	\textbf{When} did you move to Spain? - I moved there \textbf{not a long time ago}. }
\end{center}

Note that we use \textbf{did/did not} with the verb \textbf{to have}.

\begin{center}
	\textit{I didn't have enough money to buy a new computer.}
\end{center}

But we do \textbf{not use did} with the verb \textbf{to be (was/were)}.

\begin{center}
	\textit{- Why \textbf{were you} so angry? \\
	- \textbf{I wasn't} angry. \textbf{This was} my usual self.}
\end{center}

\subsection{Present Simple}
The \textbf{present simple} also called \textit{present indefinite} is a verb tense which is used to show repetition, habit or generalization. We use the \textbf{present simple} when we talk about things in general.

We use this tense to say that something:
\begin{itemize}
	\item Happens all the time.
	\item Happens repeatedly.
	\item Is true in general.
\end{itemize}

\begin{center}
	\textit{Jane \textbf{works} as a barista. Her shift \textbf{begins} at 7 a.m.}
\end{center}

\subsubsection{How to form the present simple tense}
The present tense is the \textbf{base form} of the verb
\begin{center}
	\textit{I \textbf{work} in London.}
\end{center}
But with the third person singular (she/he/it), we add an -s
\begin{center}
	\textit{She \textbf{works} in London}
\end{center}
and when the verb ends in -o, -s, -ch, -sh, -x, we add -es instead
\begin{center}
	\textit{My sister \textbf{watches} TV in the evening and my brother \textbf{does} his homework.}
\end{center}

Remember that such verbs as \textbf{to be} and \textbf{to have} are \underline{irregular}.

\begin{table}[h]
	\begin{center}
	\begin{tabular}{|c|c|c|c|}
		\hline
		\multicolumn{2}{|c|}{ \textbf{To Be}} & \multicolumn{2}{|c|}{ \textbf{To Have}} \\ \hline
		I am & we are & I have & we have \\ \hline
		you are & you are & you have & you have \\ \hline
		he/she/it is & they are & he/she/it has & they have \\ \hline
	\end{tabular}
\end{center}
\caption{\label{tab:presentsimple1}Present simple: to be - to have}
\end{table}

Note the difference between BrE and AmE:
\begin{center}
	(BrE) - \textit{I have got a car.} \quad (AmE) - \textit{I have a car.}
\end{center}

\subsubsection{Positive, negative, and questions forms}

\inline{\textbf{do not/does not + verb}}{Negative   }

\begin{itemize}
	\item (+) \textit{He \textbf{gets up} at 6 o'clock every morning.}
	\item (-) \textit{He \textbf{does not get up} at 6 o'clock every morning. \\
		He \textbf{gets up} at 7.}
	\item (?) \textit{ \textbf{Does he get up} at 6 o'clock every morning?}
	\item (?) \textit{ \textbf{When does he get up?} }
\end{itemize}

\subsubsection{Using time markers}
You can add time markers such as always, often, usually, sometimes, rarely, never, every day, etc.

\begin{center}
	\textit{I \textbf{usually} cook at home but my friends \textbf{always} eat at the local cafe. \\
		Kim is \textbf{always} late for classes.}
\end{center}

Notice where they are places in the sentences.


\subsection{Subject-Verb Agreement}
The \textbf{subject-verb agreement} is the correspondence of a verb with its subject in person (firth, second, or third) and number (singular or plural).

\begin{center}
	\textit{ \textbf{Liz is} an accountant and \textbf{she has} a typical 8-5 job. }
\end{center}

Subjects and verbs must agree with one another in person (first, second, or third).

Note that \textbf{subject-verb agreement} rules of the verb \textbf{to be} in present tenses.

\begin{table}[h]
	\begin{center}
	\begin{tabular}{|c|c|c|}
		\hline
		& \textbf{Singular} & \textbf{Plural} \\ \hline
		\textbf{$1^{st}$ person} & I am & we are \\ \hline
		\textbf{$2^{nd}$ person} & you are & you are \\ \hline
		\textbf{$3^{rd}$ person} & he/she/it is & they are \\ \hline
	\end{tabular}
\end{center}
\caption{\label{tab:verbagre1}Subject-Verb Agreement: To Be}
\end{table}

\begin{center}
	\textit{I am a student} ($1^{st}$ person), \textit{my brother is a pupil} ($3^{rd}$ person), \textit{and you are a teacher} ($2^{nd}$ person).
\end{center}

Subjects and verbs must agree with one another in number (singular or plural). Thus, if a subject is singular, its verbs must also be singular; if a subject is plural, its verb must also be plural.

\begin{center}
	\textit{ \textbf{She cooks} dinner, and \textbf{her brothers make} breakfast.}
\end{center}

When the subject of the sentence is composed of two or more nouns or pronouns connected by the conjunction \textbf{and}, use a plural verb.

\begin{center}
	\textit{ \textbf{Brothers and sisters don't} often \textbf{get along}. }
\end{center}

The words \textbf{each, each one, either, neither, everyone, everybody, anyone, anybody, nobody, somebody, someone,} and \textbf{no one} are singular and require a singular verb.

\begin{center}
	\textit{ \textbf{Each of these suggestions is} interesting. \\
	\textbf{Someone was standing} at the door.}
\end{center}

When two or more singular nouns or pronouns are connected by \textbf{or} or \textbf{nor}, use a singular verb.

\begin{center}
	\textit{ \textbf{Either your mother or dad needs} to contact me.}
\end{center}

\subsubsection{The Rule of Proximity}

When a compound subject contains both a singular and a plural noun or pronouns joined by \textbf{or} or \textbf{nor}, the verb should agree with the part of the subject that is closer. (also called \textbf{the rule of proximity}). \\

\begin{center}
	\textit{The teacher or \textbf{the students write} homework on the board. \
	The students or \textbf{the teacher writes} homework on the board.}
\end{center}

\subsubsection{The Inverted Subject}

In sentences beginning with \textbf{there is} or \textbf{there are}, the subject follows the verb (also called \textbf{the inverted subject}). As \textbf{there} is not the subject, the verbs agrees with what follows.

\begin{center}
	\textit{There \textbf{is a book} on the table. \
	There \textbf{are books} on the table.}
\end{center}

\subsubsection{More about subject-verb agreement}

Note the \textbf{subject-verb agreement} with words that indicate portions (e.g \textit{a lot, a majority, some, all}): If the noun after \textbf{of} is singular, use a singular verb; if it is plural, use a plural verb.

\begin{center}
	\textit{ \textbf{There is a lot of fuss} around his arrival. \
	\textbf{There are a lot of people} in the room.}
\end{center}

Use a singular verb with distances, periods of time, sums of money, etc. when considered as a unit.

\begin{center}
	\textit{ \textbf{Ten dollars is} a high price to pay for socks.} \\
	But: \textit{ \textbf{Ten dollars} (i.e. dollar bills) \textbf{were} scattered on the floor. }
\end{center}

Collective nouns are words that imply more than one person but are considered singular and take a singular verb (e.g. \textit{family, group, team, committee, class, etc.}).
\begin{center}
	\textit{ \textbf{My family} is very big.}
\end{center}

\subsection{Future Simple}
The \textbf{future simple tenses} is often called the \textbf{"will tense"} because we make the \textbf{future simple} with the modal auxiliary \textbf{will}.

We can refer to the future by using \textbf{will, be going to} or by using \textbf{present tenses.}

We use the \textbf{will} future when we want to talk generally about future beliefs, opinions, hopes and predictions.

\begin{center}
	\textit{I promised myself that once I start college \textbf{I will do} all my assignments on time. }
\end{center}

\subsubsection{Positive, negative, and questions forms}

\inline{\textbf{will ('ll)/}}{Positive} \inline{ \textbf{will not (won't)}}{Negative} + verb

\begin{itemize}
	\item (+) \textit{Sam \textbf{will} probably \textbf{move} to Canada next year.}
	\item (-) \textit{Sam \textbf{won't move} to Canada next year. He\textbf{'ll move} to the US.}
	\item (?) \textit{\textbf{Will} Sam \textbf{move} to Canada next year?}
	\item (?) \textit{Where \textbf{will} Sam \textbf{move} to?}
\end{itemize}

\subsubsection{Using time and probability markers}

Time markers - \textbf{tomorrow, next month, in a day, etc.} \\
Probability markers - \textbf{perhaps, probably, definitely, etc.} \\

\begin{center}
	\textit{ \textbf{Perhaps} it'll snow \textbf{tomorrow}. \\
	I'll \textbf{definitely} finish my essay \textbf{next month}.}
\end{center}

Pay attention to the word order.

\begin{center}
	(+) \textit{We'll \textbf{probably} do it tomorrow.} \\
	(-) \textit{We \textbf{probably} won't do it tomorrow.}
\end{center}

Some speakers use \textbf{shall} to refer to the future in \underline{formal situations} (with / and we).

Nowadays \textbf{shall} is used for \underline{suggestions} only.

\begin{center}
	\textit{ \textbf{Shall I} go or \textbf{shall we} leave together? }
\end{center}

\section{The Gerund}
\textbf{The gerund}  looks exactly the same as a \textbf{present participle}, but it is useful to understand the difference between the two. \textbf{The gerund} always has the same \textbf{function as a noun} (although it looks like a verb).

\begin{center}
	\textit{ \textbf{Hunting} tigers is dangerous.}
\end{center}

Some rule to form the gerund
\begin{itemize}
	\item -e + ing \\
		e.g. \textit{make-making, write-writing}
	\item vowel + cons $\Rightarrow$ double cons + -ing \\
		e.g. \textit{knit-knitting, swim-swimming}
	\item -ie $\Rightarrow$ -y + -ing \\
		e.g. \textit{lie-lying, die-dying}
\end{itemize}

\textbf{The gerund} can be made negative by adding not.

\begin{center}
	\textit{The best thing for your health is \textbf{not smoking}. }
\end{center}

The \textbf{gerund} can function as:

\begin{enumerate}[label=(\alph*)]
	\item The subject of the sentence. \\
		e.g. \textit{ \textbf{Smoking} causes lung cancer.}
	\item The complement of the verb to be. \\
		e.g. \textit{The hardest thing about learning Russian is \textbf{memorizing} the verbs of movement. }
\end{enumerate}

The \textbf{gerund} can be used:
\begin{enumerate}[label=(\alph*)]
	\item After prepositions or as part of certain expressions. (there's no point in, in spite of, etc.)\\
		e.g. \textit{Can your brother count to ten \textbf{without looking} at his fingers? \\
					 \textbf{There's no point in going} back to his place now. }
	\item After phrasal verbs. They are composed of a verb + preposition/adverb. \\
		e.g. \textit{I \textbf{ended up buying} a new computer. \
		Rachel \textbf{gave up drinking} sugar drinks.}
\end{enumerate}

\section{Present Participle}
Most commonly we use the \textbf{present participle -ing} as an element in all continuous verb forms (the present continuous, the past continuous, etc.).

The auxiliary verb indicates the tense, while the present participle remains unchanging.

\begin{center}
	\textit{ \textbf{I was playing computer games all night} } (past continuous)
\end{center}

\subsection{How to form the present participle}

\begin{itemize}
	\item Verb ending in -e + -ing \\
		e.g. \textit{like-liking, write-writing}
	\item Verb ending with vowel + cons $\Rightarrow$ double cons + -ing \\
		e.g. \textit{sit-sitting, swim, swimming}
	\item Verb ending in -ie $\Rightarrow$ -y + -ing \\
		e.g. \textit{lie-lying, die-dying}
\end{itemize}

\subsection{Uses}
The present participle is used not only form verb tenses. It can be used:
\begin{enumerate}[label=(\alph*)]
	\item After verbs of movement and position. \\
		e.g. \textit{ She went \textbf{shopping. \\
			} They came \textbf{running} towards me. }
	\item After verbs of perception in the pattern verb + object + present participle to indicate the action being perceived. \\
		e.g. \textit{We saw him \textbf{mowing} the lawn. \
		Liz heard someone \textbf{singing}.}
	\item After verbs of movement, action, or position to indicate parallel activity. \\
		e.g. \textit{He sat \textbf{looking at} the pedestrians. \\
		July walks \textbf{reading her newspaper.} }
	\item As an adjective. \\
		e.g. \textit{Have you heard of that \textbf{amazing} movie? \
		The family was trapped inside the \textbf{burning} barn.}
	\item To explain the cause or reason. The present participle is used instead of a phrase starting with \textbf{as, since, because.} \\
		e.g. \textit{ \textbf{Feeling} hungry, I made myself a sandwich.} (= I made myself a sandwich \textbf{because} I was hungry). \\
		\textit{ \textbf{Knowing} that his roommate was coming, James cleaned the living room.} (= James cleaned the living room \textbf{as} he knew that his roommate was coming.)
\end{enumerate}


\section{Continuous Tense}
The \textbf{continuous tense} shows an action that is, was, or will be in progress at a certain time. The \textbf{continuous tense} is formed with the verb \textbf{to be} + -ing form of the verb (present participle).

\subsection{Past Continuous}
We use the \textbf{past continuous} when we describe a situation, or several situations in progress, happening at the same time in the past.

This is often contrasted with a sudden event in the past simple.

\begin{center}
	\textit{ \textbf{I was working} on my computer and by brother \textbf{was reading} a book when we heard a loud bang on the door.}
\end{center}

\subsubsection{How to form the past continuous}
\inline{\textbf{Was/were}}{Positive} + Verb -ing \quad \inline{\textbf{wasn't/weren't}}{Negative} + Verb - ing

\subsubsection{Positive, negative and question forms}
\begin{itemize}
	\item (+) \textit{Jim \textbf{was playing} video games all night.}
	\item (-) \textit{Jim \textbf{was not playing} video games all night. / He \textbf{wasn't playing} video games all night. }
	\item (?) \textit{ \textbf{Was} Jim \textbf{playing} video games all night?}
	\item (?) \textit{ \textbf{Why was} he \textbf{playing} video games all night?}
\end{itemize}

\subsubsection{Using time markers}
at 7 o'clock, for two hours, in January, last week, all night, etc.

\begin{center}
	\textit{Kate was trying to find a nice apartment in her area \textbf{for 5 months}. }
\end{center}

when, while = during the time that

\begin{center}
	\textit{ \textbf{While} they were waiting for the train, it started to rain. \\
	James broke his finger \textbf{when} he was playing basketball.}
\end{center}

\subsubsection{Exceptions}
Non-continuous verbs (e.g. \textit{to love, hate, know, want, etc.} are \textbf{no used} in any \underline{continuous tenses!} Use the past simple instead.

\begin{center}
	\textit{ \textst{I was having fun at the party, but Kim was wanting to go home.} \\
	I was having fun at the party, but \textbf{Kim wanted to go home}. }
\end{center}

\subsection{Present Continuous}
We use the \textbf{present continuous} when we talk about something happening at the time of speaking, or actions happening 'around now', even though not at the moment of speaking.

This tense also has some future meanings.

\begin{center}
	\textit{Hey, \textbf{what are you doing?} - \textbf{I am working on my thesis. I am graduating this semester.} }
\end{center}

\subsubsection{How to form the present continuous}
\inline{to be}{Positive} +  Verb -ing \quad \inline{to be + not}{Negative} + Verb -ing

\subsubsection{Positive, negative and question forms}
\begin{itemize}
	\item (+) He \textbf{is sleeping} on the couch in the living room.
	\item (-) He \textbf{is not sleeping} on the couch in the living room.
	\item (-) He \textbf{isn't sleeping} there.
	\item (?) Where is he? \textbf{Is he sleeping?}
\end{itemize}

\subsubsection{Using time markers}
Now, right now, at the moment, today, this week, etc.

\begin{center}
	\textit{I'm quite busy \textbf{this year} as I'm trying to start my small business.}
\end{center}

\subsubsection{Other uses}
Use the present continuous to talk about changing situations

\begin{center}
	\textit{ \textst{The population of the world increases very fast.} \\
	The population of the world \textbf{is increasing} very fast.}
\end{center}

\subsection{Future Continuous}
We use the future continuous to say that we will be in the middle of doing something at a certain time in the future.

We often use this tense when we compare what we are doing now with what we will be doing in the future.

\begin{center}
	\textit{The movie starts at 8 and ends at 10. At 9 \textbf{I will be watching} the movie}
\end{center}

\subsubsection{How to form the future continuous}
\inline{will}{Positive} + be + Verb -ing \quad \inline{won't}{Negative} + be + Verb -ing

\subsubsection{Positive, negative and question forms}
\begin{itemize}
	\item (+) Sarah \textbf{will be flying} home at 5 o'clock tomorrow.
	\item (-) Sarah \textbf{will not be flying} home at 5 o'clock tomorrow. / She \textbf{won't be flying} home at 5 o'clock tomorrow.
	\item (?) \textbf{Will} Sarah \textbf{be flying} home at 5 o'clock tomorrow?
	\item (?) \textbf{Where will} she \textbf{be flying} at 5 o'clock tomorrow?
\end{itemize}

\subsubsection{Using time markers}
at 5 o'clock, at that time tomorrow, this evening, in 5 years' time, etc.

\begin{center}
	\textit{Where will you be living \textbf{in 3 years' time}? }
\end{center}

\subsubsection{Other uses}
Use the future continuous to say that something will definitely happen in the future.

\begin{center}
	\textit{I'\textbf{be going} to the shop later. Can I get you anything?}
\end{center}

\subsection{Comparing continuous tenses}
Compare \textbf{will be doing} with other continuous forms.

\begin{center}
	\textit{Jane has an ordinary 9/8 job. \\
	At 11 o'clock yesterday she was working.} (past continuous) \\
	\textit{At 11 o'clock today she is working.} (present continuous) \\
	\textit{At 11 o'clock tomorrow she will be working.} (future continuous)
\end{center}

\section{Past Participle}
A \textbf{past participle} refers to the form of a verb which is used in forming perfect and passive tenses (and sometimes used as an adjective).

\begin{center}
	\textit{Olivia has \textbf{lived} in Greece for 4 years.}
\end{center}

\subsection{How to form the past participle}
We usually add -(e)d to the base form of the regular verb to form the past participle

\begin{center}
	\textit{Jun has just \textbf{painted} this picture.} (present perfect, active voice). \\
	\textit{This picture was \textbf{painted} by Jun a month ago.} (past simple, passive voice)
\end{center}

There is no pattern as to forming the past participle of the irregular verbs. You should always consult a dictionary.

\subsection{Uses}
\begin{enumerate}[label=(\alph*)]
	\item In the perfect tenses (Present Perfect, Past Perfect, Future Perfect). \\
		e.g. \textit{I've \textbf{eaten} to much! I can't move.} (present perfect) \\
			\textit{James had already \textbf{left} when Pam arrived.} (past perfect) \\
			\textit{We will have \textbf{landed} by that hour.} (future perfect)
	\item In the passive voice. \\
		e.g. \textit{He was \textbf{driven} by genuine interest and curiosity.\\
		This dress was \textbf{made} by a famous Italian designer.}
	\item As an adjective. In this case, place it before a noun. \\
		e.g. \textit{Mike has \textbf{broken} his arm.} $\Rightarrow$ \textit{He has a \textbf{broken} arm now.\\
		Someone has stolen Ann's purse.} $\Rightarrow$ \textit{Her purse was \textbf{stolen.} }
\end{enumerate}

\section{Perfect Tense}
The \textbf{perfect tense} or aspect is a verb form that indicates that an action or circumstance occurred earlier than the time under consideration, often focusing attention on the resulting state rather than on the ocurrence itself.

\begin{center}
	\textit{ \textbf{I have made} dinner}\\
	Although this gives information about a prior action (my making of the dinner), the focus is likely to be on the present consequences of that action (the fact that the dinner is now ready).
\end{center}

\subsection{Present Perfect}
We use the present perfect to describe past events which are connected to the present.

Although this tense can be used to describe different situations.

\begin{center}
	\textit{Sam \textbf{has lost} his keys.} (= He is looking for his keys and he still hasn't found them.)
\end{center}

\subsubsection{How to form the present participle}
\begin{center}
	\inline{have/has}{Positive} + Verb ending in -ed (\textbf{past participle}) or Simple Verb \\
	\inline{haven't/hasn't}{Negative} + Verb ending -ed (\textbf{past participle})  or Simple Verb
\end{center}

\subsubsection{Positive, negative and question forms}
\begin{itemize}
	\item (+) I \textbf{have} already \textbf{seen} that movie. / I\textbf{'ve} already \textbf{seen} that movie.
	\item (-) I \textbf{have not seen} that movie yet. / I \textbf{haven't seen} it yet.
	\item (?) \textbf{Have} I \textbf{seen} that movie?
\end{itemize}

\subsubsection{Uses}
\begin{enumerate}[label=(\alph*)]
	\item Experiences in our life up to now. \\
		e.g. \textit{I'\textbf{ve been} to Spain and Portugal. I really want to go to the UK. I \textbf{haven't been} there yet. }
	\item An event in the past that has a result in the present. \\
		e.g. \textit{Lilly \textbf{has broken} her foot. Her foot is still in a cast.}
	\item A situation that started in the past and continues until the preset. \\
		I'\textbf{ve lived} here \textbf{for twenty years}. And I am still living here now.
	\item An event in the past that has a result in the preset. \\
		e.g. \textit{Peter \textbf{has ready} 50 pages of his book so far. There are 150 pages left.}
\end{enumerate}

\subsubsection{Using time markers}
Pay attention to the time markers:
\begin{enumerate}[label=(\alph*)]
	\item We use \textbf{every} and \textbf{never} to ask or talk about our experiences in life. \\
		e.g. \textit{Have you \textbf{ever} eaten Chinese food? - I've \textbf{never} eaten it. }
	\item We use \textbf{already} to describe an action which has happened before; \textbf{yet} - an action which hasn't happened before.
		e.g. \textit{I haven't finished this book \textbf{yet}, and my sister has \textbf{already} begun reading another one.}
	\item We use \textbf{just} when we describe a very recent event. \\
		e.g. \textit{My mom has \textbf{just} come home from work.}
	\item \textbf{Always, often, etc.} can also be used in the present perfect. \\
		e.g. \textit{He has \textbf{always} loved Ann.}
	\item We use \textbf{for} to describe the length of a time period. We use \textbf{since} to describe the point when the time period started. \\
		e.g. \textit{Chris has worked here \textbf{for 5 months}. He has worked here \textbf{since May $5^{th}$.} }
\end{enumerate}

\subsection{Past Perfect}
We use the \textbf{past perfect} to show clearly that one past event happened before another past event.

We use the past perfect in the earlier event.

\begin{center}
	\textit{When I arrived at the party, Tom wasn't there. \textbf{He had gone home} }
\end{center}

\subsubsection{How to form the past perfect}
\inline{had}{Positive} + Verb ending in -ed (past participle) or Simple verb. \\
\inline{hadn't}{Negative} + Verb ending in -ed (part participle) or Simple verb.

\subsubsection{Positive, negative and question forms}

\begin{itemize}
	\item (?) \textbf{Had} Kate \textbf{gone} to bed when you arrived home?
	\item (+) \textbf{Yes, she had.} She \textbf{had gone} to bed when I arrived home. She\textbf{'d gone} to bed.
	\item (-) \textbf{No, she hadn't.} She \textbf{hadn't gone} to bed when I arrived home.
\end{itemize}

\subsubsection{Past perfect vs Present perfect}
The \textbf{past perfect} (\textit{I had done)}  is the past of the \textbf{present perfect} (\textit{I have done})

\begin{itemize}
	\item \textbf{Present} \\
		\textit{I\textbf{'m not} hungry. \textbf{I've just had breakfast}. \\
		Your room \textbf{is} dirty. \textbf{You haven't cleaned it for months.} }
	\item Past \\
		\textit{I \textbf{wasn't} hungry. \textbf{I'd just had breakfast.} \\
		Your room \textbf{was} dirty. \textbf{You hadn't cleaned it for months.} }
\end{itemize}


to think, know, be sure, realize, remember, suspect, understand, etc.

\begin{center}
	\textit{She \textbf{was sure} she hadn't locked the door. \\
	When I got home I \textbf{realized} I'd left my computer at Starbucks.}
\end{center}

\subsubsection{Other Uses}
Many speakers use the \textbf{past perfect} (in case of \textbf{before} or \textbf{after}) to show a strong connection between the two events.

\begin{center}
	\textit{Pam left her house before her parents arrived.} (past simple) \\
	\textit{Pam \textbf{had left} her house before her parents arrived.} (past simple + past perfect)
\end{center}


\subsection{Future Perfect}
We use the \textbf{future perfect} to look back from one point in the future to an earlier event.

The situations has not happened yet, but at a certain time in the future it will happen.

\begin{center}
	\textit{By next week \textbf{I'll have written} 20 pages for my new book. }
\end{center}

\subsubsection{How to form the future perfect}
will + have + Verb ending in -ed (past participle) or simple verb.

\subsubsection{Positive, negative and question forms}
\begin{itemize}
	\item (+) John \textbf{will have arrived} here by 5 p.m. tomorrow.
	\item (-) He \textbf{won't have arrived} here by 5 p.m. tomorrow.
	\item (?) \textbf{Will} he \textbf{have arrived} here by 5 p.m. tomorrow?
\end{itemize}

\subsubsection{Time expressions}
by + time expression

\begin{center}
	\textit{Won't they have invited us \textbf{by Friday}? \\
	James will have finished his thesis \textbf{by this time next week.} }
\end{center}

when, as soon as, before, by the time, etc.

\begin{center}
	\textit{Will you have dressed up \textbf{when I pick you up}? \\
	\textbf{By the time you read this} I will have left the city.}
\end{center}

\subsubsection{Other uses}
The \textbf{future perfect} is used only for actions that will be completed by a particular time in the future.

If the \underline{deadline is not mentioned}, use the future simple instead.

\begin{center}
	\textit{She will leave her hometown. \\
	\textst{She will have left her hometown.} \textbf{She will have left her hometown by this time next year.}  }
\end{center}


\section{Prefect Continuous Tense}
\subsection{Present Perfect Continuous}
We use the \textbf{present perfect continuous} to talk about an action (quite a long one) which began in the past and has recently or just stopped.

This tense usually emphasizes:
\begin{itemize}
	\item Duration of the action
	\item That the action is temporary
	\item That the action is repeated
\end{itemize}

\begin{center}
	\textit{- Is it snowing now? \\ - No, it isn't but there is 5 cm of snow outside. \textbf{It has been snowing all night.}}
\end{center}

\subsubsection{How to form the present perfect continuous}
\inline{Have}{Positive} + been + V -ing

\subsubsection{Positive, negative and question forms}
\begin{itemize}
	\item (+) Ann \textbf{has been waiting} for Sam for over an hour
	\item (-) Ann \textbf{hasn't been waiting} for Sam for over an hour. She \textbf{has been waiting} for only 10 minutes.
	\item (?) \textbf{Has Ann been waiting} for Sam for over an hour?
	\item (?) \textbf{How long has Ann been waiting} for Sam?
\end{itemize}

\subsubsection{Time markers}
All day, all morning, for day, for ages, lately, recently, since, for, etc.

\begin{center}
	\textit{My brother has been playing tennis \textbf{since} he was seven. \\
	I haven't been feeling well \textbf{recently}. \
	\textbf{How long} have you been learning English? - I've been learning it \textbf{for  years}.}
\end{center}

Non-continuous verbs (e.g. \textit{to love, hate, know, want, etc.}) are \underline{not used} in any continuous tenses!

\textbf{Use the present perfect instead.}

\begin{center}
	\textit{ \textst{I've been wanting to visit Paris for years.} \\
	\textbf{I've wanted} to visit Paris for years.}
\end{center}

\subsection{Past Perfect Continuous}
We use the \textbf{past perfect continuous} when we talk about an action (quite a long one) which began in the past and continued up until another time in the past.

\begin{center}
	\textit{Sammy \textbf{had been playing} with his food when his mom walked into the kitchen.}
\end{center}

\subsubsection{How to form the past perfect continuous}
had + been + V -ing

\subsubsection{Positive, negative and question forms}
\begin{itemize}
	\item (+) Tom was very tired when he got home. He \textbf{had been working} all day. / He\textbf{'d been working} all day.
	\item (-) Tom wasn't very tired when he got home. He \textbf{hadn't been working} all day.
	\item (?) Why was Tom tired when he got home? \textbf{Had} he \textbf{been working} all day?
\end{itemize}

\begin{table}[h]
\begin{center}
\begin{tabular}{|c|c|}
	\hline
	\textbf{Present} & \textbf{Past} \\ \hline
	I hope the buss \textbf{comes} soon. & At last the bus \textbf{came}. We'\textbf{d been} \\
	We'\textbf{ve been waiting} for 30 minutes. &  \textbf{waiting} for 30 minutes. \\ \hline
	Lilly \textbf{is} out of breath. & Lilly \textbf{was} out of breath. \textbf{She had} \\
	She \textbf{has been running}. & \textbf{been running}. \\ \hline
\end{tabular}
\end{center}
\caption{Present perfect continuous VS Past perfect continuous} \label{tab:ppvspp}
\end{table}

\subsubsection{Time markers}
all day, all morning, for days, for ages, when etc.

\begin{center}
	\textit{Samantha went to the doctor last Monday. She hadn't been feeling well \textbf{for some time}.\\
	My sister had been playing with her friends outside \textbf{for an hour when} it started to rain heavily.}
\end{center}

Non-continuous verbs (e.g. \textit{to love, hate, know, want, etc.}) are \underline{not used} in any continuous tenses! \textbf{Use the past perfect instead.}

\begin{center}
	\textit{We were good friends. \textst{We had been knowing each other for years.}\\
	We were good friends. We had known each other for years.}
\end{center}

\subsection{Future Perfect Continuous}
We use the \textbf{future perfect continuous} when we describe an action (quite a long one) that has begun sometime in the past, present or future, and is expected to continue in the future.

\begin{center}
	\textit{When Petter turns 40, he \textbf{will have been painting} for 35 years.}
\end{center}

\subsubsection{How to form the future perfect continuous}
Will + have + been + V -ing

\subsubsection{Positive, negative and question forms}
\begin{itemize}
	\item (+) At 6 o'clock I \textbf{will have been waiting} here for an hour. / At that time I\textbf{'ll have been waiting} here for an hour.
	\item (-) I \textbf{won't have been waiting} here for an hour at 6 o'clock.
	\item (?) \textbf{Will I have been waiting} here for an hour at 6 o'clock?
\end{itemize}

\subsubsection{by + time expression}
\begin{center}
	\textit{ \textbf{By 2025} he'll have been living in London for 10 years.}
\end{center}

\textbf{When, as soon as, before, by the time, etc.}

\begin{center}
	\textit{ \textbf{When} I complete my studies, I'll have been learning English for 17 years.}
\end{center}

Non-continuous verbs (e.g. \textit{to love, hate, know, want, etc.}) are \underline{not used} in any continuous tenses! \textbf{Use the future perfect instead.}

\begin{center}
	\textit{ \textst{In March I'll have been knowing you for a year.} \\
	In March I'll have known you for a year.}
\end{center}

\section{Helping Verbs (Auxiliary Verbs)}
We use \textbf{auxiliary verbs} (also known as \textbf{helping verbs}) to form:

\begin{itemize}
	\item Questions
	\item Negative sentences
	\item Compound tenses (the perfect tense or the continuous tense)
	\item The passive voice
\end{itemize}

The basic auxiliary verbs are \textbf{to be, to do, to have}.

\newpage
\subsection{To Be}
\textbf{To be} can be used as an auxiliary and a main verb.

\begin{center}
	\textit{My sister \textbf{is} kind.} (Main verb) \\
	\textit{My sister \textbf{is cooking} dinner} (Auxiliary verb; helps to build the present continuous tense)
\end{center}

\begin{table}[h]
\begin{center}
\begin{tabular}{|c|c|}
	\hline
	\textbf{Base form} & be \\ \hline
	\textbf{Present form}  & am/is/are \\ \hline
	\textbf{Past form} & was/were \\ \hline
	\textbf{Present Participle/Gerund} & being \\ \hline
	\textbf{Past Participle} & been \\ \hline
\end{tabular}
\end{center}
\caption{The verb \textbf{to be} is irregular} \label{tab:tobe1}
\end{table}

\subsubsection{Uses}
\begin{enumerate}[label=(\alph*)]
	\item When you don't want to repeat something. \\
		e.g. \textit{Everyone was working that day, but I \textbf{wasn't}.} (=I wasn't working.)
	\item To deny something or to say that is not true \\
		e.g. \textit{You're being unreasonable. - No, \textbf{I'm not.}} (=I'm not being unreasonable.)
	\item To show interest in what somebody has said, or to show surprise. \\
		e.g. \textit{Kelly and Peter are dating. - \textbf{Are they?} Really?}
	\item With so (when you agree) and neither/nor (when you disagree). \\
		e.g. \textit{I'm sleepy. - \textbf{So am I}.} (=I'm sleepy too.) \\
		\textit{My parents are never late. - \textbf{Neither are mine}.} (=My parents are never late either.)
\end{enumerate}

\subsection{To Do and To Have}
The verbs \textbf{to do} and \textbf{to have} can be used as auxiliary and main verbs.

\begin{center}
	\textit{My sister \textbf{does} her own taxes.} (Main verb) \\
	\textit{\textbf{Do} you believe in ghosts?} (Auxiliary verb)  \\
	\textit{Ann \textbf{has} a well-paying job.} (Main verb ; AmE) \\
	\textit{Ann \textbf{has got} a well-paying job.} (Auxiliary verb; BrE)
\end{center}

\begin{table}[h]
\begin{center}
\begin{tabular}{|c|c|c|}
	\hline
	\textbf{Base form} & do & have \\ \hline
	\textbf{Present form}  & do/does & have/has \\ \hline
	\textbf{Past form} & did & had \\ \hline
	\textbf{Present Participle/Gerund} & doing & having \\ \hline
	\textbf{Past Participle} & done & had \\ \hline
\end{tabular}
\end{center}
\caption{The verbs \textbf{to do} and \textbf{to have} are irregular} \label{tab:tdth1}
\end{table}

\subsubsection{Uses}
You can use the auxiliary verbs \textbf{to do} and \textbf{to have}:
\begin{enumerate}[label=(\alph*)]
	\item When you don't want to repeat something. \\
		e.g. \textit{Everyone likes going to the movies but I \textbf{don't}.} (=I don't like going to the movies.)
	\item To deny something or say that it is not true. \\
		e.g. \textit{Have you ever been abroad? - No, I \textbf{haven't}.} (=I haven't been abroad.)
	\item To show interest in what somebody has said, or to show surprise. \\
		e.g. \textit{They have been married for 50 years. - \textbf{Have they?} That's unbelievable! }
	\item With so (when you agree) and neither/nor (when you disagree). In this case, an auxiliary verb goes before the subject. \\
		e.g. \textit{She has helped me a lot. - \textbf{So have I!}} (=I have helped you too.) \\
		\textit{I don't want to go to work. - \textbf{Neither do I.}} (=I don't want to go to work either.)
\end{enumerate}

\subsection{Modal Verbs}
We use \textbf{modal verbs} to show if we believe something is certain, probable or possible (or not).\\

We also use modal verbs to ask permission, make requests and offers etc.\\

\textbf{Modal verbs} fall into the category of \underline{auxiliary verbs} (also known as \underline{helping verbs} ). It means that they are used together with a main verb to give grammatical information and additional meaning to a sentence.

\subsubsection{Can and Could}
The modal verb \textbf{can} has only two forms: \textit{can}(present) and \textit{could}(past).\\We use:

\begin{table}[h]
\begin{center}
\begin{tabular}{|c|c|}
	\hline
	\textbf{Can} & \textbf{Could} \\ \hline
	To talk about general abilities or & To talk about general abilities or \\
	skills int the \textbf{present} . & skills int the \textbf{past} \\
	\textit{I \textbf{can} cook and bake.} & \textit{I \textbf{could} paint beautifully as a kid.} \\ \hline
	To make general statements about & To make general statements about \\
	what \textbf{is} possible/impossible (not allowed). & what \textbf{was} possible/impossible (not allowed) \\
	\textit{It \textbf{can} be very hot in summer.} & \textit{It \textbf{could} be very hot in summer.} \\
	\textit{You \textbf{can't} smoke here.} & \textit{He \textbf{couldn't} do it! He is such a sweet guy.} \\ \hline
	To ask for permission (informal). & To ask for permission (formal) \\
	\textit{\textbf{Can} I borrow your pencil, please?} & \textit{\textbf{Could} I use your phone, please?} \\ \hline
	To request something (informal). & To request something (formal). \\
	\textit{\textbf{Can} you help me, please?} & \textit{\textbf{Could} you show me the way, please?} \\ \hline
	To make offers & To make suggestions \\
	\textit{\textbf{Can} I carry these bags for you?} & \textit{We \textbf{could} go to the bar if you want.} \\ \hline
\end{tabular}
\end{center}
\caption{Can and Could uses} \label{tab:cancould1}
\end{table}

\subsubsection{Must}
You can use the modal verb \textbf{must}:

\begin{enumerate}[label=(\alph*)]
	\item To express obligation, duty, or prohibition (this also refers to laws and regulations). \\
		e.g. \textit{You \textbf{must} wear a seatbelt at all times. \\
		You \textbf{mustn't} use your smartphone while driving.}
	\item To emphasize the necessity of something. \\
		e.g. \textit{People \textbf{must} drink a lot of water during the day.}
	\item To express our certainty in something being true. \\
		e.g. \textit{Look! There are puddles everywhere. \textbf{It must have rained.}\\
		You are still working? \textbf{You must be tired!} }
	\item To give a strong recommendation. \\
		e.g. \textit{You \textbf{must} listen to this song, it's so catchy!}
\end{enumerate}

\subsubsection{May (Modal verb)}
You can use the modal verb \textbf{may}:

\begin{enumerate}[label=(\alph*)]
	\item To give permission or prohibit something. \\
		e.g. \textit{If you have finished the test, you \textbf{may} leave the room. \\
		You \textbf{may not} park here.}
	\item To ask for permission (more polite than \textit{can}). \\
		e.g. \textit{\textbf{May} I use your bathroom, please?}
	\item To express wishes. \\
		e.g. \textit{\textbf{May} you both live happily!}
	\item In academic (or scientific) language to refer to things that typically happen in certain situations. \\
		e.g. \textit{Drivers \textbf{may} feel tired after driving for 3 hours straight.}
\end{enumerate}

Note that we usually use the modal verbs \textbf{may} and \textbf{might} without a significant difference in meaning when expressing \textbf{possibility}.

However, \textbf{might} often implies a smaller chance of something happening.

\begin{center}
 	\textit{I \textbf{might} go to the movies tonight. I'm not sure.}
 \end{center}

\subsubsection{Shall}
Nowadays, the most common use of \textbf{shall} in everyday English is in questions that serve as offers or suggestions (\textbf{Shall I? Shall we?}).

\begin{center}
	\textit{ \textbf{Shall I} order some pizza? \\
	\textbf{Shall we} go now? It's getting late.}
\end{center}

\subsubsection{Should}
You can use the modal verb \textbf{should}:

\begin{enumerate}[label=(\alph*)]
	\item To give advice, a recommendation, or a suggestion. \\
		e.g. \textit{I think you \textbf{should} study more.}
	\item To express that a situation is likely in the present or in the future (a prediction). \\
		e.g. \textit{Kelly \textbf{should} be at home by now. You can stop by.\\
		I ordered some t-shirts 10 days ago. They \textbf{should} come in mail this week.}
	\item To express an obligation (not as strong as \textbf{must}). It is used instead of must to make rules, orders or instructions sound more polite. \\
		e.g. \textit{You \textbf{should} never lie to your parents.}
	\item To say that something was expected in the past but didn't happen (in this case, use \textit{should + have + past participle}). \\
		e.g. \textit{I \textbf{should have studied} more but I was too lazy.}
\end{enumerate}

\subsubsection{Will}
You can use the modal verb \textbf{will}:
\begin{enumerate}[label=(\alph*)]
	\item To express rapid decision. \\
		e.g. \textit{Which one? - Hmm, I \textbf{will} have the tuna sandwich.}
	\item To express thoughts or beliefs about the future. \\
		e.g. \textit{I think they \textbf{will} remain friends forever.}
	\item To make an offer, a promise, or a threat. \\
		e.g. \textit{I \textbf{will not} disappoint you!}
	\item To talk about predictable behaviour. \\
		e.g. \textit{He \textbf{will} eat chocolate when he feels anxious}
\end{enumerate}

We use \textbf{won't} when someone refuses to do something.

\begin{center}
	\textit{I tried reassuring him, but he \textbf{won't} listen to me.}
\end{center}

\subsubsection{Would}
You can use the modal verb \textbf{would}:
\begin{enumerate}[label=(\alph*)]
	\item As a polite invitation or to offer. \\
		e.g. \textit{\textbf{Would} you like to spend this evening together with me?}
	\item To describe a prediction. \\
		e.g. \textit{It \textbf{would} be nice to be a little bit funnier.}
	\item Not to sound impolite when disagreeing with someone. \\
		e.g. \textit{I \textbf{wouldn't} put it like that.}
	\item To describe past habits. \\
		e.g. \textit{She \textbf{would} fall asleep when she was on a train.}
\end{enumerate}

We use \textbf{wouldn't} when someone refused to do something.

\begin{center}
	\textit{James said that he \textbf{wouldn't} help us at all.}
\end{center}

\section{Adjectives}
An \textbf{adjective} is a word or set of words that modifies (i.e. describes) a noun or pronoun.

\textbf{Adjectives} may come before or after the word they modify.

\begin{center}
	\textit{This is a \textbf{cute} cat. This cat is \textbf{cute}.}
\end{center}

Adjectives can modify nouns (e.g. \textit{girl, boy, etc.}) or pronouns (e.g. \textit{we, it, etc.}).

\begin{center}
	\textit{Lilly is an \textbf{honest} person. \\
	The movie was \textbf{awful}! The plot is simply \textbf{boring}.}
\end{center}

Remember that if something is \underline{-ing}, it makes you \underline{-ed}.

\begin{center}
	\textit{He is \textbf{excited} because the event is \textbf{exciting}.\\
	I am \textbf{annoyed} because this whole situation is \textbf{annoying}.}
\end{center}

Sometimes we use two or more adjectives together.

\begin{table}[h]
\begin{center}
\begin{tabular}{|c|c|c|c|c|}
	\hline
	\textbf{Article} & \textbf{Quantity/Number} & \textbf{Quality/Opinion} & \textbf{Fact} & \textbf{Noun} \\ \hline
	a &     & nice        & sunny              & morning \\ \hline
	  & two & intelligent & young              & ladies  \\ \hline
	a &     & beautiful   & large round wooden & table   \\ \hline
\end{tabular}
\end{center}
\caption{More than 1 adjective structure} \label{tab:adj1}
\end{table}

There are times when we use two or more fact adjectives.

\begin{table}[h]
\begin{center}
\begin{tabular}{|c|c|c|c|c|c|c|c|}
	\hline
	\textbf{Article} & \textbf{Size} & \textbf{Age} & \textbf{Shape} & \textbf{Colour} & \textbf{Material/Origin} & \textbf{Purpose} & \textbf{Noun} \\ \hline
	a & big & old & round & & wooden & & table \\ \hline
	  & & new &  & white & & tennis & shoes \\ \hline
	a & tall & young & & & Polish & & boy \\ \hline
\end{tabular}
\end{center}
\caption{More than 1 fact adjective structure} \label{tab:adj2}
\end{table}

\subsection{Descriptive Adjectives}
\textbf{Descriptive adjectives} describe nouns or pronouns in detail by giving an attribute to that partiular word.

They usually express things through the five senses (touch, taste, sight, smell, and sound).

\begin{center}
	\textit{This is a \textbf{delicious} sandwich.}
\end{center}

\textbf{Descriptive adjectives} can be organized into the following categories:

\begin{enumerate}[label=(\alph*)]
	\item Simple adjectives are the most basic type of descriptive adjectives \\
		e.g. \textit{It was a \textbf{beautiful} day yesterday. \textbf{Clear} sky, \textbf{sweet} smell of blossoming trees, \textbf{green} grass, \textbf{cheerful} people... It seemed as if the world has united to celebrate the coming of spring.}
	\item Compund adjectives are created when two words are combined to create a descriptive adjective. The two words are typically connectes with a hyphen. \\
		e.g. \textit{Pam was a \textbf{baby-faced long-legged} girl.}
\end{enumerate}

\subsection{Proper Adjectives}
\textbf{Proper adjectives} are formed from proper nouns and modify nouns and pronouns.

\begin{center}
	\textit{I love \textbf{Italian} culture.}
\end{center}

A \textbf{proper noun} is the specific name used for any person, place, or thing.

\textbf{Proper adjectives} typically look like their original proper nouns but have some sort of alternative ending.

\begin{center}
	\textit{He lives in \textbf{America}. } (proper noun) \\
	\textit{He likes \textbf{American} holidays.} (proper adjective)
\end{center}

\textbf{Proper adjectives} are derived from proper nouns. For this reason, \textbf{they are capitalized}.

\begin{center}
	\textit{When she lived in Chine, Liz ate a lot of \textbf{Chinese} food.}
\end{center}

When a proper adjective has a prefix, the prefix itself is never capitalized. However, the proper adjective itself is still capitalized.

\begin{center}
	\textit{In \textbf{pre-Columbian} America corn was the only cultivated cereal.}
\end{center}

\begin{table}[h]
\begin{center}
\begin{tabular}{|c|c|c|c|c|}
	\hline
	\textbf{-ian/-ean/-an} & \textbf{-ic} & \textbf{-ese} & \textbf{-i} & \textbf{-ish} \\ \hline
	Italian & Icelandic & Chinese & Iraqi & Danish \\ \hline
	Korean & Nordic & Japanese & Israeli & Finnish \\ \hline
	Moroccan & Hispanic & Portuguese & Pakistani & Irish \\ \hline
\end{tabular}
\end{center}
\caption{Common proper adjective endings} \label{tab:pa1}
\end{table}

\subsection{Limiting Adjectives}
Limiting adjectives help to define or 'limit' a noun or pronoun by telling which one, what kind, or how many.

\begin{center}
	\textit{\textbf{This} sandwich is delicious.}
\end{center}

There are the following categories of limiting adjectives:

\begin{enumerate}[label=(\alph*)]
	\item \textbf{Article}  are the most commonly used adjectives. \textbf{A, an, the} indicate whether the noun is used indefinitely or definitely. \\
		e.g. \textit{There is \textbf{a} bed, \textbf{a} mirror, \textbf{a} wardrobe, and \textbf{an} easel in \textbf{the} room.}
	\item \textbf{Demonstrative adjectives} are adjectives that are used to modify a noun so that we know which specific person, place, or thing is mentioned. The most common demonstrate adjectives are \textbf{this, that, these, those.} \\
		e.g. \textit{\textbf{This} is July, and \textbf{that} girl other there is Judy.}
	\item \textbf{Numerals} can function as limiting adjectives limiting the noun to a specific number or amount. \\
		e.g. \textit{\textbf{One} chocolate bar, \textbf{two} cups of coffee, and \textbf{ten} hours of hard work were put into this.}
	\item \textbf{Indefinite adjectives} are used to describe a noun in a non-specific sense. The most common indefinite adjectives are \textbf{any, each, few, many, much, most, several, some.}  \\
		e.g. \textit{There were \textbf{several} people in the room.}
\end{enumerate}


\subsubsection{Possessive Adjectives}
In this category (limiting adjectives) there are \textbf{possessive adjectives}. They modify the noun following it in order to show possession.

These adjectives are: my, your, his, her, its, our, their.

\begin{center}
	\textit{I told \textbf{my} friend that I like someone, then she told that to \textbf{her} friend, and that friend told that to \textbf{his} friends, and now everyone knows everything. }
\end{center}

\begin{table}[h]
\begin{center}
\begin{tabular}{|c|c|c|c|c|}
	\hline
	\textbf{Person} & \textbf{Subject} & \textbf{Object} & \textbf{Possessive Adjective} & \textbf{Possessive Pronoun} \\ \hline
	First Singular & I & me & \textbf{my} & mine \\ \hline
	Second Singular & You & you & \textbf{your} & yours \\ \hline
	\multirow{3}{*}{Third Singular} & He & him & \textbf{his} & his \\ \hline
	 & She & her & \textbf{her} & hers \\ \hline
	 & It & it & \textbf{its} & its \\ \hline
	First Plural & We & us & \textbf{our} & ours \\ \hline
	Second Plural & They & them & \textbf{Their} & theirs \\ \hline
	Third Plural & You & you & \textbf{your} & yours \\ \hline
\end{tabular}
\end{center}
\caption{Possessive adjectives} \label{tab:posa1}
\end{table}

The \textbf{possessive adjective} needs to agree with the possessor and not with the thing that is possessed.

However, the verb that is used needs to be in agreement with the noun.

\begin{center}
	\textit{She has a boyfriend. \textbf{Her} boyfriend is very kind. \
	Peter likes to cook. \textbf{His} cooking skills are great.}
\end{center}

Possessive adjectives are often confused with possessive pronouns.

\begin{center}
	\textit{\textbf{Your} cat is black.} (\textbf{Your} is an adjective which modifies the word 'car') \\
	\textit{\textbf{Mine} is white.} (\textbf{Mine} is a pronoun which functions as the subject of the sentence.)
\end{center}

Do not confuse \textbf{its} and \textbf{it's}

\textbf{Its} is the \textbf{possessive adjective} for \textbf{it}.

\textbf{It's} is a \textbf{contraction} of \textbf{it is}.

\begin{center}
	\textit{\textbf{It is} a beautiful day.} = \textit{\textbf{It's} a beautiful day.}\\
	\textit{The dog was wiggling \textbf{its} tail.}
\end{center}

Do not confuse \textbf{their} and \textbf{they're}

\textbf{Their} is the \textbf{possessive adjective} for \textbf{they}.

\textbf{They're} is a \textbf{contraction} for \textbf{they are}.

\begin{center}
	\textit{\textbf{They are} best friends.} = \textit{\textbf{They're} best friends.} \\
	\textit{I wanted to see \textbf{their} performance.}
\end{center}

\subsubsection{Pronominal Adjectives}
In this category (limiting adjectives) there are \textbf{pronominal adjectives}. They are pronouns which are used to modify nouns.

\begin{center}
	\textit{\textbf{This} book is interesting.} (\textbf{This} is a pronominal adjective. It modifies the noun \textbf{book}.) \\
	\textit{\textbf{This} is an interesting book.} (\textbf{This} is a pronoun. It represents the noun \textbf{book}.)
\end{center}


Pronominal adjectives can be subdivided into the following groups:

\begin{enumerate}[label=(\alph*)]
	\item \textbf{Demonstrative adjectives} \textit{(this, that, these, those)}. \\
		e.g. \textit{\textbf{Those} shoes were old-fashioned. \textbf{These} shoes are much better.}
	\item \textbf{Possessive adjectives} \textit{(my, your, his, her, its, our, their)}. \\
		e.g. \textit{\textbf{Their} cat likes to sleep on the floor.}
	\item \textbf{Distributive adjectives} \textit{(each, every, either, neither)}. \\
		e.g. \textit{\textbf{Every} attempt was met with suspicion.}
	\item \textbf{Interrogative adjectives} \textit{(which, what, whose)}. \\
		e.g. \textit{\textbf{Whose} pants are these?}
	\item \textbf{Indefinite adjectives} \textit{(some, any, all, few, several, many, both, little, much, more, most)}. \\
		e.g. \textit{\textbf{Both} parents were present.}
\end{enumerate}


\subsection{Degrees of Adjectives}

Most adjectives can show degree of quality or quantity by forming two degrees of comparison: the comparitive and the superlative degree.

These degrees are formed from the positive degree, which is the usual form of adjectives.

\begin{table}[h]
\begin{center}
\begin{tabular}{|c|c|c|}
	\hline
	\textbf{Positive} & \textbf{Comparative} & \textbf{Superlative} \\ \hline
	This is a tall building. & This building is taller & This is the tallest \\
	& than that one. & building. \\ \hline
\end{tabular}
\end{center}
\caption{Positive, Comparative and Superlative} \label{tab:pcs}
\end{table}

\begin{table}[h]
\begin{center}
\begin{tabular}{|c|c|c|}
	\hline
	\textbf{Positive} & \textbf{Comparative}  & \textbf{Superlative} \\ \hline
	good & better & the best \\ \hline
	bad & worse & the worst \\ \hline
	far & farther/further & the farthest/furthest \\ \hline
	little & less & the least \\ \hline
	much/many & more & the most \\ \hline
\end{tabular}
\end{center}
\caption{Irregular Adjectives} \label{tab:pcs2}
\end{table}

\subsubsection{Comparative adjectives}
noun/pronoun (subject) + verb + comparitive adjective + than + noun/pronoun (object)

\begin{center}
	\textit{My room is \textbf{larger} than Jake's.}
\end{center}

Sometimes the second item of comparison can be omitted.
\begin{center}
	\textit{If you start working out you'll get \textbf{thinner}.}
\end{center}

\begin{itemize}
	\item One syllable + -er \\
		e.g. \textit{smart - smarter}
	\item vowel + cons $\Rightarrow$ double cons + -er \\
		e.g. \textit{big - bigger}
	\item const + -y $\Rightarrow$ const + -i + -er \\
		e.g. \textit{dry - drier}
	\item two syllables + -er OR more \\
		e.g. \textit{happy - happier, tangled - more tangled}
	\item three syllables + more \\
		e.g. \textit{beautiful - more beautiful}
\end{itemize}


\subsubsection{Superlative Adjectives}
noun/pronoun(subject) + verb + the + superlative adjective + noun/pronoun(object)
\begin{center}
	\textit{My room is \textbf{the largest} one in the house.}
\end{center}

Sometimes the group that is being compared with can be omitted.

\begin{center}
	\textit{She is \textbf{the prettiest} (girl in the office).}
\end{center}

\begin{itemize}
	\item one syllable + -est \\
		e.g. \textit{smart - the smartest}
	\item vowel + const $\Rightarrow$ double cons + -est \
		e.g. \textit{big - the biggest}
	\item cons + -y $\Rightarrow$ cons + -i + -est \\
		e.g. \textit{dry - the driest}
	\item two syllables + -est OR the mostl \\
		e.g. \textit{happy - the happiest, tangled - the most tangled}
	\item three syllables + the most \
		e.g. \textit{beautiful - the most beautiful}
\end{itemize}


\section{Adverbs}
An \textbf{adverb} is a word or set of words that modifies verbs, adjectives, or other adverbs.

Usually adverbs modify verbs, telling us how, how often, when, or where something was done.

\begin{center}
	\textit{We walked really \textbf{slowly}.}
\end{center}

An \textbf{adverb} is a word or set of words that modifies:

\begin{enumerate}[label=(\alph*)]
	\item Verbs, telling how, how often, when, or where something was done. \
		e.g. \textit{The cars drove \textbf{fast}.}
	\item Adjectives, making them stronger or weaker. \\
		e.g. \textit{Ann looked \textbf{absolutely} amazing.}
	\item Other adverbs, changing their degree or precision. \\
		e.g. \textit{You're speaking \textbf{too} loudly.}
\end{enumerate}

\subsection{Forming Adverbs}
\begin{itemize}
	\item adjective + -ly \\
		e.g. \textit{slow - slowly}
	\item adjective ending in -l + -ly \\
		e.g. \textit{careful - carefully}
	\item adjective ending in -y $\Rightarrow$ -i + -ly \\
		e.g. \textit{easy - easily}
	\item adjective ending in -able -ible or -le $\Rightarrow$ replace -e with -y \\
		e.g. \textit{probable - probably, terrible - terribly, gentle - gently}
	\item adjective ending in -ic + -ally \\
		e.g. \textit{economic - economically}
\end{itemize}

Adjectives ending in -ly (friendly, lively) can't be made into adverbs by adding -ly. We can use 'in a friendly way/manner' instead.

\begin{center}
	\textit{He talked to me \textbf{n a friendly manner}. }
\end{center}

The following adverbs have the same form as the adjectives: early, fast, hard, high, late, near, straight, wrong.

\begin{center}
	\textit{The train is very \textbf{fast}.} (adjective) \\
	\textit{The train goes \textbf{fast}.} (adverb) \\
\end{center}

The adverb \textbf{well} corresponds to the adjective \textbf{good}.

\begin{center}
	\textit{Tom is a \textbf{good} student. He studies \textbf{well}.}
\end{center}

The adverb \textbf{hardly} is no realted to the meaning of hard. The adverb \textbf{hardly} has the meaning 'almost not'.

\begin{center}
 	\textit{\textbf{Hardly} anyone writes to me the days.} = \textit{Almost no one write to me these days.} \\
	\textit{Susan ate \textbf{hardly} anything.} = \textit{Susan ate almost nothing.}
 \end{center}


\subsection{Adverbs of manner}
\textbf{Adverbs of manner} tell us how something happens.

\begin{center}
	\textit{I \textbf{carefully} read the note left on the counter.}
\end{center}

\textbf{Adverbs of Manner} are usually placed either before the main verb or after the object.

\begin{center}
	\textit{Tom \textbf{quickly} left the building.\
	Tom left the building \textbf{quickly}. }
\end{center}

Note that such adverbs as \textbf{well, badly, hard, fast,} are always placed after the verb.

\begin{center}
	\textit{ \textst{Alice hard worked.} Alice worked \textbf{hard}.}
\end{center}
When ther is more than one verb in a clause, the position of the adverb is very important.

\begin{center}
	\textit{Samuel \textbf{slowly decided} to leave the party.} (The adverb modifies the verb 'decided')\\
	\textit{Samuel decided \textbf{to leave the party slowly}.} (The adverb describes the clause 'to leave the party'.)
\end{center}

Sometimes a writer puts an adverb of manner at the beginning of the sentence to catch the reader's attention.

\begin{center}
	\textit{\textbf{Confidently} she entered the room.}
\end{center}

\subsection{Adverbs of Place}
\textbf{Adverbs of placed} tell us where something happens. They do not modify adjectives or other adverbs.

\begin{center}
	\textit{I'm going \textbf{back} to school in a month.}
\end{center}

Adverbs of place are usually placed after the main verb or after the clause that they modify.

\begin{center}
	\textit{Come \textbf{in}!}\\
	\textit{Helen looked \textbf{around} trying to find a familiar face in the crowd.}
\end{center}

Adverbs of place that end in \textbf{-where} express the idea of location without specifying a specific location or direction.

\begin{center}
	\textit{I couldn't find my cat \textbf{anywhere.}}
\end{center}

Adverbs of place that end in \textbf{-wards} express movement in a particular direction.

\begin{center}
	\textit{Our dog likes to walk \textbf{backwards}.}
\end{center}

With verbs of movement, \textbf{here} means 'towards of with the speaker' and \textbf{there} means 'away from, or not with the speaker'.

\begin{center}
	\textit{You can hang your coat \textbf{here}.} (You are standing near a hanger.)\\
	\textit{And you can put your shoes \textbf{there}.} (You are pointing at the shoes rack. You are not standing near it.)
\end{center}

\textbf{Here} and \textbf{there} are combined with preposition to make many common adverbial phrases.

\begin{center}
	\textit{Could you come \textbf{over here}?\\
	What are you doing \textbf{up there}?}
\end{center}

\textbf{Here} and \textbf{there} are placed at the beginning of the sentence in exclamations or when emphasis is needed.

They are followed by the verb if the subject is a noun or by a pronoun if the subject is a pronoun.

\begin{center}
	\textit{\textbf{Here} comes the train!\\
	\textbf{There} it is! }
\end{center}

\subsection{Adverbs of time}
\textbf{Adverbs of time} tell us when an action happened, for how long, or how often.
Adverbs of time are invariable.

\begin{center}
	\textit{Sorry, I'll call you \textbf{in a minute}.}
\end{center}

Adverbs of time are usually placed at the end of the sentence.

\begin{center}
	\textit{I'll do it \textbf{tomorrow}.}
\end{center}

Sometimes these adverbs can be put at the beginning of the sentence to give different emphasis.

\begin{center}
	\textit{\textbf{Later} they nnotices his absence.}
\end{center}

In the adverbial phrases that tell us for how long something has been happening, \textbf{for} is always followed by an expression of duration, while \textbf{since} is always followed by an expression of a point in time.

\begin{center}
	\textit{They'll be away \textbf{for twenty days.}}\\
	\textit{I haven't seen you \textbf{since June!}.}
\end{center}

Adverbs that tell us how often something happens express the frequency of an action. They are usually placed before the main verb but after auxiliary verbs.

\begin{center}
	\textit{Sarah \textbf{usually} wakes up at \textbf{7 a.m.}\\
	You must \textbf{always} be kind to others.}
\end{center}

The only exception is when the main verb is \textbf{to be}, in which case the adverbs goes after the main verb.

\begin{center}
	\textit{I am \textbf{never} late.}
\end{center}

If you need to use more than one adverb of time in a sentence, use them in the following order: (1) \textbf{How long}, (2) \textbf{How often}, (3) \textbf{When}.

\begin{center}
	\textit{Peter worked at the mall \textbf{for 4 days (1) every week (2) last year (3)}. }
\end{center}

\subsection{Adverbs of Frequency}
\textbf{Adverbs of frequency} tell us how often something happens. They are also used to indicate routine or repeated activities.

\begin{center}
	\textit{I \textbf{always} do my homework.}
\end{center}

These adverbs are usually placed before the main verb but after auxiliary verbs.

\begin{center}
	\textit{Our company \textbf{frequently} has brunches with potential clients. \
	You should \textbf{always} wait for the green light to cross the road.}
\end{center}

The only exception is when the main verb is \textbf{to be}, in which case the adverb goes after the main verb.

\begin{center}
	\textit{We are \textbf{usually} optimistic.}
\end{center}

\begin{table}[h]
\begin{center}
\begin{tabular}{|c|c|c|}
	\hline
	\textbf{Frequency} & \textbf{Adverb of Frequency} & \textbf{Example} \\ \hline
	100\% & always & Sarah \textbf{always} helps her mom with dinner. \\ \hline
	90\% & usually & We \textbf{usually} go out on Fridays. \\ \hline
	80\% & normally/generally & Shaun \textbf{normally} eats breakfast at 8 a.m. \\ \hline
	70\% & often/frequently & They \textbf{often} go to their parents at weekends. \\ \hline
	50\% & sometimes & Peter \textbf{sometimes} forgets his kids' birthdays. \\ \hline
	30\% & occasionally & I \textbf{occasionally} eat vegetarion food. \\ \hline
	10\% & seldom & We \textbf{seldom} go on vacation together. \\ \hline
	0\% & never & They \textbf{never} eat junk food. \\ \hline
\end{tabular}
\end{center}
\caption{Adverbs of Frequency} \label{tab:af1}
\end{table}

We can also use the following expressions when we want to be more specific about the frequency: every day, once a month, twice a yer, three times a day, every other week, daily, monthly, annually, etc.

\begin{center}
	\textit{I usually eat pizza \textbf{once a month}. }
\end{center}

If you need to use more than one adverb of time in a sentence, use them in the following order: (1) \textbf{How long}, (2) \textbf{How often}, (3) \textbf{When}.

\begin{center}
	\textit{Peter worked at the mall \textbf{for 4 days every week last year}. }
\end{center}

\subsection{Adverbs of Degree}
\textbf{Adverbs of degree} tell us about the intensity of something. They are usually placed befor the adjective, adverb, or verb that they modify.

\begin{center}
	\textit{I was \textbf{too} scared to move forward.}
\end{center}

The most common adverbs of degree are \textbf{extremely, quite, just, almost, very, too, enough}, etc.

\textbf{Enough} as an adverbs meaning 'to the necessary degree' goes after the adjective or adverbs that it is modifying.

\begin{center}
	\textit{This bed isn't comfortable \textbf{enough}.}
\end{center}

\textbf{Enough} is often followed by + infinitive or for something/something.

\begin{center}
	\textit{They're not old \textbf{enough to get married}. \\
	This suit is big \textbf{enough for Mike.} }
\end{center}

\textbf{Too} as an adverb meaning 'also' goes at the end of the phrase it modifies.

\textbf{Too} as an adverb meaning 'excessively' goes before the adjective or adverb it modifies.

\textbf{Too} is often followed by to + infinitive or for something/something.

\begin{center}
	\textit{I'd like to go to the cinema \textbf{too}!\\
	Is he \textbf{too young to become a president}? - No, he isn't \textbf{too young for that.}}
\end{center}

Note that there is a big difference in meaning between \textbf{too} and \textbf{very}.

\textbf{Very} expresses a fact while \textbf{too} suggests that there is a problem.

\begin{center}
	\textit{She speaks \textbf{very quickly}. \\
	She speak \textbf{too quickly}. I can't understand her. }
\end{center}

\subsection{Comparative and Superlative Adverbs}
Most adverbs can show degree of quality or quantity by forming two degrees of comparision: \textbf{the comparative degree} and \textbf{the superlative degree}.

These degrees are formed from the positive degree, which is the usual form of adverbs.

\begin{table}[h]
\begin{center}
\begin{tabular}{|c|c|c|}
	\hline
	\textbf{Positive} & \textbf{Comparative} & \textbf{Superlative} \\ \hline
	She eats slowly & She eats more slowly & She eats the most \\
	 & than we do & slowly of us all \\ \hline
\end{tabular}
\end{center}
\caption{Positive, Comparative and Superlative} \label{tab:csad1}
\end{table}

The comparative form is used for comparing two actions or states, while the superlative is used for comparing one action or state with all the others in the same category.

\begin{center}
	\textit{He runs \textbf{faster than} Jack does. But we need to check for sure who runs \textbf{the fastest}. }
\end{center}

adverb ending in -ly + more/the most
e.g. \textit{happily - more happily - the most happily}

adverb ending in -e + -r/-st
e.g. \textit{late - later - the latest}

\begin{table}[h]
\begin{center}
\begin{tabular}{|c|c|c|}
	\hline
	\textbf{Positive} & \textbf{Comparative} & \textbf{Superlative} \\ \hline
	well & better & the best \\ \hline
	badly & worse & the worst \\ \hline
	much & more & the most \\ \hline
	little & less & the least \\ \hline
	far & farther/further & the farthest/the furthest \\ \hline
\end{tabular}
\end{center}
\caption{Positive, Comparative, Superlative examples} \label{tab:csad2}
\end{table}

Note that it's \textbf{impossible} to have comparatives or superlatives of certain adverbs, especially those of:\\
\indent \textbf{Time} (e.g. \textit{daily, yesterday, then})\\
\indent \textbf{Place} (e.g. \textit{there, up, down})\\
\indent \textbf{Degree} (e.g. \textit{very, just, too})

\subsection{Order of Adverbs}
As adverbs are used to modify verbs, adjectives, other adverbs, phrases, clauses, or even entire sentences, they are able to function nearly anywhere in the sentence, depending on their type and what they are modifying.

If we use more than one adverb to describe a verb, there is a general order in which the different categories of adverbs should appear (sometimes called \textbf{the royal order of adverbs}):

\begin{center}
	\textbf{1. Manner, 2. Place, 3. Frequency, 4. Time, 5. Purpose}
\end{center}

Adverbs of manner tell us how something happens, how someone does something, or give character to a description.

\begin{center}
	\textit{Alice sings \textbf{beautifully}.}
\end{center}

Adverbs of place tell us an aspect of location associated with the action of a verb, specifying the direction, distance, movement, or position involved in the action.

\begin{center}
	\textit{We looked \textbf{upwards} at the fireworks.}
\end{center}

Adverbs of frequency tell us how often something happens.

\begin{center}
	\textit{Peter goes abroad \textbf{twice a year.} }
\end{center}

Adverbs of time tell us when or for how long something happens or is the case.

\begin{center}
	\textit{They've been dating \textbf{for 4 years.} }
\end{center}

\section{Prepositions}
A \textbf{preposition} is usually a short word used to link nouns, pronouns, or phrases to other words within a sentence.

\begin{center}
	\textit{If I'm not mistaken, her birthdat \textbf{is in} May. }
\end{center}

Adverbs of purpose tell us why something happens.

\begin{center}
	\textit{The dress is handcrafted and \textbf{hence} expensive. }
\end{center}

Adverbs indicating the attitude and point of view of the speaker or writer usually go at the beginning.

\begin{center}
	\textit{\textbf{Actually}, I don't want to go there.}
\end{center}

Prepositions \textbf{do not} change their form.

\begin{center}
	\textit{I want (what?) \textbf{to} go (where?) \textbf{to} the movies. =
	I had a desire \textbf{to} go \textbf{to} the movies.}
\end{center}

Preposition can consiste of one, two or more words.

\begin{center}
	\textit{Josh went \textbf{to} the club \textbf{instead of} studying for his exams. \\
	There was a huge traffic jam \textbf{in fron of} us.}
\end{center}

Prepositions can be divided into the following categories:
\begin{enumerate}[label=(\alph*)]
	\item \textbf{Prepositions of place} state the position or location of one thing with another. \
		e.g. \textit{Kate works \textbf{at} Starbucks.}
	\item \textbf{Prepositions of time} denote specific time periods. \\
		e.g. \textit{We usually go to our relatives \textbf{at} Christmas.}
	\item \textbf{Prepositions of direction or motion} indicate movement from one place to the other. \\
		e.g. \textit{There's a great pub \textbf{across} the street.}
	\item \textbf{Preposition of manner} express the maner in which something is done. \\
		e.g. \textit{You can't achieve success \textbf{by} doing nothing.}
	\item \textbf{Prepositions of cause, purpose, and reason} indicate why, what for, or because of what something happens. \\
		e.g. \textit{She couldn't attend the metting \textbf{due to} some family issues.}
\end{enumerate}

\subsection{Prepositions of place}
There are many types of prepositions. Among them there are prepositions of place.

They are used to show the position or location of one thing with another. We usually use prepositions of place when we answer the question beginning with \textit{'Where?'}.

\begin{center}
	\textit{\textbf{Where} do you live? - I live \textbf{in} New York.}
\end{center}

There are three main prepositions of place:
\begin{enumerate}[label=(\alph*)]
	\item \textbf{at} denotes specific point or location of something. \\
		e.g. \textit{There's someone standing \textbf{at} the door.} (specific location)\\
		\textit{There weren't many people \textbf{at} the theatre. It's Monday after all.} (specific location)\\
		\textit{Alex lives \textbf{at} number 25 Emerald Street.} (adress)\\
		\textit{Ashley works \textbf{at} Apple.} (company or workplace)
	\item \textbf{in} implies that something is located in an enclosed space or within a larger area.\\
		e.g. \textit{I think I left my phone \textbf{in} the living-room.} (the living-room is part of your house)\\
		\textit{Jake lives \textbf{in} the U.S. He lives \textbf{in} Texas.} (country, state, etc.) \\
	\item \textbf{on} implies that something is located on the surface. \\
		e.g. \textit{Could you grab my phone? It's \textbf{on} the coffee table in the living-room.} (the surface of furniture)\\
		\textit{Jake's sister is \textbf{on} the west coast. She absolutely love the Pacific!} (position along a road, river or by the sea, lake, etc.) \\
		\textit{Alex live \textbf{on} the third floor. } (the floor in the building) \\
		\textit{Sorry, I'll call you back. I'm \textbf{on} the train now.} (public transport) \\
		\textit{My grandparents work \textbf{on} a farm.} (open fields = the surface of the earth)
\end{enumerate}

Sometimes you can use both \textbf{an} and \textbf{in} when you talk about the location, although there is a slight difference in meaning.

Study the following examples:\\

\textit{My siblings are \textbf{at} the mall now.} (You are stating the location in general. Your siblings could be insed the mall, somewhere at the entrance, or at the parking lot.)\\

\textit{My sibling are \textbf{in} the mall now.} (You are specifying that your siblings are inside the mall building.)

\subsection{Prepositions of time}
\textbf{Prepositions of time} are used to denote specific time periods. We usually use prepositions of time when we answer the question beginning with \textit{'When?'} .

\begin{center}
	\textit{\textbf{When} did you move to New York? - I move there \textbf{in} 2007. }
\end{center}

There are three main prepositions of time:
\begin{enumerate}[label=(\alph*)]
	\item \textbf{at} denotes precise time. \\
		e.g. \textit{I'll pick you up \textbf{at} 5.\\
		We're going to be sleeping \textbf{at} midnitgh}  \\

		Note that \textbf{at} is also used with such expressions as \textbf{at night, at weekend, at Christmas, at the moment, at present, at the same time}.

		\begin{center}
			\textit{Mr. Ruffus isn't \textbf{at} the moment. May I take a message?}
		\end{center}
	\item \textbf{on} is used for days and dates. \\
		e.g. \textit{I'm meeting up with my friends \textbf{on} Saturday. And \textbf{on} Sunday morning I'm flying to Seattle.}\\
		\textit{Mike has the project presentation \textbf{on} 11 November.}\\
		\textit{My family does nothing \textbf{on} Christmas day.}
	\item \textbf{in} denotes longer period of time like months, years, centuries, etc. \\
		e.g. \textit{The Parkers are moving to Greece \textbf{in} March.\\
		The story is set \textbf{in} the 80s.\\
		Life \textbf{in} the Middle Ages wasn't like in a fairy tale. I don't know how people lived in the past!} \\

		Note that \textbf{in} is also used with such phrases as \textbf{in the morning/afternoon/evening}.

		\begin{center}
			\textit{Theo is an owl. He has a hard time getting up \textbf{in} the mornings. }
		\end{center}

\end{enumerate}

Note that we do not use prepositions before \textbf{last/next/every/this}.\\
\begin{center}
	\textit{\textst{I guess we'll see Alice on next Monday.}\\
	I guess we'll see Alice \textbf{next} Monday.}
\end{center}

\subsection{Prepositions of Direction and Motion}
They are used to show movement from one place to the other. We usually use prepositions of direction or motion when we answer the question beginning with \textit{'Where?'}.

\begin{center}
	\textit{\textbf{Where} are you going? - I'm going \textbf{to} the supermarket.}
\end{center}

There are several commonly used prepositions of direction or motion:
\begin{enumerate}[label=(\alph*)]
	\item \textbf{to} is used to show movement in a specific direction. \\
		e.g. \textit{I'll head off \textbf{to} work in a couple of minutes.\
		Kimberly moved \textbf{to} Florida a year ago.} \\

		Note that you can also use \textbf{towards} in the meaning 'in the direction of'.
		\begin{center}
			\textit{Why are these policemen running \textbf{towards} Erick? }
		\end{center}
	\item \textbf{into} is used to show movement into something (enclosed space), while \textbf{onto} shows movement on top of something (surface). \\
		e.g. \textit{The dog jumped \textbf{into} the kennel, while the cat leaped \textbf{onto} the roof of the kennel.}
	\item \textbf{across} is used to show movement from one side to the other side of something. \\
		e.g. \textit{You can't walk \textbf{across} the street wherever you want.}
	\item \textbf{over} is used to show an upward and forward direction across something. \\
		e.g. \textit{The boys jumped \textbf{over} the fence and chases the cat.}
	\item \textbf{through} is used to show movement within an enclosed space from one point to the other. \\
		e.g. \textit{I don't like driving \textbf{through} the tunnels. I feel a bit anxious then.}
	\item \textbf{past} is used to indicate movement near something while you are on your way to another location. \\
		e.g. \textit{I waved at Mary but shee walked \textbf{past} me.}
\end{enumerate}

\subsection{Prepositions of Manner}

\textbf{Prepositions of manner} are used to express the manner in which something is done. We usually use prepositions of manner when we answer the question beginning with \textit{'How?'}.

\begin{center}
	\textit{\textbf{How} did she lose weight? - She lost weight \textbf{by} exercising.}
\end{center}

There are several groups of prepositions of manner:
\begin{enumerate}[label=(\alph*)]
	\item \textbf{in, with} are used to describe the way in which something is carried out. \\
		\begin{center}
			\textit{She left the stage \textbf{in} tears.\
			She was singing \textbf{with} tears in her eyes.}
		\end{center}
	\item \textbf{by} is used to denote either a person or a means of transportation, while \textbf{with} denotes an instrument.
		\begin{center}
			\textit{This house was built \textbf{by} my grandfather.}\\
			\textit{Helen goes to work \textbf{by} bus.} \\
			\textit{You need to cut the cake \textbf{with} a knife. }\\
			\textbf{We can also use by + V-ing.}\\
			\textit{You can't prove them wrong \textbf{by} doing nothing.}
		\end{center}
	\item \textbf{at} can be used to describe aggressive behaviour.\\
		Compare the following examples:
		\begin{center}
			\textit{He talked \textbf{to} his wife.} (neutral)\\
			\textit{He talked \textbf{at} his wife.} (aggressive behaviour)
		\end{center}
\end{enumerate}

We can also use the phrase \textbf{in a friendly way/manner} to describe actions.

\begin{center}
	\textit{Mrs. Anderson spoke to me \textbf{in an extremely polite manner}.}
\end{center}

\subsection{Prepositions of Cause, Purpose, and Reason}
They are used to indicate why, what for, or because of what something happens. We usually use these prepositions when we answer the question beginning with \textit{'Why?'}.

\begin{center}
	\textit{\textbf{Why} don't you eat breakfast? - I don't eat breakfast \textbf{to} sleep more in the mornings.}
\end{center}

There are several commonly used prepositions of cause, purpose, and reason:
\begin{enumerate}[label=(\alph*)]
	\item \textbf{due to} is used to express the cause of the action.\\
		e.g. \textit{\textbf{Due to} her strict parents, Liz rarely went out. It was difficult for her to make friends.} (Her parents were the cause of her not making friends.)
	\item \textbf{to}  is used to express the purpose of the action. (usually followed by a verb) \\
		e.g. \textit{People go to clubs \textbf{to} dance and \textbf{to} meet new people.} (These two things are the purpose of people going to clubs.)
	\item \textbf{for} is used to express the reason of the action. (usually followed by a noun/pronoun or a gerund)\\
		e.g. \textit{He was taken to the police station \textbf{for} driving under influence.} (DUI was the reason he was taken to the police station.)
		\begin{center}
			\textbf{\textit{Because of} is also used to express the reason of something happening. }
		\end{center}
		\textit{I need to go home earlier \textbf{because of} my sick cat.} (My cat is sick that's why I need to go home earlier.)
\end{enumerate}

\subsection{Prepositional Phrases}
A \textbf{prepositional phrase} is a group of words consisting of a preposition and a noun, pronoun, gerund or clase.

\begin{center}
	\textit{She tried to calm down the baby \textbf{by singing} lullabies.}
\end{center}

A prepositional phrase always consists of two basic parts at minimum: the preposition and its object.

\begin{center}
	\textit{I think I'll be \textbf{at (preposition) home (noun).} }
\end{center}

A prepositional phrase is a group of words that can consist of:
\begin{enumerate}[label=(\alph*)]
	\item a preposition and a noun. \
		e.g. \textit{Erick was fired \textbf{from McDonald's} }
	\item a preposition and a pronoun. \
		e.g. \textit{He always leaves little presents \textbf{for me}. }
	\item a preposition and a gerund. \\
		e.g. \textit{Carol managed to lose some weight \textbf{thanks to exercising}. }
	\item a preposition and a clause. \\
		e.g. \textit{I need to talk to you \textbf{about stuff we need for our trip}. }
\end{enumerate}

A prepositional phrase can function either as an adjective or an adverb in the sentence.

As an adjective, the prepositional phrase answers the question \textit{'Which one?'}

\begin{center}
	\textit{The boy \textbf{with red hair} was taking photos outside. \
	\textbf{Which one?} The one \textbf{with red hair}.  }
\end{center}

As an adverb, the prepositional phrase answers the questions \textit{'How?/When?/Where?'}

\begin{center}
	\textit{Gaby went for a run \textbf{at 5 o'clock.}\\
	\textbf{When} did she go for a run? \textbf{At 5 o'clock.} }
\end{center}

\section{Conjunctions}
\textbf{Conjunctions} are words that link other words, phrases, clauses, or sentences together.

\begin{center}
	\textit{Susan is an amazing wife \textbf{and} a wonderful mom.}
\end{center}

Conjunctions add complexity to our speech.

They also allow us to form complex sentences instead of using multiple short ones.

\begin{center}
	\textit{Bran likes eating. He doesn't like cooking. He finds cooking boring.\
	Bran likes eating \textbf{but} he doesn't like cooking \textbf{as} he finds it boring.}
\end{center}

Conjunctions can be divided into the following categories:
\begin{enumerate}[label=(\alph*)]
	\item \textbf{Subordinating conjunctions} link two clauses, a main (independent) one and a subordinate (dependent) one.\\
		The most commonly used subordinating conjunctions are \textit{although, as, because, if, though, unless, etc}.\\
		e.g. \textit{She won't speak with her parents \textbf{unless} they apologise first.}
	\item \textbf{Correlative conjunctions} connect two equal grammatical items.\\
		These conjunctions come in pairs: either...or, neither...nor, not only...but also. \\
		e.g. \textit{\textbf{Either} we go to the party \textbf{or} we stay at home.}
	\item \textbf{Compound conjunctions} are phrases which are used as conjunctions. \
		A compound conjunction has two or three words that go together - \textit{so that, as long as, even though, etc.}\\
		e.g. \textit{Mike lied to his parents \textbf{so that} he could go to the party.}
	\item \textbf{Coordinating conjunctions} are used to link words, phrases, and clauses of equal impotance in a sentence. \\
		There are seven coordinating conjunctions: \textit{for, and, nor, but, or, yet, so} (you can remember them with the help of the acronym FANBOYS)\\
		e.g. \textit{Beth doesn't like cheese, \textbf{yet} she eats pizza nearly every day.}
\end{enumerate}

\subsection{Coordinating Conjunctions}
There are many types of conjunctions. Among them there are coordinating conjunctions.

They are used to link words, phrases, and clauses of equal importance in a sentence.

\begin{center}
	\textit{She complains about his job, \textbf{yet} he doesn't try to find a new one.}
\end{center}

There are seven coordinating conjunctions: \textit{for, and, nor, but, or, yet, so} (you can remember them with the help of the acronym FANBOYS)

\begin{center}
	\textit{They could't afford to rent the apartment, \textbf{for} it was to expensive.\
	You can't have your cake \textbf{and} eat it.\\
	Samantha doesn't want to go out, \textbf{nor} does she invte us to her place.\\
	I was quite anxious at the beginning, \textbf{but} eventually I managed to pull myself together.\\
	You can call me \textbf{or} send a message when you get off from work.\\
	Ben says that he is busy all the time, \textbf{yet} he has time to play online games every day.\
	Bill is allergic to dairy, \textbf{so} he doesn't eat any cheese.}
\end{center}

\subsection{Subordinating Conjunctions}
\textbf{Subordinating conjunctions} link two clauses, a main (independent) one and a subordinate (dependent) one.

\begin{center}
	\textit{\textbf{Although} Emma wanted to go together with them, she declined the invitation.}
\end{center}

The most commonly used subordinating conjunctions are: \textit{although, as, because, if, since, though, unless, while, whereas, etc.}

Subordinating conjunctions perform two functions in a sentence: they state the importance of the independent clause and provide a transition between two ideas within a sentence.

\begin{center}
	\textit{[\textbf{Once} she stopped caring about strangers opinions], [Liz becam happier]}\\
	Main clause, Subordinate clause.
\end{center}

If the subordinate clause follows the main one, we do not usually use a comma.

\begin{center}
	\textit{[My mom cries] [\textbf{whenever} she watches a romantic comedy].\\
	Main clause, Subordinate clause.}
\end{center}

If the subordinate clause precedes the main one, use a comma to separate clauses.

\begin{center}
	\textit{[\textbf{After} he had completed his studies], [George decided to travel for a year].\\
	Main clause, Subordinate clause.}
\end{center}

\subsection{Correlative Conjunctions}
They connect two equal grammatical items in a sentence.

\begin{center}
	\textit{\textbf{Either} you apologise \textbf{order} I'll mommy!}
\end{center}

These conjunctions come in pairs: \textit{either...or, neither...nor, not only...but also, rather...than, etc.}

When using correlative conjunctions, pay attention to the subjec-predicate agreement so that you have parallel structures.

\begin{center}
	\textit{\textst{College life is not only about partying, but also study like crazy.}\\
	College life is \textbf{not only} about partying, \textbf{but also} about studying like crazy.}
\end{center}

Note that a negative correlative like \textit{neither...nor} can go at the beginning of a sentence.

In this case, the word order is inverted, and the auxiliary verb comes before the subject.

Compare the sentences:

\begin{center}
	\textit{\textbf{Neither did} Sam clean the apartment \textbf{nor did} he buy groceries.\\
	Sam \textbf{neither cleaned} the apartment \textbf{nor bought} groceries.}
\end{center}

\subsection{Compund Conjunctions}
\textbf{Compund conjunctions} are phrases which are used as conjunctions. A compound conjunction has two or three words that go together, \textit{so that, as long as, even though, etc.}

\begin{center}
	\textit{You can buy whatever you want \textbf{as long ask} you use your own money.}
\end{center}

Even though compound conjunctions have two or three words that go together, they are different from correlative conjunctions which are conjunctions used only in pairs.

Compare the sentences:

\begin{center}
	\textit{Beth likes painting \textbf{as well as} drawing.\
	Beth think thay you can be good \textbf{either} at painting \textbf{or} at drawing.}
\end{center}

There are several commonly used compound conjunctions:

\begin{center}
	\textit{You can buy clothes \textbf{as well as} shoes there.}\\
	(You can buy clothes and shoes there.)\\ \vspace{0.3cm}
	\textit{\textbf{As soon as} it started raining, they opened the windows in the apartment.}\\
	(It started raining and they inmediately opened the windows.)\\ \vspace{0.3cm}
	\textit{The kid sang so loudly \textbf{as if/as though} there was no one in the room.}\\
	(The kids sand so loudly like there was no one in the room.)\\ \vspace{0.3cm}
	\textit{\textbf{Even if/Even though} I don't like plain honey. I'll eat sweets with honey in them.}\\
	(I don't like plain honey but nevertheless I'll eat sweets with honey in them.)\\ \vspace{0.3cm}
	\textit{John will pass the test \textbf{provided that} he studies every day.}\\
	(If John studies every day, he'll pass the test.)\\ \vspace{0.3cm}
	\textit{I'll turn off my phone \textbf{so that} no one distrub us.} \\ \vspace{0.3cm}
	\textit{Please turn off your phones \textbf{in order that} we are not disturbed by anyone. (\textbf{in order that} is more formal than \textbf{so that} )}
\end{center}

\subsection{Pseudo Conjunctions}
Pseudo conjunctions are other parts of speech that act like conjunctions:

\begin{enumerate}[label=(\alph*)]
	\item \textbf{Adverbial conjunctions} (also called conjuctive adverbs) are used to indicate a relationship between sentences and independent clauses by comparing or contrating ideas (e.g. \textit{consequently, finally, however, otherwise, then ,etc.}) We usually use commas to separate an adverbial conjunction from the rest of the sentences \\
		e.g. \textit{Jhon's mom wanted him to go to college. \textbf{Instead}, he took a gap year and travelled around the world. }
	\item \textbf{Nominal conjunctions} introduce or conclude ideas (e.g. \textit{the moment, the instant, etc.}) Nominal conjunctions function as nouns in a sentence. \\
		e.g. \textit{I was petrified \textbf{the moment} I heard the news.}
	\item \textbf{Verbal conjunctions} are used to introduce additional information in a sentence (e.g. \textit{assuming (that), given (that), etc.}) Verbal conjunctions are deriven from verbs. \\
		e.g. \textit{Shall we go out tonight \textbf{assuming that} you are free? }
\end{enumerate}


\section{Phrasal Verbs}
A \textbf{phrasal verb} is a verb that is made up of a main verb together with an adverb or preposition, or both.

Typically, their meaning is no obvious from the meanings of the individual words themselves.

\subsection{Give Up}

To \textbf{give up} has several meanings:

\begin{enumerate}[label=(\alph*)]
	\item To give something up/to give up doing something, means to stop doing something, especially something that you do regularly. \\
		e.g. \textit{Bella \textbf{gave up} her job and became a stay-at-home mom. \\
		Why don't you \textbf{give up} drinking beer?}
	\item to give yourself/somebody up (to), means to allow yourself or someone else to be caught by the police or enemy soldiers. \\
		e.g. \textit{The burglar \textbf{gave himself up (to the police).} }
	\item to give up something, means to use some of your time to do a particular thing. \\
		e.g. \textit{Emily did't like \textbf{giving up time to} do laundry.}
	\item to give something/somebody up, means to give something that is yours to someone else. \
		e.g. \textit{They had to \textbf{give up their} lands.}
	\item to give up on somebody/something, means to stop hoping that someone or something will change or improve. \\
		e.g. \textit{Greg had been in a coma for a year, and doctors \textbf{had} almost \textbf{given up on him}.}
	\item to give yourself up to something, means to allow yourself to feel an emotion completely, without trying to control it. \\
		e.g. \textit{They \textbf{gave themselves up to} laugther after hearing the joke.}
\end{enumerate}

\subsection{Turn out}
To \textbf{turn out} has several meanings:

\begin{enumerate}[label=(\alph*)]
	\item to turn out, means to happen in a particular way, or to have a particular result, especially one that you did not expect. \\
		e.g. \textit{I though I'd failed my exam, but \textbf{it turned out fine}. \\
		\textbf{As it turns out}, they have been dating for over a year.\\
		James \textbf{turned out to be} Lily's cousin. }
	\item to turn out for, means that a lot of people go to watch the event or take part in it. \\
		e.g. \textit{About 80\% of the population \textbf{turned out for} the election.}
	\item to turn somebody out, means to force someone to leave a place permanently, especially their home. \
		e.g. \textit{If you don't pay the rent, they will \textbf{turn you out} in a week. }
\end{enumerate}

\subsection{Carry on}
To \textbf{carry on} has several meanings:
\begin{enumerate}[label=(\alph*)]
	\item to carry on doing something/with something, means to continue doing something. \\
		e.g. \textit{Sorry, I interrupted you. \textbf{Carry on}, please.\\
		You'll put on weight if you \textbf{carry on} eating fast food.\
		I want to \textbf{carry on with} my business idea. }
	\item to carry on, means to continue moving. \\
		e.g. \textit{\textbf{Carry straight on} until you see the red building. }
	\item to carry on something, means to do or take part in a particular kind of work or activity. \\
		e.g. \textit{It was so noisy there that it was difucult for us to \textbf{carry on} a conversation. }
	\item (spoken) to carry on about, means to speak with overwhelming enthusiasm.\\
		e.g. \textit{I wish my friends would stop \textbf{carrying on} about their trip. }
\end{enumerate}

\subsection{Put Off}
\textbf{To put off} has several meanings:
\begin{enumerate}[label=(\alph*)]
	\item to put something off/ to put off something, means to delay doing something or to arrange to do something at a later time or date, especially because there is a problem or you do not want to do it at that time. \\
		e.g. \textit{The game \textbf{has been put off} until tomorrow because of bad weather. \\
		I\textbf{'ve been putting off} working on my thesis because I'm never int the mood. }
	\item to put somebody off/put somebody off (doing) something, means to make you dislike something or not want to do something. \
		e.g. \textit{Don't let his humour \textbf{put you off} - he's a nice guy actually.\\
		I don't want my fears \textbf{put you off} finding a job in another state. }
	\item to put somebody off, means to make someone wait because you do not want to meet them, talk to them etc. until later. \
		e.g. \textit{If my brother calls, \textbf{put him off} as long as possible. }
	\item to put somebody off (something), means to make it difucult for someone to pay attention to what they are doing by talking, making a noise etc. \\
		e.g. \textit{It \textbf{puts me off} when you're listening to music while I'm talking to you. }
\end{enumerate}

\subsection{Turn Down}
\textbf{To turn down} has several meanings:
\begin{enumerate}[label=(\alph*)]
	\item to turn down, means to turn the switch on a machine (e.g. \textit{an oven, radio, etc.}) so that it produces less heat, sound, etc. (opposite to \textbf{to turn up}). \\
		e.g. \textit{Can you \textbf{turn down} the TV? I'm trying to study.\\
		I'll \textbf{turn down} the heater. It's too hot in the room. }
	\item to turn down, means to refuse an offer, request, or invitation.\
		e.g. \textit{Ann offered Peter the job but he \textbf{turned} it \textbf{down}.}
	\item to turn down, means to refuse someone's offer of marriage. \
		e.g. \textit{We were shocked to hear that Lilly \textbf{turned} him \textbf{down}.}
	\item if the economy turns down, it means that the level of activity falls, companies become less profitable, etc. \\
		e.g. \textit{After the crisis in 2008 \textbf{the economy has turned down.} }
\end{enumerate}

\subsection{Break Up}
\textbf{To break up} has several meanings:
\begin{enumerate}[label=(\alph*)]
	\item to break up, means to break into a lot of small pieces. \\
		e.g. \textit{The vase just \textbf{broke up} in my hands.}
	\item to break up, means to separate something into several smaller parts. \\
		e.g. \textit{I think that their intention is to \textbf{break up} our company into several smaller ones. }
	\item to break up, means to stop a fight. \\
		e.g. \textit{Their mom was the one to \textbf{break up} fights.}
	\item to break up, means to make people leave a place where they have been meeting or protesting. \\
		e.g. \textit{Police \textbf{broke up} the demonstration. }
	\item to break up (with) (when speaking of marriage, group of people, or relationship) indicates that the people in this relationship separate and do not live or work together anymore. \\
		e.g. \textit{I was so sad to hear that my favourite band \textbf{broke up.}\\
		James \textbf{broke up with} Kate last year. }
\end{enumerate}

\section{Pre-determiners}
\textbf{Predeterminers} are words placed before determiners in a sentence, i.e. they modify the determiner.

\begin{center}
	\textit{\textbf{What} a great day!}
\end{center}

Predeterminers are usually placed before an \textit{indefinite article + adjective + noun} to express an opinion about the noun they modify.

Predeterminers can be classified into the following categories:
\begin{enumerate}[label=(\alph*)]
	\item \textbf{Multipliers} (twice, three times) are used to express more than the specified amount. \\
		e.g. \textit{My brothers make \textbf{twice} my annual salary. \\
		I try to call my parents at least \textbf{three times} a week. }
	\item \textbf{Fractions} (half, one-eight) are used to express a fraction of an amount. \\
		e.g. \textit{The bus will arrive in \textbf{half} and hour. We've got plenty of time. \\
		I ate \textbf{one-third} of the pizza we cooked last night. }
	\item \textbf{Intensifiers} (such, what, quite, rather) are used to express surprise, disappointment, pleasure, or other emothins.
		\begin{center}
			\textit{Such} and \textbf{what} are used to express surprise or other emotions. \\
		e.g. \textit{Alice is \textbf{such} a kind person!\\
		\textbf{What} a fantastic meal it is! }
		\end{center}
		\begin{center}
			\textit{Quite} and \textbf{rather} refer to the degree of a particular quality. They can express disappointment, pleasure, or other emotions depending on the adjective. \\
			e.g. \textit{Actually, it was \textbf{quite} a nice meal, I am suprised.\\
			He's alwats been \textbf{rather} a difficult child. } (BrE)
		\end{center}
	\item Other pre-determiners such as \textit{all, both} do not fall into the other groups. They are used to express the entire amount. \\
		e.g. \textit{Jake broke \textbf{both} his legs when hiking.\\
		How did you manage to read \textbf{all} these books in one week? }
\end{enumerate}

\section{Passive vs. Active Voice}
In sentences written in the \textbf{active voice}, the subject performs the action.

In sentences written in the \textbf{passie voice}, the subject receives the action.

\begin{center}
	\textit{I \textbf{wrote} a book.} (Active Voice) \\
	\textit{That book \textbf{was written} by me.} (Passive Voice)
\end{center}

The passive form is made up of the varb \textbf{to be} and the \textbf{past participle}. Depending on the tense, the form of the verb to be can change.

\begin{center}
	\textit{The dinner \textbf{is being cooked}.} (Present Continuous)\\
	\textit{The dinner \textbf{was cooked}.} (Past simple)\\
	\textit{The dinner \textbf{has been cooked}.} (Present Perfect)
\end{center}

The passive voice is usually used:
\begin{enumerate}[label=(\alph*)]
	\item To emphasize the action rather than the person or thing perfoming it. \\
		e.g. \textit{The decision \textbf{has been made}.}
	\item To avoid mentioning the person or thing performing the action. \\
		e.g. \textit{The rumours \textbf{have been spreading} at the office.} (Either you know who spreads the rumours or you are not sure who does that.)
	\item To describe a situation where the subject is not important. \\
		e.g. \textit{Up to 7 billion trees \textbf{are being cut down} avery year.}
	\item To give instructions, set rules, etc. \\
		e.g. \textit{Somking \textbf{is prohibited}.\\
		Anyone under the age of 18 \textbf{is not allowed} in any bar.}
\end{enumerate}

\section{Conditionals}
\textbf{Conditionals} are sentences with two clauses, a main clase and and if clause.

Conditionals state that the action in the main clause can only take place if a certain condition in the if clause is fulfilled.

\begin{center}
	\textit{\textbf{If} we don't hurry, we will be late!}
\end{center}

The order of the main and if clauses is not fixed. Although when the if clause precedes the main one, use a comma.

\newpage
There are five main type of conditionals in Enlgish:

\begin{enumerate}[label=(\alph*)]
	\item \textbf{Zero conditionals} are used to describe things are always or generally true. Thus we refer to the real and possible situations, general truths, or scientific facts. Zero conditionals follo the pattern:
		\begin{table}[h]
		\begin{center}
		\begin{tabular}{|c|c|}
			\hline
			\textbf{If + present simple} & \textbf{Present simple} \\ \hline
			\textbf{If} the food is too spicy & drink milk \\ \hline
		\end{tabular}
		\end{center}
		\caption{Zero Conditionals Pattern} \label{tab:zcp}
		\end{table}
	\item \textbf{Conditionals type 1} or first conditionals are used to describe future events that will happen or are likely to happen. These sentences are based on facts, thus we make statements about the real world or particular situation. First conditionals follow the patter:
		\begin{table}[h]
		\begin{center}
		\begin{tabular}{|c|c|}
			\hline
			\textbf{If + present simple} & \textbf{future simple}\\ \hline
			\textbf{If} everything goes according to the plan & we'll be very rich. \\ \hline
		\end{tabular}
		\end{center}
		\caption{Conditionals Type 1 Pattern} \label{tab:ct1}
		\end{table}
	\item \textbf{Conditionals type 2} or second conditionals are used to describe hypothetical, unlikely, or impossible situations. These sentences are not based on facts, this we can refer to any time. Second conditionals follow the patter:
		\begin{table}[h]
		\begin{center}
		\begin{tabular}{|c|c|}
			\hline
			\textbf{If + past simple} & \textbf{would + V} \\ \hline
			\textbf{If} I won the lottery, & I would put the money in the bank. \\ \hline
		\end{tabular}
		\end{center}
		\caption{Conditionals Type 2 Pattern} \label{tab:ct2}
		\end{table}

	\item \textbf{Conditionals type 3} or third conditionals are used to describe a past event that is different to what really happened. These sentences are solely hypothetical, thus there is always some implication of regret. Third conditionals follow the pattern:
		\begin{table}[h!]
		\begin{center}
		\begin{tabular}{|c|c|}
			\hline
			\textbf{If + past perfect} & \textbf{would have + Ved/past participle} \\ \hline
			\textbf{If} we hadn't slept in, & we wouldn't have missed our flight. \\ \hline
		\end{tabular}
		\end{center}
		\caption{Conditionals Type 3 Pattern} \label{tab:ct3}
		\end{table}
	\item \textbf{Mixed conditionals} refer to conditional sentences that combine two different types of conditional patterns. They are used to refer to a time in the past, and a situation that is ongoin in the present. Mixed conditionals usually follow the pattern:
		\begin{table}[h!]
		\begin{center}
		\begin{tabular}{|c|c|}
			\hline
			\textbf{If + past perfect} & \textbf{would + V} \\ \hline
			\textbf{If} they had argued less, & they would be a perfect couple. \\ \hline
		\end{tabular}
		\end{center}
		\caption{Mixed Conditionals Pattern} \label{tab:mi}
		\end{table}

\end{enumerate}
\subsection{Conditionals Zero Type}
\textbf{Zero conditionals} are used to describe things that are always or generally true. Thus we refer to the real and possible situations, general truths, or scientific facts.

\begin{center}
	\textit{\textbf{If} two people fall in love, they become a couple.} (In general, people become a couple if the fall in love each other.)
\end{center}

Use the present simple tense in both parts of the zero conditionals. Note that the order of the main and \textit{if} clauses is not fiex, Although when the \textit{if} clause precedes the main one, use a comma.

\begin{table}[h]
\begin{center}
\begin{tabular}{|c|c|}
	\hline
	\textbf{IF clause} + \textbf{MAIN clause} \\ \hline
	\textbf{If + present simple} & + \textbf{present simple} \\ \hline
	\textbf{If} the temperature \textbf{is} above 0 degrees outside, & the snow \textbf{melts.} \\ \hline
	\textbf{If} my friend \textbf{invites me over}, & I always \textbf{accept}   her invitation. \\ \hline
\end{tabular}
\end{center}
\caption{Conditionals Zero Type} \label{tab:czt}
\end{table}

Note that we can use \textit{when} instead of \textit{if} without any changes in the meaning.

\begin{center}
	\textit{\textbf{When} winter comes, the birds fly to the south. }
\end{center}

Zero conditionals are often used to give instructions. In this case, we use the imperatives in the main clasue.

\begin{center}
	\textit{\textbf{Call me} if you need any help.}\\
	\textit{If you are not satisfied with your major, \textbf{change it}. }
\end{center}

\subsection{Conditionals Type 1}
\textbf{Conditionals type 1} or first conditionals are used to describe future events that will happen or are likely to happen. These sentences are based on facts, thus we make statements about the real world or particular situation.

\begin{center}
	\textit{\textbf{If} you don't study, you will fail the exam.} (Sometimes you can pass an exam without studying, but this time it won't work.)
\end{center}

Use the present simple tense in the \textit{if} clause and the future simple tense in the main clause. Note that the order of the main and \textbf{if} clauses is not fixed. Although when the \textbf{if} clause precedes the main one, use a comma.

\begin{table}[h]
\begin{center}
\begin{tabular}{|c|c|}
	\hline
	\textbf{IF clause} & \textbf{MAIN clause} \\ \hline
	\textbf{if + present simple} & \textbf{future simple} \\ \hline
	\textbf{If} the weather is great, & we'll go to the park. \\ hiking
	\textbf{If} you don't stop fighting with each other, & you two will be grounded! \\ \hline
\end{tabular}
\end{center}
\caption{Conditionals Type 1} \label{tab:ct1t}
\end{table}

Note that it is possible to use other present tenses (e.g. \textit{present continuous, present perfect}) in the \textit{if} clause.

\begin{center}
	\textit{If \textbf{you're going}, I'll go too!} (if + present continuous, future simple)\\
	\textit{If \textbf{they've already received} your information, they will let you know.} (if + present perfect, future simple)
\end{center}

\subsection{Conditionals Type 2}
\textbf{Conditionals type 2} or second conditionals are used to describe hypothetical, unlikely, or impossible situations. These sentences are not based on facts, thus we can refer to any time.

\begin{center}
	\textit{\textbf{If} Peter cleaned his place, he would let use come in.}
\end{center}

Use the past simple tense in the \textit{if} clause and \textit{would + the base form of the verb} in the main clause. Note that the order of the main and \textit{if} clauses is not fixed. Although when the \textit{if} clauses precedes the main one, use a comma.

\begin{table}[h]
\begin{center}
\begin{tabular}{|c|c|}
	\hline
	\textbf{IF clause} & \textbf{MAIN clause} \\ \hline
	\textbf{if + past simple} & \textbf{would + V} \\ \hline
	\textbf{If} I won 1 million dollars, & I would give it to charity. \\ \hline
	\textbf{If} you found a formal black dress, & it would look perfect on you. \\ \hline
\end{tabular}
\end{center}
\caption{Conditionals Type 2} \label{tab:ct2t}
\end{table}

Note that if we use the verb \textit{to be} in the \textit{if} clause, the form \textit{were} is used even with the $1^{st}$ and $3^{rd}$ person. We often use \textit{'if I were you...'} to express our opinion or to give advice.

\begin{center}
	\textit{\textbf{If I were} a chef, I would work at some Italian restaurant. }\\
	\textit{\textbf{If I were you}, I wouldn't take that job.}
\end{center}

Compare the first conditional and the second conditional:

\begin{center}
	(It is December right now.) \textit{If it snows today, we will definitely make a snowman.}\\
	(It is May right now.) \textit{If it snowed today, we would be surprised.}
\end{center}

\subsection{Conditionals Type 3}
\textbf{Conditionals type 3} or third conditionals are used to describe a past event that is different to what really happened. These sentences are solely hypothetical, thus there is always some implication of regret.

\begin{center}
	\textit{\textbf{If} we hadn't booked this trip, we wouldn't have missed her graduation ceremony.} (We wanted to be at her graduation ceremony but we missed it because we ere on a trp somewhere else.)
\end{center}

Use the past perfect tense in the \textit{if} clause and \textit{would have + past participle} in the main clause. Note that the order of the main and if clauses is not fixed. Although when the \textit{if} clause precedes the main one, use a comma.

\begin{table}[h]
\begin{center}
\begin{tabular}{|c|c|}
	\hline
	\textbf{IF clause} & \textbf{MAIN clause} \\ \hline
	\textbf{If + past perfect} & \textbf{would have + Ved/past participle} \\ \hline
	\textbf{If} we had bought that lottery ticket, & we would have won. \\ \hline
	\textbf{If} Jake hadn't drunk that night, & he wouldn't have got into the car accident. \\ \hline
\end{tabular}
\end{center}
\caption{Conditionals Type 3} \label{tab:ct3t}
\end{table}

Note that both \textit{would} and \textit{had} can be contracted to \textit{'d}. Remember that \textit{would} never appears in the \textit{if} clause.

\begin{center}
	\textit{\textbf{If} I \textbf{had} know that, I \textbf{would} have warned you. = \textbf{If} I\textbf{'d} known that, I\textbf{'d} have warned you.}
\end{center}

\subsection{Mixed Conditionals}
\textbf{Mixed conditionals} refer to conditional sentences that combine two different types of conditional patterns. The mixed conditional is used to refer to a time in the past, and a situation that is ongoing in the present.

\begin{center}
	\textit{\textbf{If} I had won the lottery, I would buy a huge house.} (I didn't win the lottery in the past and I'm living in a small apartment right now.)
\end{center}

\newpage
The most common mixed conditional is when we have a third conditional in the \textit{if} clause (\textit{if + past perfect}) followed by a second conditional (\textit{would + the base form of the verb}) in the main clause. Note that the order of the main and \textit{if} clauses is not fixed. Although when the \textit{if} clause precedes the main one, use a comma.

\begin{table}[h!]
\begin{center}
\begin{tabular}{|c|c|}
	\hline
	\textbf{IF clause} & \textbf{MAIN clause} \\ \hline
	\textbf{If + past perfect} & \textbf{would + V} \\ \hline
	\textbf{If} Kate had studied more, & she would have a better GPA. \\ \hline
	\textbf{If} he had started painting the picture in June, & it would be finished now. \\ \hline
\end{tabular}
\end{center}
\caption{Mixed Conditionals} \label{tab:mct}
\end{table}
The less common mixed conditional is when we have a second conditional in the \textit{if} clause (\textit{if + past simple}) followed by a third conditional (\textit{would have + past participle}) in the main clause. This conditional refers to an unreal present situation and its possible (but unreal) past result.

\begin{center}
	\textit{\textbf{If} I weren't afraid of flying, I would have travelled by air.} (I am afraid of flying in general. And that time I travelled by train.)
\end{center}

\section{Clauses}

A \textbf{clause} is a combination of words containing a subject and a predicate.

\begin{center}
	\textit{Peter goes to the gym.} (one clause)\\
	\textit{Peter goes to the gym after he finishes his work.} (two clauses)
\end{center}

There are two types of clauses in English:

\begin{enumerate}[label=(\alph*)]
	\item An independent (main) clause contains a subject and a predicate and expresses a finished thought. Thus it can stand alone as a sentence.\\
		e.g. \textit{Pam likes drawing and painting.\\
		Andy is currently unemployed.}\\

		Note that the independent clause is a simple sentence when dependent clauses within one sentence are absent.

	\item A dependent (subordinate) clause gives aditional information to the main sentence, thus it cannot stand alone as a separate sentence.\\
		e.g. \textit{Mr. Klarkson, \textbf{whose works are critically acclaimed,} has published a new book. \\
		Mary started laughing \textbf{when she saw a pug wearing a costume.} }
\end{enumerate}

An independent clause forms a complex sentence together with a dependent clause.

\begin{center}
	\textit{I'd like to know \textbf{why I can't book a room at his hotel.}\\
	The woman stood crying \textbf{as people were passing by.} }
\end{center}

\subsection{Independent Clause}
An \textbf{independent (main) clause} contains a subject and a predicate and expresses a finished thought. Thus it can stand alone as a sentence.

\begin{center}
	\textit{Sarah wants to study Spanish.}
\end{center}

The independent clause is a simple sentence when dependent clauses withing one sentence are absent:

\begin{center}
	\textit{I don't want to go to the pub tonight.} (simple sentence)\\
	\textit{I have to work tomorrow.} (simple sentence)
\end{center}

\newpage
The independent clause forms a complex sentence together with a dependent clause. In this case, usa a conjunction.

\begin{center}
	\textit{I don't want to go to the pub tonight \textbf{because} I have work tomorrow.} (complex sentence consisting of an independent clause and a dependent one; two clauses are connected with the help of a conjunction \textbf{because})
\end{center}

Two independent clauses can form a sentence. In this case, use a semicolon (;).

\begin{center}
	\textit{My little sister doesn't like reading; she falls asleep within 2 minutes.}\\
	\textit{Lucy has a business trip in a week; Lucy's mom will help with the kids.}
\end{center}

\subsection{Dependent Clause}
A \textbf{dependent (subordinate) clause} gives additional information to the main sentence, thus it cannot stand alone as a separate sentence.

\begin{center}
	\textit{When she comes back home after a long day at work, she likes to take a bath.}
\end{center}

Dependent clauses can be divided into the following categories:

\begin{enumerate}[label=(\alph*)]
	\item A \textbf{noun clause} is a dependent clause that acts as a noun, this it can be a subject, an object, or an object of a preposition in the sentence.\\
		e.g. \textit{\textbf{Whoever comes first} wins!}\\
		\textit{We don't know \textbf{who left the note at the door.} }

	\item An \textbf{adjective clause} is a dependent clause that modifies nouns or pronouns providing additional information. \\
		e.g. \textit{A women \textbf{who can cook well} will become my wife.}\\
		\textit{Broccoli, \textbf{which not everyone likes,} are part of my daily ration.}

	\item An \textbf{adverb clause} is a dependent clause that modifies an adjective, an adverb, or a verb/verb phrase.\\
		e.g. \textit{We were swimming in the ocean \textbf{when we saw the lightning.}\\
		Let's eat dinner \textbf{before the food gets cold.} }
\end{enumerate}

\subsubsection{Noun Clause}
A \textbf{noun clause} is a dependent clause that acts as a noun.

\begin{center}
	\textit{She loves \textbf{violet}.} (noun)\\
	\textit{I know \textbf{that she loves violet}.} (noun clause)
\end{center}

A noun clause can begin with words such as \textit{what, who, when, where, whether, which, why, how, etc.}

\begin{center}
	\textit{I don't know \textbf{who called me.}} \\
	\textit{It's important to state in your application \textbf{why you want to work at the company.} }
\end{center}

A noun clause can act as a subject, an object, or an object of a preposition.

\begin{center}
	\textit{\textbf{Why he did this} was beyond my understanding.} (subject)\\
	\textit{We would like to know \textbf{whether you see yourself coming back to our resort next year.}} (object)\\
	\textit{She told us about \textbf{how she managed to get her intern position.}} (object of a preposition)
\end{center}


\subsubsection{Adjective Clause}
An \textbf{adjective clause} is a dependent clause that modifies nouns or pronouns providing additional information.

\begin{center}
	\textit{The house \textbf{where we were born} was demolished last month.} (the adjective clause modifies the noun 'house')
\end{center}

An adjective clause can begin with words such as \textit{that, who, whom, whose, which, when, where,} and \textit{why}.

\begin{center}
	\textit{Children \textbf{whose parents spend a lot of time with them} are bound to be happier.}
\end{center}

There are two types of adjective clauses:

\begin{enumerate}[label=(\alph*)]
	\item A \textbf{restrictive (essential) adjective clause} provides information that is necessary to distinguish the modified word, thus it cannot be omitted. These clauses usually begin with \textit{that} and are not set off with commas. \\
		e.g. \textit{The English course \textbf{that Ann takes} is aimed at written skills.} (There are different types of English courses, but the peculiarity of the course that Ann takes is that it is aimed at written skills.)

	\item A \textbf{non-restrictive (non-essential) adjective clause} provides additional information, thus it can be omitted without any loss of meaning. These clauses usually begin with \textit{which} and are always set off with commas. \\
		e.g. \textit{Bananas, \textbf{which I eat daily}, are packed with nutrients and vitamins.} (Bananas are very healthy. \textbf{By the way, I eat them every day.} $\Rightarrow$ This additional information doesn't change the fact that bananas are healthy.)
\end{enumerate}

\subsubsection{Adverb Clause}
An \textbf{adverb clause} is a dependent clause that modifies an adjective, an adverb, or a verb/verb phrase providing additional information.

\begin{center}
	\textit{Give me a call \textbf{when you get home}.} (the adverb clause modifies the verb phrase)
\end{center}

An adverb clause can begin with words such as \textit{after, because, since, until, when, etc.}

\begin{center}
	\textit{We were at the beach \textbf{when it started to rain}.}\\
	\textit{Mike is running every day \textbf{as he is going to run a marathon in a month.} }
\end{center}

An adverb clause can be placed at the beginning and the end of the sentence without a change in meaning. Use a comma if the clause is placed at the beginning of the sentence.

\begin{center}
	\textit{You should brush your teeth \textbf{before yo go to bed.}}\\
	\textit{\textbf{Before you go to bed}, you should brush your teeth. }
\end{center}

\section{Reported Speech}

\end{document}
