\documentclass[hidelinks,10pt,a4paper]{article}
\usepackage[utf8]{inputenc}
\usepackage[english]{babel}
\usepackage{amsmath}
\usepackage{amsfonts}
\usepackage{amssymb}
\usepackage{soul}
\usepackage[margin=1in]{geometry}
\usepackage{enumitem}
\usepackage{adjustbox}
\usepackage[driverfallback=hypertex]{hyperref}

\newlength{\drop}

\usepackage{tikz}

\newcommand{\inline}[2]{%
    \begin{tikzpicture}[baseline=(word.base), txt/.style={shape=rectangle, inner sep=0pt}]% the baseline key ensures that nodes won't shift up if there's text with descenders, and the txt style removes extra spacing so you can use this inline
    \node[txt] (word) {#1};% the first argument is the contents of the main node
    \node[above] at (word.north) {\footnotesize{#2}};% the second argument is the tag; you can play with the positioning as necessary
    \end{tikzpicture}%
    }

\begin{document}

\begin{titlepage}

\drop=0.1\textheight
    \centering
    \vspace*{\baselineskip}
    \rule{\textwidth}{1.6pt}\vspace*{-\baselineskip}\vspace*{2pt}
    \rule{\textwidth}{0.6pt}\\[\baselineskip]
    {\LARGE ENGLISH\\[0.2\baselineskip] NOTES}\\[0.2\baselineskip]
    \rule{\textwidth}{0.4pt}\vspace*{-\baselineskip}\vspace{3.2pt}
    \rule{\textwidth}{1.6pt}\\[\baselineskip]
    \scshape
    From A1 to C2 \\
    \vspace*{2\baselineskip}
    Edited by \\[\baselineskip]
    {\Large MAXIMILIANO PONCE\par}
    {\itshape My notes from \\Langpill english grammar course from Udemy, and internet\\\par}
    \vfill
    {\scshape April 2020} \\
    {\large MAXIMILIANO PONCE}\par

\end{titlepage}

\tableofcontents
\newpage

\section{Introduction}
\indent
This document was written for me to understand the english grammar from a beginner to an advanced level. Because I find that I can unsertand better if I take notes for myself while I'm watching english lessons, and I decided to make my notes with \LaTeX  beacause it is a usefull and cool tool for writing awesome docuements.\\

\indent I know that I a need to improve my english writing and as I improve it I will update this document. I hope my note become useful to other people so this document is divided in the main topics of english grammar, so that you can go directly to the topic that you want to learn.\\

\indent
Remember that is important to practice on your own to master the English language, another advice for you is to speak with a friend of you who wants to learn English or already knows it, or even to your self, about any topic you like and if you are stuck to explain something then you can search on internet or ask to your friend to get feedback.\\

\newpage

\section{Nouns}
A \textbf{noun} is a word that is used to name a person, animal, place, action or thing either generally (\textbf{common noun}) or specifically ( \textbf{proper noun)}.
\begin{center}
		\textit{ You can buy a} \inline{ \textit{\textbf{pencil}}}{Common noun}
		\textit{at} \inline{\textit{\textbf{Office Depot}}}{Proper noun}.
\end{center}

\subsection{Plural form}
There are rules to form plural forms, but remember that some nouns are irregular so that you can't use those rules to form the plural form.

\begin{itemize}
		\item Generally we can use $\Rightarrow$ \underline{Noun $+$ S}, to get the plural form \\e.g.\textit{cat-cats, dog-dogs, pencil-pencils}
		\item Singular noun ending in \underline{$s$, $ss$, $sh$, $ch$, $x$, $z$, $o$ $+$ $es$}, $\Rightarrow$ Plural form \\ e.g. \textit{ tax-taxes, bus-buses, box-boxes}
		\item In some cases, singular nouns ending in \underline{$s$ or $z$} $\Rightarrow$ Double $s$ or $z$ \\
			e.g \textit{fez-fezzes, gas-gasses}
		\item Noun ending in \underline{$f$, $fe$} $\Rightarrow$ Change $f$ into $ve$ $+$ $s$ \\e.g. \textit{life-lives, wolf-wolves, wife-wives} \quad (Exceptions: \textit{roof-roofs, belief-beliefs, chef-chefs})
		\item Noun ending in $y$ and the letter before is a \underline{consonant} $\Rightarrow$ Change the ending to $ies$ \\
			e.g. \textit{city-cities, puppy-puppies}
		\item Noun ending in $y$ and the letter begfore is a \underline{vowel} $\Rightarrow$ Add $s$ \\
			e.g. \textit{ray-rays, boy-boys}
		\item Noun ending in $o$ $\Rightarrow$ Add $es$ \\
			e.g. \textit{potato-potatoes, tomato-tomatoes} \quad (Exceptions: \textit{photo-photos, piano-pianos, halo-halos}
		\item Nound ending in $us$ $\Rightarrow$ Frequently change $us$ to $i$ \\
			e.g. \textit{cactus-cacti, focus-foci}
		\item Noun ending in $is$ $\Rightarrow$ Change $is$ to $es$ \\
			e.g. \textit{analysis-analyses, ellipsis-ellipses}
		\item Noun ending in $on$ $\Rightarrow$ Change $on$ to $a$ \\
			e.g. \textit{phenomenon-phenomena, criterion-criteria}
		\item Some nouns \underline{don't change} \\
			e.g. \textit{sheep-sheeps, series-series, species-species}
\end{itemize}
Here we have some \textbf{irregular nouns}, they don't follow specific rules\\
\begin{table}[h]
\begin{center}
\begin{tabular}{|c|c|}
\hline
\textbf{Singular} & \textbf{Plural}\\
\hline
man & men\\ \hline
woman & women\\\hline
person & people\\\hline
child & children\\\hline
tooth & teeth\\\hline
foot & feet\\\hline
mouse & mice\\
\hline
\end{tabular}
\end{center}
\caption{\label{tab:nouns1}Some irregular nouns}
\end{table}


\subsection{Common nouns}
\textbf{Common nouns} refer to classes or categories of people, animals, places, things, or a concept, as opposed to a particular individual.
\begin{center}
	\textit{I have a \textbf{computer}, a \textbf{keyboard}, a \textbf{mouse} and many \textbf{books}. }
\end{center}
\indent
\textbf{Common nouns} are \textbf{not capitalized} unless they begin a sentence or are part of a title.
\begin{center}
		\textit{
		\textbf{Apples} are delicius fruits\\
				I don't like \textbf{apples} }
\end{center}

\subsection{Proper nouns}
\textbf{Proper nouns} are used to name to specific items rather than refer to a category or a class, such as names, names of cities, countries, etc.
\begin{center}
		\textit{I'm from \textbf{Mexico} }
\end{center}
Note that proper nouns are unique names. \textbf{They are capitalized}
\begin{center}
	\textit{My friend \textbf{George} is from \textbf{Brazil}  }
\end{center}
We should also capitalize:
\begin{enumerate}[label=\alph*)]
		\item Festiavals\\
				e.g. \textit{\textbf{Christmas} and \textbf{Thanksgiving} are my two favourite holidays!}
		\item People's titles\\
				e.g. \textit{Everything depends on \textbf{President} Trump and his decisions.}
		\item The names of books, films, plays, paintings. We use capital letters for the nouns, adjectives, and verbs in the title.\\
				e.g. \textit{I've just finished reading \textbf{'The Old Man and the Sea'}}
\end{enumerate}
Sometimes we use a person's name to refer to something they have created.
\begin{center}
\textit{We were listening to \textbf{Mozart} the other day.}\\
\textit{I'm reading \textbf{an Iris Murdoch} now.}
\end{center}

When you use a word about a family member (e.g. \textit{mom, dad, uncle}), capitalize it only if the word is being used exactly as you would use a name, i.e. if you were addressing the person directly. If the word is not being used as a name, it is not capitalized.
\begin{center}
		\textit{Please ask \textbf{Dad} if he can buy wine on his way home.}\\
		\textit{Is your \textbf{dad} coming over for dinner?}
\end{center}
Whenever you see a capitalized word, question whether or not it is a proper noun. Make sure that the capitalized word is in fact a noun as there are also proper adjectives.
\begin{center}
		\textit{ \textbf{Asia} is one of the continents of the wold.} (proper noun)\\
		\textit{I don't like \textbf{Asian} food.} (proper adjective).
\end{center}

\subsection{Material Nouns}
\textbf{Material nouns} denote a material or substance from which things are made of.
\begin{center}
		\textit{		a \textbf{plastic} bottle, a \textbf{diamond} ring, etc.}
\end{center}
Material nouns are uncountable, thus they do not have a plural form. Generally, articles are not used with material nouns as they are uncountable.
\begin{center}
\textit{
		\textst{I really want to buy these cottons pants.}\\
I really want to buy these \textbf{cotton} pants.}
\end{center}

\newpage
Material nouns fall into several categories:
\begin{center}
		\begin{enumerate}[label=\alph*)]
		\item Related to nature\\
				e.g. air, salt, coal, silver, gold, etc.
		\item Related to animals\\
				e.g. meat, milk, egg, wool, etc.
		\item Related to plants\\
				e.g. cotton, coffee, tea, wood, etc.
		\item Arificial or man-made materiales\\
				e.g. alcohol, cheese, brick, steel, etc.
		\end{enumerate}
\end{center}

\subsection{Compund nouns}
A \textbf{compund noun} contains two or more words which are joined together and form a single noun. Compund nouns can be words written together, words that are hyphenated, or separate words.\\
The first word usually describes or modifies the second word, denoting its type or purpose, Consequently, the second word identifies the item itself.
\begin{center}
\textit{
I need to buy a new \textbf{toothbrush}.} ( a brush used for cleaning one's teeth)
\end{center}

There is no exact rule as to when we should write compund nouns together, hyphenated, or as separate words. If you are not sure how to write a compund nound, \textbf{consult a dictionary}.
\begin{center}
\textit{
		Could you go with me to the \textbf{bus stop}?\\
		My \textbf{in-laws} are incredible people.\\
I love your new \textbf{haircut}! You look fantastic!}
\end{center}
Note that the stress usually falls on the first syllable in compund nouns. As a result, the word stress helps to differentiate between a compund noun and an adjective + noun.
\begin{center}
\textit{
A \textbf{greenhouse} is a glass building used for growing plants that need warmth, light, and protection.} (compund noun)\\
\textit{
A \textbf{green house} is a building that someone lives in. This building is painted green.} (adjective + noun)
\end{center}

\newpage
\subsection{Countable vs Uncountable Nouns}
\begin{table}[h]
\begin{center}
\begin{tabular}{|c|c|}
		\hline
		\textbf{Countable Nouns (e.g. apple, song, house, etc.)} & \textbf{Uncountable Nouns (e,g, tea, money, love, etc.)}\\
		\hline
		\parbox[t]{8cm}{\vspace{0.08cm} Things that \textbf{can be counted}, even if the number\\ might be extremely high ( \textit{e.g. all the people in\\ the world}).\vspace{0.08cm}} & \parbox[t]{8cm}{\vspace{0.08cm}Things that we \textbf{cannot count} with numbers.\\ They may be the names for abstract ideas or\\ qualities or for physical objects that are too\\ small to count or shapeless ( \textit{e.g. liquids, gases, etc.}).\vspace{0.4cm}}\\
		\hline
		\parbox[t]{8cm}{\vspace{0.08cm}Can be singular or plural.\\ \textit{I have an \textbf{apple} and you have three \textbf{apples}.}\vspace{0.08cm}} & \parbox[t]{8cm}{ \vspace{0.08cm}No plural form.\\\textit{We're goint to have \textbf{rice} for lunch.}\vspace{0.4cm}}\\
		\hline
		\parbox[t]{8cm}{\vspace{0.08cm}You can use \textit{a/an} with singular countable nouns.\\ \textit{There is \textbf{a girl} outside. She is wearing \textbf{a beautiful dress}.}\vspace{0.08cm}} & \parbox[t]{8cm}{\vspace{0.08cm}You can't use \textit{a/an} wih uncountable nouns. But \\ you can often use the phrase \textit{a (bag, cup, etc.) of}.\\ \textit{There is \textbf{a bowl of rice} and \textbf{a bottle of juice} on the table.} \vspace{0.4cm}}\\
		\hline
\parbox[t]{8cm}{\vspace{0.08cm}If you want to ask about the quantity of a\\ countable noun, you ask \textit{'How many?'} combined \\ with the plural countable noun.\\ \textit{ \textbf{How many dogs} are there? - There are \textbf{five dogs.}}\vspace{0.08cm}} & \parbox[t]{8cm}{\vspace{0.08cm}If you want to ask about the quantity of an \\ uncountable noun, you ask \textit{'How much?'}\\ combined with the uncountable noun.\\ \textit{ \textbf{How much coffee} do we have left? - We don't have \textbf{much coffee} left.}\vspace{0.4cm}} \\
\hline
\parbox[t]{8cm}{\vspace{0.08cm}You can use \textit{many, a few, few} with plural\\ countable nouns.\\ \textit{Sorry, but I didn't take \textbf{many pictures.}\\I've got \textbf{a few relatives} leaving here.}\vspace{0.08cm}} & \parbox[t]{8cm}{{\vspace{0.08cm}You can use \textit{much, a little, little} with uncountable\\ nouns.\\ \textit{We didn't do \textbf{much shopping} there. \\ We have \textbf{a little sugar} left.} }\vspace{0.4cm}} \\
\hline
\multicolumn{2}{|c|}{ \parbox[t]{16cm}{\begin{center}You can use \textit{some, any, a lot of, both} with plural countable nouns and uncountable nouns. \end{center}}} \\
\hline
\textit{We like singing \textbf{some crazy songs} at karaoke.} & \textit{We listened to \textbf{some music} there.}\\
\hline
\textit{Did you buy \textbf{any oranges}?} & \textit{I didn't buy \textbf{any orange juice}.}\\
\hline
\textit{She showed \textbf{a lot of signs} of affection.} & \textit{There is \textbf{a lot of love} in the air.}\\ \hline

\end{tabular}
\end{center}
\caption{Countable vs Uncountable Nouns} \label{tab:nouns2}
\end{table}

\subsection{Collective Nouns}
A collective noun is used to refer to an entire group of people, animals, or things.\\
Therefore it includes more than one member.

\begin{center}
		\textit{My \textbf{family} is very big.}
\end{center}

Collective nouns can refer to:
\begin{enumerate}[label=\alph*)]
		\item People\\
				e.g. \textit{family, class, committee, staff, etc.}
		\item Animals\\
				e.g. \textit{a pack of dogs, a swarm of flies, a herd of horses, a litter of puppies, etc.}
		\item Things\\
				e.g. \textit{pack, set, bunch, stack, etc,}
\end{enumerate}
\newpage
When the members within one group behave in the same manner, they are part of a collective noun, thus this noun becomes singular and requires a singular verb.
\begin{center}
		\textit{Every day \textbf{the football team} follows its coach out to the field for practice.}
\end{center}

When the members are acting as individuals, the collective noun is plural and requiresa plural verb.\\
In many cases, it may sound more natural to make the subject plural in form by adding words like \textit{members, mates, etc.}
\begin{center}
		\textit{After the practice \textbf{the team(mates)} shower, change into their casual clothes, and head to their homes.}
\end{center}

\subsection{Concrete and Abstract Nouns}
\hspace{0.8cm} Nouns can be concrete or abstract. \\
\textbf{Concrete nouns} are tangible and you can experience them with your five senses.\\
\textbf{Abstract nouns} refer to intangible things, like \textit{actions, feelings, ideals, concepts, and qualities}.
\begin{center}
		\textit{ \textbf{Food} is great. But \textbf{love} is even greater.}
\end{center}
\subsubsection{Concrete nouns}
A \textbf{concrete noun} is a noun that can be identified through one of the five senses: \textit{touch, sight, hearing, smell, or taste}.
\begin{center}
		\textit{Who turned off the \textbf{TV}?} (The noun \textit{TV} is a concrete noun)\\
		\textit{What is that \textbf{noise}?} (Even though \textit{nose} can't be touched, you can hear it, so it's a concrete noun)
\end{center}
Concrete nouns fall into several categories:
\begin{enumerate}[label=\alph*)]
		\item People\\
				e.g. \textit{mother, friend, teacher, stranger, etc}.
		\item Places\\
				e.g. \textit{school, McDonald's, Las Vegas, India, etc}.
		\item Things you can touch and see\\
				e.g. \textit{plane, cup, lamp, book, etc}.
		\item Things you can hear\\
				e.g. \textit{music, noise, someone's voice, song, etc.}
		\item Things you can smell and taste\\
				e.g. \textit{herbs, cookies, bread, wine, etc.}
\end{enumerate}

\subsubsection{Abstract Nouns}
Remember that \textbf{abstract nouns} refer to a intangible things, like \textit{actions, feelings, ideals, concepts, and qualities}.\\
Abstract nouns fall into several categories:
\begin{enumerate}[label=\alph*)]
		\item Emotions and feelings\\
				e.g. \textit{anger, sadness, love, grief, etc.}
		\item Human qualities and characteristics\\
				e.g. \textit{beauty, maturity, humour, patience, etc.}
		\item Ideas and concepts\\
				e.g. \textit{knowledge, freedom, luxury, comfort, etc.}
		\item Events\\
				e.g. \textit{marriage, birthday, career, adventure, etc.}
\end{enumerate}

Many abstract nouns are formed from adjectives, verbs, or nouns. Sometimes you can add a suffix to the concrete noun or alter the word root to form abstract nouns.
\begin{center}
		\inline{ \textit{child}}{(Concrete noun)} \inline{ \textit{child\textbf{hood}}}{(abstract noun)}
\end{center}
\newpage
Nouns with the following suffixes are often abstract:
\begin{table}[h]
\begin{center}
\begin{tabular}{|c|c|c|c|c|c|}
		\hline
		\multicolumn{2}{|c|}{\textbf{-tion} e.g. \textit{devotion}} & \multicolumn{2}{|c|}{\textbf{-ism} e.g. \textit{pessimism}} & \multicolumn{2}{|c|}{\textbf{-ity} e.g. \textit{hospitality}}\\ \hline
		\multicolumn{2}{|c|}{\textbf{-ment} e.g. \textit{movement}} & \multicolumn{2}{|c|}{\textbf{-ness} e.g. \textit{restlesness}} & \multicolumn{2}{|c|}{\textbf{-age} e.g. \textit{marriage}} \\ \hline
		\multicolumn{2}{|c|}{\textbf{-ance} e.g. \textit{brilliance}} & \multicolumn{2}{|c|}{\textbf{-ence} e.g. \textit{indifference}} & \multicolumn{2}{|c|}{\textbf{-ship} e.g. \textit{relationship}} \\ \hline
		\multicolumn{3}{|c|}{\textbf{-ability} e.g. \textit{vailability}} & \multicolumn{3}{|c|}{\textbf{-acy} e.g. \textit{bureacracy}} \\ \hline
\end{tabular}
\end{center}
\caption{Common suffixes for abstract nouns} \label{tab:nouns3}
\end{table}

\subsection{Possessive Nouns}
The \textbf{Possessive} form is used with \textbf{nouns} referring to people, groups of people, countries, and animals.\\
It shows a relationship of belonging between one thing and another.
\begin{center}
		\textit{ \textbf{Lelie's} aunt is a doctor.}
\end{center}
To form the possessive, add an \textbf{apostrophe + -s} to the noun.
\begin{center}
\textit{My \textbf{brother's} computer was stolen a week ago.\\
		\textbf{Children's} toys were on the ground.\\}
\end{center}
If the noun already \textbf{ends in -s}, just add an \textbf{apostrophe}.
\begin{center}
\textit{ \textbf{Student's} homework will be assessed later.}
\end{center}
For names \textbf{ending in -s}, you can either add an \textbf{apostrophe + -s}, or just an \textbf{apostrophe}. The first option is more common.
\begin{center}
\textit{They want to sell \textbf{Jame's }car.}
\end{center}
Study some of the fixed expressions where the possessive form is used.
\begin{center}
\textit{a day's work, a month's pay, in a year's time, for God's sake}
\end{center}
Note that the possessive is also used to refer to \textit{shops, restaurants, churches, universities, etc.}, using the name or job title of the owner.
\begin{center}
\textit{
		I want to go to \textbf{Luigi's} for dinner.\\
		Peter has an appointment \textbf{at the dentist's} at 10 a.m.
}
\end{center}
\section{Pronouns}
A \textbf{pronoun} is a word that replaces a noun in a sentence, making the subject a person or a thing
\subsection{Subject Pronouns}
A \textbf{subject} is the person or thing that performs the action in the clause or sentence.

A \textbf{subject pronoun} is a pronoun that takes the place of a noun as the subject of a sentence

\begin{center}
\textit{
\textbf{She} told me about her worries.}
\end{center}
Subject pronouns replace nouns that are the subject of their clause.
\begin{table}[h]
\begin{center}
		\begin{tabular}{|c|c|c|}
		\hline
		& \textbf{Singular} & \textbf{Plural} \\
		\hline
		$1^{st}$ person & I & we \\ \hline
		$2^{nd}$ person & you & you \\ \hline
		$3^{rd}$ person & he/she/it & they \\ \hline
	\end{tabular}
\end{center}
\caption{\label{tab:nouns4}Singular and plural forms for subject pronouns}
\end{table}

\newpage
We should replace the subject with a subject pronoun to avoid repetition.
\begin{center}
\textit{
\textst{Mary is a student and Mary is very hard working.}\\
Mary is a student and \textbf{she} is very hard working.}
\end{center}

We use the subject pronoun \textit{it} when we refer to objects, things, animals, or ideas.
\begin{center}
		\textit{ Love is eternal. \textbf{It} will last forever.}
\end{center}

Sometimes when we don't know the sex of a baby, we can use \textit{it}'.
\begin{center}
\textit{
Their baby is so small. \textbf{It} only weights 2 kilos.}
\end{center}
\hspace{0.4cm} We use \textit{it} when we talk about \textit{time, weather, or temperature}.
\begin{center}
\textit{
	What time is \textbf{it}? - \textbf{It}'s 7 o'clock.\\
	\textbf{It}'s quite cold today.}
\end{center}

\subsection{Object Pronouns}
An \textbf{object} is the person or thing that receives the action in the clause or sentence.\\
An \textbf{object pronoun} is a pronoun that takes the place of a noun as the object of a sentence.
\begin{center}
\textit{
She told \textbf{me} about her worries.}
\end{center}

Object pronouns are used to replace nouns that are the direct or indirect object of a clause.
\begin{table}[h]
\begin{center}
\begin{tabular}{|c|c|}
	\hline
	\textbf{Subject} & \textbf{Object} \\
	\hline
	I & me \\ \hline
	you & you \\ \hline
	he & him \\ \hline
	she & her \\ \hline
	it & it \\ \hline
	we & us \\ \hline
	they & them \\
	\hline
\end{tabular}
\end{center}
\caption{\label{tab:nouns5}Subject and Object Pronouns}
\end{table}

Object pronouns come either after a verb or a preposition.
\begin{center}
\textit{
Ethan asked \textbf{me} to talk to \textbf{them}.}
\end{center}

Note that the subject pronoun \textit{it} and the object pronoun \textit{it} look the same.

\begin{center}
\textit{
Do you know the movie 'Pertty Lady'? \textit{it} is my favourite!} (subject pronoun)\\
\textit{ I've seen \textit{it} many times.} (object pronoun)
\end{center}
Remember that object nouns are always the recipients of the action in sentence.
\begin{center}
\textit{
		\textst{He and me went to the movies}. \textbf{He and I} went to the movies.\\
\textst{Mrs. Keith called her and I}. Mrs. Keith called \textbf{her and me}.}
\end{center}

We should replace the object with an object pronoun to avoid repetition.
\begin{center}
\textit{
I can't stop thinking about Amy. \textst{I can't stop imagining my future with Amy}. I can't stop imagining my future with \textbf{her}.}
\end{center}

\subsection{Possessive Pronouns}
\textbf{Possessive pronouns} are pronouns that demostrate ownership.
\begin{center}
\textit{This car is \textbf{mine}.}
\end{center}
\newpage
Possessive pronouns are used instead of a possessive adjective and noun. Study the following table:
\begin{table}[h]
\begin{center}
\begin{tabular}{|c|c|c|c|}
		\hline
		\textbf{Subject} & \textbf{Object} & \textbf{Possessive Adjective} & \textbf{Possessive Pronoun}\\ \hline
		I & me & my & mine \\ \hline
		you & you & your & yours \\ \hline
		he & him & his & his \\ \hline
		she & her & her & hers \\ \hline
		it & it & its & its \\ \hline
		they & them & their & theirs\\ \hline
\end{tabular}
\end{center}
\caption{Possessive Adjectives \& Possessive Pronouns} \label{tab:nouns6}
\end{table}

\section{Articles}
\textbf{Articles} are words that define a noun as specific or unspecific.\\
English has two types of articles:
\begin{itemize}
		\item Indefinite: \textit{a/an}
		\item Definite: \textit{the}
\end{itemize}
\begin{center}
\textit{I'm \textbf{a} nurse. \textbf{The} hospital I'm working in is huge.}
\end{center}

\subsection{Indefinite Article}
The \textbf{indefinite article} takes two forms: \textbf{a/an}. Use the indefinite article \textbf{a} when it precedes a word that \textbf{begin} with a \textbf{consonant}. Use the indefinite article \textbf{an} when it precedes a word that \textbf{begins} with a \textbf{vowel}.
\begin{center}
		\textit{ \textbf{a} table, \textbf{an} umbrella, \textbf{a} university, \textbf{an} honest person.}
\end{center}
The indefinite article \textbf{a/an} indicates that a noun refers to a general idea rather than a particular thing.
\begin{center}
\textit{What does \textbf{a} fox say?}
\end{center}
We use \textbf{a/an} when the listener does not know which person or thing we are talking about.
\begin{center}
		\textit{Helen's brother works in \textbf{a} factory. I don't know which factory exactly.}
\end{center}
If we refer to something for the first time, it will be new information for the listener so we use \textbf{a/an}.\\
When referencing to the same thing again use \textbf{the} because now the listener knows what we are talking about.
\begin{center}
\textit{I bought \textbf{a} new computer. It's really great! \textbf{The} computer is much better than my previous one.}
\end{center}

\subsection{Definite Article}
The \textbf{definite article} is the word \textbf{the}. It limits the meaning of a noun to one particular thing. We use \textbf{the} when it is clear which thing or person we are talking about.
\begin{center}
\textit{ \textbf{The} cake is in the fridge. I know that Kate made it.}
\end{center}

We use the definite article \textbf{the} with:
\begin{enumerate}[label=\alph*)]
\item Nationalities and other groups\\
		e.g. \textit{ \textbf{the} French, \textbf{the} Italians, \textbf{the} old, \textbf{the} poor.}
\item Time\\
		e.g. \textit{in \textbf{the} past, in \textbf{the} future (but: \textbf{at present}.)}
\item Superlatives\\
		e.g. \textit{You are \textbf{the} first one!}
\item Musical instruments\\
		e.g. \textit{I played \textbf{the} piano as a kid.}
\item Countries which are a group or plual\\
		e.g. \textit{ \textbf{the} U.S., \textbf{the} U.K., \textbf{the} United Arab Emirates, \textbf{the} Netherlands }
\item Names of ship.\\
		e.g. \textit{We sailed on \textbf{the} Claudia}
\item Oceans\\
		e.g. \textit{ \textbf{the} Pacific, \textbf{the} Atlantic}
\item Rivers\\
		e.g. \textit{ \textbf{the} Amazon, \textbf{the} Nile}
\end{enumerate}

Note that we use \textbf{zero article} with \textbf{plurals} and \textbf{uncountable nouns} when we are generally talking about something.
\begin{center}
		\textit{ \textbf{Dogs} are not allowed in that shop.} (We are talking about dogs in general.)\\
		\textit{ \textbf{The dogs} next door were barking at night.} (W are talking about the particular dogs.)
\end{center}

\section{Demonstratives}

\textbf{Demonstratives} are words that show which person or thing is being referred to.
Demonstratives show where an object, event, or person is in relation to the speaker. They can refer to a physical or a psychological closeness or distance.
\begin{center}
		\textit{\textbf{This} is Hugh, and \textbf{that} is Kevin.}
\end{center}

\begin{table}[h]
\begin{center}
\begin{tabular}{|c|c|c|}
		\hline
		     	 & \textbf{Near the speaker} & \textbf{Far from the speaker} \\ \hline
		Adverbs  & here & there \\ \hline
		Demonstratives with singular and uncountable nouns & this & that \\ \hline
		Demonstratives with plurar countable nouns & these & those\\ \hline
\end{tabular}
\end{center}
\caption{\label{tab:nouns7}Demonstratives}
\end{table}
Demonstratives can be placed before the noun or the adjective that modifies the noun.
\begin{center}
\textit{ \textbf{That old man} stole my purse!\\
\textbf{These oranges} are delicious!}
\end{center}
Demonstratives can also appear before a number by itself when the noun is understood from the context.
\begin{center}
\textit{I'll take \textbf{this one}, please.} = \textit{I'll take this watermelon, please.}
\end{center}
Demonstratives can be used by themselves when the noun they modify is understood from the context.
\begin{center}
\textit{ \textbf{Those} aren't yours. Put them back.} = \textit{Those shoes aren't yours. Put them back.}
\end{center}
When talking about events, the \textbf{near demonstratives} are often used to refer to the \textbf{present} while the \textbf{far demonstratives} often refer to the \textbf{past}.
\begin{center}
\textit{ \textbf{This situation} is quite unstable. \\
\textbf{That event} made me realise how important my family is to me.}
\end{center}

\section{Distributives}
\textbf{Distributives determiners} or simply \textbf{distributives} refer to a group of people or things, and to individual members of the group.\\
\indent They show different ways of looking at the individuals within a group, and they express how something is distributed, shared, or divided.
\begin{center}
\textit{ \textbf{All people} want to love and to be loved.\\
\textbf{Each} person is unique. \textbf{Every} person is unique.\\
\textbf{Both of us} like Mexican food.}
\end{center}

\subsection{All}
The distributive determinal \textbf{all} is used to talk about a whole group, with a special emphasis on the fact that nothing has been left out.\\
\indent \textbf{All} can beused with uncountable nouns and plural countable nouns by itself. In this usage, it refers to the group as a concept rather than as individuals.
\begin{center}
\textit{ \textbf{All parents} want the best for their childre.}
\end{center}
\textbf{All} can be used with uncountable nouns and plural countable nouns preceded by \textbf{the} or a \textbf{possessive adjective}. In these uses, the word \textbf{of} can be added just after \textbf{all} with no change in meaning.
\begin{center}
\textit{ Have you eaten \textbf{all the cookies} in the jar?} = \textit{Have you eaten \textbf{all of the cookies} in the jar?.}
\end{center}
\textbf{All} can be used with \textbf{plural pronouns} preceded by \textbf{of}.
\begin{center}
\textit{ \textbf{All of us} are going to be there tonight.}
\end{center}
\textbf{All} can be used in questions and exclamations with \textbf{uncountable nouns} preceded by \textbf{this/that} or with \textbf{countable nouns} preceded by \textbf{hese/those}.  In these uses, the word \textbf{of} can be added just after \textbf{all} with no change in meaning.
\begin{center}
\textit{ Look at \textbf{al this snow} out there!\\
What are \textbf{all these people} doing in our house?}
\end{center}

\subsection{Half}
The distributive determiner \textbf{half} is used to talk about a whole group divided in \textbf{two}. \textbf{Half} can be used as a distributive in several different patterns.\\
\textbf{Half} can refer to measurements if it is followed by an indefinite article \textbf{a/an} and a noun.
\begin{center}
\textit{I'll be back in \textbf{half an hour}.}
\end{center}
\textbf{Half} can be used with plural pronouns preceded by \textbf{of}.
\begin{center}
\textit{ \textbf{Only half of us} are going to be there tonight.}
\end{center}
\textbf{Half} can be used with nouns preceded by \textbf{the, a/b, a demonstrative, or a possessive adjective}. In this case, the meaning refers to a concrete, physical division.
The word \textbf{of} can be added just after \textbf{half} with no change in meaning.
\begin{center}
\textit{ \textbf{Half the people} have already left the party.\\
Putting \textbf{half a kilo of sugar} into the topping will ruin the cake.\\
I want \textbf{half of that cake}!\\
Sorry, but I used \textbf{half of your eggs} making breakfast today.}
\end{center}

\subsection{Each and Every}
The distributives \textbf{each} and \textbf{every} are both related to describing the members of a group. These distributives can only be used with \textbf{countable nouns} by being placed before the nouns. \\
\indent In many cases, they are interchangable but there is a \textbf{subtle difference} between them.
\subsubsection{Each}
\textbf{Each} is used to describe and highlight an individual member of a group, or multiple individuals. By using \textbf{each} you recognise the item is a part of a group, but that it also needs to be pointed out as a singular item too.
\begin{center}
\textit{ \textbf{Each book} on the shelf had a unique cover.}
\end{center}
\textbf{Each} can be used with plural nouns and pronouns but \textbf{must} be followed by \textbf{of}.
\begin{center}
\textit{ \textbf{Each of the pupils} received a Christmas card.}
\end{center}
\textbf{Each} can be used after the subject or at the end of a sentence.
\begin{center}
\textit{ \textbf{My siblings each} have their own room.\\
My mother gave my sister and I \$20 \textbf{each}.} = (gave \$20 to each of us.)
\end{center}
\subsubsection{Every}
\textbf{Every} by contrast is a way of referring to the group as a collection of individual members. \textbf{Every} cannot be used with plural nouns.
\begin{center}
\textit{ \textst{ \textbf{Every boys} in my class wanted that computer game.} \textbf{Every boy} in my class wanted that computer game.}
\end{center}
\textbf{Every} can express different points in a series, especially with time expressions.
\begin{center}
\textit{ \textbf{Every} morning Phillip goes for a run.\\
And \textbf{every time} Ann would forgive him.}
\end{center}

\subsection{Both}
\textbf{Both} refers to the whole pair and is equivalent to \textit{'one and the other'}. \textbf{Both} can be used with plural nouns on its own, or it can be followed by \textbf{of}, with \textbf{of} without an article. When followed by a plural pronoun, \textbf{both} must be separated from the pronoun by \textbf{of}.
\begin{center}
		\textit{ \textbf{Both (of) my parents} approve of me going to college.\\
		I told \textbf{both of them} to give me a call.}
\end{center}
\textbf{Both} cannot be used with singular nouns, because it refer to two thigns.
\begin{center}
\textit{ \textst{Both my sister likes travelling.} \textbf{Both my sisters like travelling.}}
\end{center}

\subsection{Either}
\textbf{Either} is positive and when used alone refers to one of the two members of the pair. It is equivalent to \textit{'one or the other'}. Because it refers to just one member of a pair, \textbf{either} must be used before a singular noun. It can also be used with a plural noun or pronoun if followed by \textbf{of}.
\begin{center}
		\textit{ \textbf{Either day} is fine.\\
		We could stay at \textbf{either of the hotels}.}
\end{center}
\textbf{Either} can also be used with \textbf{or} in a construction that talks avout each member of the par in turn. The meaning reamins the same, but in this case \textbf{either} is not functioning as a distributive. It is functioning as a \textbf{conjunction}.
\begin{center}
\textit{ You can have \textbf{either} ice cream \textbf{or} cake.}
\end{center}

\subsection{Neither}
\textbf{Neither} is negative and when used alone refer to the whole pair. It is equivalent to \textit{'not one or the other'}. Because it refers to just one member of a pair, \textbf{neither} must be used before a singular noun. It cal also be used with a plural noun or pronoun if followed by \textbf{of}.
\begin{center}
\textit{ \textbf{Neither date} is convenient for me.\\
\textbf{Neither of these dresses} suits her.}
\end{center}
\textbf{Neither} can also be used with \textbf{nor} in a construction that talks about each member of the pair it turn. The meaning reamins the same, but in this case \textbf{neither} is not functioning as a distributive. It is functioning as a \textbf{conjunction}.
\begin{center}
\textit{ It is \textbf{neither} snowing \textbf{nor} raining.}
\end{center}

\newpage
\section{Quantifiers}
We use \textbf{quantifiers} when we want to give someone information about the number of something, the are adjectives and adjectival phrases that give approximate or specific ansers to the questions \textit{'How much?'} and \textit{'How manyt?'}

\begin{center}
	\textit{ \textbf{Most} children start school at the age of five. \\
		I ate some \textbf{rice}. \\
		There are \textbf{a lot of} dogs. }
\end{center}

We can use \textbf{quantifiers} with both \textbf{count} and \textbf{uncountable} nouns:
 \begin{center}
 	\textit{How \textbf{much} coffee do we have left. \\
	 		How many \textbf{cookies}  do you have?}
 \end{center}

\textbf{How much} is used to ask about uncountable nouns and when we want to know the price of something.
\begin{center}
	\textit{ \textbf{How much} this computer cost? }
\end{center}

\subsection{A few, a little}
A (very) few, (very) little are generally used in affirmative statements, not negatives or questions.
\begin{table}[h]
	\begin{center}
	\begin{tabular}{|c|c|}
		\hline
		\textbf{With countable nouns} & \textbf{With uncountable nouns} \\ \hline
		\textbf{(very) few} = hardly any or not enough & \textbf{(very) little} = hardly any or not enough \\ \hline
		\textit{I have \textbf{(very) few} toys.} & \textit{We have \textbf{(very) little} coffe left. } \\ \hline
		\textbf{a few} = some or enough & \textbf{a little} = some or enough \\ \hline
		\textit{I have \textbf{a few} examples to show } & \textit{I have \textbf{a little} coffe left but I can make me a cup of coffe } \\ \hline
	\end{tabular}
\end{center}
\caption{\label{tab:quantifiers1}A few vs A little}
\end{table}

\subsection{Much and Many}
Normally, we use \textbf{much} and \textbf{many} only in questions and negative clauses. But can be used in affirmative sentences in combination with \textit{too} and \textit{so}. In this case, they denote the excessive amount of something.

\begin{center}
	\textit{How \textbf{much} money do you have left? \\
	There are \textbf{too many} people. \\
	You put a \textbf{lot of sugar} on my coffee! }
\end{center}

We use \textbf{much} to talk about the quantity of uncountable nouns or the price of something, while we use \textbf{many} when we talk about the quantity of countable nouns.
\begin{center}
	\textit{I have \textbf{many} friends. \\
	She has too \textbf{much}  money }
\end{center}

\subsection{A lot, most}
Note that in spoken English and informal writing when we want to indicate a large quantity of something we tend to use \textbf{ a lot, a lot of, lots of}.\\

\textbf{A lot} means very often or very much. It is used as an adverb. It often comes at the end of a sentence and \textbf{never} before a noun.

\begin{center}
	\textit{My brother plays videogames \textbf{a lot}. \\
	She's \textbf{a lot} happier after quitting her job. }
\end{center}

We use the quantifier \textbf{most} to talk aboout quantities, amounts and degree. We can use it with a noun (as a determiner) or without a nount (as a pronoun). \\
\newpage
We use \textbf{most} with nouns in the meaning \textbf{the majority of}. If there is no article, demonstrative or possessive pronoun, we use \textbf{most} right before the noun.
 \begin{center}
 	\textit{ \textbf{Most tap water} is drinkable. }
 \end{center}

When we are talking about the majority of a specific set of something, we use \textbf{most of the + noun}.

\begin{center}
	\textit{ \textbf{Most cakes} are sweet. } (cakes in general) \\
	\textit{ The party was amazing. Kate made \textbf{most of the cakes} herself. } (a specific set of cakes at the party)
\end{center}

We can leave out the noun with \textbf{most} when th noun is obvius from the context.

\begin{center}
	\textit{Students can eat in the cafeteria but \textbf{most} bring food from home.} (=most students)
\end{center}

\subsection{Some, Any and Enough}

We use \textbf{some, any} when we are talking about limited but rather indefinite number of quantities.

In general, we use \textbf{some} for affirmative sentences, and \textbf{any} for negatives and questions. Both can be used with countable and uncountable nouns.
\begin{center}
	\textit{Jane bought \textbf{some} flowers. \\
	Did Jave buy \textbf{any} flowers? - No, she didn't buy \textbf{any}. }
\end{center}

\textbf{Some} can be used for questions, typically offers and requests, if we think the answer will be positive.

\begin{center}
	\textit{Would you like \textbf{some} tea.}
\end{center}

\textbf{Any} can be used in the meaning \textit{'it doesn't matter wich'}.

\begin{center}
	\textit{You cant take \textbf{any} bus. They all go to the centre.} (=it doesn't matter which bus you take=
\end{center}

We use \textbf{enough} to indicate sufficiency, while in negative sentences it means less than sufficient or less than necessary.
\begin{center}
	\textit{I'll take your t-shirt. It's \textbf{big enough} to fit me. \\
	Sorry, but I can't go with you. I don't have \textbf{enough money} for that. }
\end{center}

\section{Verb Conjugation}

\textbf{Verb conjugation} refer to how a verb changes to indicate a different person, number, tense, or mood. In other words, \textbf{conjugation} is the changing of a verb's form to express a different person, number, tense, aspect, or gender. In order to communicate in more than one tone, verbs must be conjugates. To conjugate something is to change a verb's form to express a different meaning.

\begin{center}
	\textit{I'm a stundent.} ($1^{st}$ person, singular, present simple, indicative mood)
\end{center}

\subsection{First, Second and Third Person}
Verbs should be conjugated with regard to person. Depending on the subject, a verb can stand in the first, second, or third person.

\begin{table}[h]
	\begin{center}
	\begin{tabular}{|c|c|c|}
		\hline
		  & \textbf{Singular} & \textbf{Plural} \\ \hline
		\textbf{$1^{st}$ person} & I & we \\ \hline
		\textbf{$2^{nd}$ person} & you & you \\ \hline
		\textbf{$3^{rd}$ person} & he, she, it & they \\ \hline
	\end{tabular}
\end{center}
	\caption{\label{tab:vrbcjg1}First, Second and Third Person (Singular and Plural).}
\end{table}

\newpage
As you can se, the pronouns \textbf{I, were} refer to the first person; \textbf{you}, to the second person; \textbf{he, she, it, they}, to the third person.

\begin{center}
	\textit{We work on Saturdays.} (first person) \\
	\textit{You need to take a break.} (second person) \\
	\textit{It is snowing outside.} (third person)
\end{center}

Usually we assume the person of the verb in the sentence automatically as we almost always state a subject explicitly.

\begin{center}
	\textit{Sarah has signed up for a yoga class.} ( \textbf{Sarah} can be substituted with the pronoun \textbf{she}; the verb is in the third person)
 \end{center}

Note that the verb \textbf{to be} is irregular and has three forms in present tenses and two forms in past tenses. These forms depend on the person expressed by the subject.

\begin{table}[h]
	\begin{center}
	\begin{tabular}{|c|c|c|c|c|c}
		\hline
		 & \multicolumn{2}{|c|}{ \textbf{Present}} & \multicolumn{2}{|c|}{ \textbf{Past}} \\ \hline
		\textbf{$1^{st}$ person} & I am & we are & I was & we were \\ \hline
		\textbf{$2^{nd}$ person} & you are & you are & you were & you were \\ \hline
		\textbf{$3^{rd}$ person} & he/she/it is & they are & he/she/it was & they were \\ \hline
	\end{tabular}
\end{center}
\caption{\label{tab:vrbcjg2}Verb To Be forms.}
\end{table}

\section{Simple Tense}
\subsection{Past Simple}
The \textbf{past simple} is used to write and talk about completed actions that happened in a time before the present. It is the basic form of the past tense in English. We use the \textbf{past simple} when we talk about an action which happened at a definite time in the past.

This tense emphasizes that the action is finished.

We can also use this tense to talk about how someone felt about something.

\begin{center}
	\textit{I \textbf{solved} the puzzle. \\
	I \textbf{was} happy for your succes.}
\end{center}

\subsubsection{How to form the past simple tense}
\begin{itemize}
	\item infinitive + (e)d \\
		e.g. \textit{He \textbf{worked} part-time as a waiter. \\
		We \textbf{liked} our stay at the hotel.} \\
	Note that \underline{all persons}  have the same form.
	\item cons + -y $\Rightarrow$ cons + -ied \\
		e.g. \textit{cry-cr\textbf{ied}, try,tr\textbf{ied}}
	\item vowel + const $\Rightarrow$ vowel + double const + ed \\
		e.g. \textit{stop-sto\textbf{pp}ed, regret-regre\textbf{tt}ed}
\end{itemize}

Remember that irregular verbs don't follow the rules above, use \underline{the past tense form of the irregular verbs} to make sentences in the past simple.
\begin{center}
	\textit{be-\textbf{was/were}, eat-\textbf{ate}, drink-\textbf{drank}}
\end{center}

The \textbf{past tense} of the verb \textbf{to be} depends on the person of the subject. (Table - \ref{tab:vrbcjg2})

\begin{table}[h]
	\begin{center}
	\begin{tabular}{|c|c|}
		\hline
		I was & We were \\ \hline
		You were & You were \\ \hline
		he/she/it was & They were \\ \hline
	\end{tabular}
\end{center}
\caption{\label{tab:pastsimple1}Past forms of verb To Be}
\end{table}


\subsubsection{Positive, negative, and questions forms}

\inline{ \textbf{did}}{Positive}/\inline{\textbf{did not}}{Negative} + Verb

\begin{itemize}
	\item (+) \textit{His sister \textbf{lived} in Sutton, London.}
	\item (-) \textit{His sister \textbf{did not live} in Sutton. She \textbf{lived} in Harrow. }
	\item (?) \textit{ \textbf{Did} his sister \textbf{live} in Sutton?  }
	\item (?) \textit{Where \textbf{did} his sister \textbf{live} in London?}
\end{itemize}

\subsubsection{Using time markers}

Yesterday, last night, (not) a long time ago, two years ago, etc.

\begin{center}
	\textit{Shakespeare died \textbf{in 1616}.\\
	Ryan did not go to work \textbf{yesterday}. He got sick. \\
	\textbf{When} did you move to Spain? - I moved ther \textbf{not a long time ago}. }
\end{center}

Note that we use \textbf{did/did not} with the verb \textbf{to have}.

\begin{center}
	\textit{I didn't have enough money to buy a new computer.}
\end{center}

But we do \textbf{not use did} with the verb \textbf{to be (was/were)}.

\begin{center}
	\textit{- Why \textbf{were you} so angry? \\
	- \textbf{I wasn't} angry. \textbf{This was} my usual self.}
\end{center}

\subsection{Present Simple}
The \textbf{present simple} also called \textit{present indefinite} is a verb tesne which is used to show repetition, habit or generalization. We use the \textbf{present simple} when we talk about things in general.

We use this tense to say that something:
\begin{itemize}
	\item Happens all the taime.
	\item Happens repeatedly.
	\item Is true in general.
\end{itemize}

\begin{center}
	\textit{Jane \textbf{works} as a barista. Her shift \textbf{begins} at 7 a.m.}
\end{center}

\subsubsection{How to form the present simple tense}
The present tense is the \textbf{base form} of the verb
\begin{center}
	\textit{I \textbf{work} in London.}
\end{center}
But with the third person singular (she/he/it), we add an -s
\begin{center}
	\textit{She \textbf{works} in London}
\end{center}
and when the verb ends in -o, -s, -ch, -sh, -x, we add -es instead
\begin{center}
	\textit{My sister \textbf{watches} TV in the evening and my brother \textbf{does} his homework.}
\end{center}

Remember that such verbs as \textbf{to be} and \textbf{to have} are \underline{irregular}.

\begin{table}[h]
	\begin{center}
	\begin{tabular}{|c|c|c|c|}
		\hline
		\multicolumn{2}{|c|}{ \textbf{To Be}} & \multicolumn{2}{|c|}{ \textbf{To Have}} \\ \hline
		I am & we are & I have & we have \\ \hline
		you are & you are & you have & you have \\ \hline
		he/she/it is & they are & he/she/it has & they have \\ \hline
	\end{tabular}
\end{center}
\caption{\label{tab:presentsimple1}Present simple: to be - to have}
\end{table}

Note the difference between BrE and AmE:
\begin{center}
	(BrE) - \textit{I have got a car.} \quad (AmE) - \textit{I have a car.}
\end{center}

\subsubsection{Positive, negative, and questions forms}

\inline{\textbf{do not/does not + verb}}{Negative   }

\begin{itemize}
	\item (+) \textit{He \textbf{gets up} at 6 o'clock every morning.}
	\item (-) \textit{He \textbf{does not get up} at 6 o'clock every morning. \\
		He \textbf{gets up} at 7.}
	\item (?) \textit{ \textbf{Does he get up} at 6 o'clock every morning?}
	\item (?) \textit{ \textbf{When does he get up?} }
\end{itemize}

\subsubsection{Using time markers}
You can add time markes such as always, often, usually, sometimes, rarely, never, every day, etc.

\begin{center}
	\textit{I \textbf{usually} cook at home but my friends \textbf{always} eat at the local cafe. \\
		Kim is \textbf{always} late for classes.}
\end{center}

Notice where they are places in the sentences.


\subsection{Subject-Verb Agreement}
The \textbf{subject-verb agreement} is the correspondence of a verb with its subject in person (firth, second, or third) and number (singular or plural).

\begin{center}
	\textit{ \textbf{Liz is} an accountant and \textbf{she has} a typical 8-5 job. }
\end{center}

Subjects and verbs must agree with one another in person (first, second, or third).

Note that \textbf{subject-verb agreement} rules of the verb \textbf{to be} in present tenses.

\begin{table}[h]
	\begin{center}
	\begin{tabular}{|c|c|c|}
		\hline
		& \textbf{Singular} & \textbf{Plural} \\ \hline
		\textbf{$1^{st}$ person} & I am & we are \\ \hline
		\textbf{$2^{nd}$ person} & you are & you are \\ \hline
		\textbf{$3^{rd}$ person} & he/she/it is & they are \\ \hline
	\end{tabular}
\end{center}
\caption{\label{tab:verbagre1}Subject-Verb Agreement: To Be}
\end{table}

\begin{center}
	\textit{I am a student} ($1^{st}$ person), \textit{my brother is a pupil} ($3^{rd}$ person), \textit{and you are a teacher} ($2^{nd}$ person).
\end{center}

Subjects and verbs must agree with one another in number (singular or plural). Thus, if a subject is singular, its verbs must also be singular; if a subject is plural, its verb must also be plural.

\begin{center}
	\textit{ \textbf{She cooks} dinner, and \textbf{her brothers make} breakfast.}
\end{center}

When the subject of the sentence is composed of two or more nouns or pronouns connected by the conjunction \textbf{and}, use a plural verb.

\begin{center}
	\textit{ \textbf{Brothers and sisters don't} often \textbf{get along}. }
\end{center}

The words \textbf{each, each one, either, neither, everyone, everybody, anyone, anybody, nobody, somebody, someone,} and \textbf{no one} are singular and require a singular verb.

\begin{center}
	\textit{ \textbf{Each of these suggestions is} interesting. \\
	\textbf{Someone was standing} at the door.}
\end{center}

When two or more singular nouns or pronouns are connected by \textbf{or} or \textbf{nor}, use a singular verb.

\begin{center}
	\textit{ \textbf{Either your mother or dad needs} to contact me.}
\end{center}

\subsubsection{The Rule of Proximity}

When a compound subject contains both a singular and a plural noun or pronouns joined by \textbf{or} or \textbf{nor}, the verb should agree with the part of the subject that is closer. (also called \textbf{the rule of proximity}). \\

\begin{center}
	\textit{The teacher or \textbf{the students write} homework on the board. \
	The students or \textbf{the teacher writes} homework on the board.}
\end{center}

\subsubsection{The Inverted Subject}

In sentences beginning with \textbf{there is} or \textbf{there are}, the subject follows the verb (also called \textbf{the inverted subject}). As \textbf{there} is not the subject, the verbs agrees with what follows.

\begin{center}
	\textit{There \textbf{is a book} on the table. \
	There \textbf{are books} on the table.}
\end{center}

\subsubsection{More about subject-verb agreement}

Note the \textbf{subject-verb agreement} with words that indicate portions (e.g \textit{a lot, a majority, some, all}): If the noun after \textbf{of} is singular, use a singular verb; if it is plural, use a plural verb.

\begin{center}
	\textit{ \textbf{There is a lot of fuss} around his arrivel. \
	\textbf{There are a lot of people} in the room.}
\end{center}

Use a singular verb with distances, periods of time, sumes of money, etc. when considered as a unit.

\begin{center}
	\textit{ \textbf{Ten dollars is} a high price to pay for socks.} \\
	But: \textit{ \textbf{Ten dollars} (i.e. dollar bills) \textbf{were} scattered on the floor. }
\end{center}

Collective nouns are words that imply more than one person but are considered singular and take a singular verb (e.g. \textit{family, group, team, committee, class, etc.}).
\begin{center}
	\textit{ \textbf{My family} is very big.}
\end{center}

\subsection{Future Simple}
The \textbf{future simple tenses} is often called the \textbf{"will tense"} because we make the \textbf{future simple} with the modal auxiliary \textbf{will}.

We can refer to the future by using \textbf{will, be going to} or by using \textbf{present tenses.}

We use the \textbf{will} future when we want to talk generally about future beliefs, opinions, hopes and predictions.

\begin{center}
	\textit{I promised myself that once I start college \textbf{I will do} all my assignments on time. }
\end{center}

\subsubsection{Positive, negative, and questions forms}

\inline{\textbf{will ('ll)/}}{Positive} \inline{ \textbf{will not (won't)}}{Negative} + verb

\begin{itemize}
	\item (+) \textit{Sam \textbf{will} probably \textbf{move} to Canada next year.}
	\item (-) \textit{Sam \textbf{won't move} to Canada next year. He\textbf{'ll move} to the US.}
	\item (?) \textit{\textbf{Will} Sam \textbf{move} to Canada next year?}
	\item (?) \textit{Where \textbf{will} Sam \textbf{move} to?}
\end{itemize}

\subsubsection{Using time and probability markers}

Time markers - \textbf{tomorrow, next month, in a day, etc.} \\
Probability markers - \textbf{perhaps, probably, definitely, etc.} \\

\begin{center}
	\textit{ \textbf{Perhaps} it'll snow \textbf{tomorrow}. \\
	I'll \textbf{definitely} finish my essay \textbf{next month}.}
\end{center}

Pay attention to the word order.

\begin{center}
	(+) \textit{We'll \textbf{probably} do it tomorrow.} \\
	(-) \textit{We \textbf{probably} won't do it tomorrow.}
\end{center}

Some speakers use \textbf{shall} to refer to the future in \underline{formal situations} (with / and we).

Nowadays \textbf{shall} is used for \underline{suggestions} only.

\begin{center}
	\textit{ \textbf{Shall I} go or \textbf{shall we} leave together? }
\end{center}

\section{The Gerund}
\textbf{The gerund}  looks exactly the same as a \textbf{present participle}, but it is useful to understand the difference between the two. \textbf{The gerund} always has the same \textbf{function as a noun} (although it looks like a verb).

\begin{center}
	\textit{ \textbf{Hunting} tigers is dangerous.}
\end{center}

Some rule to form the gerund
\begin{itemize}
	\item -e + ing \\
		e.g. \textit{make-making, write-writing}
	\item vowel + cons $\Rightarrow$ double cons + -ing \\
		e.g. \textit{knit-knitting, swim-swimming}
	\item -ie $\Rightarrow$ -y + -ing \\
		e.g. \textit{lie-lying, die-dying}
\end{itemize}

\textbf{The gerund} can be made negativa by adding not.

\begin{center}
	\textit{The best thing for your health is \textbf{not smoking}. }
\end{center}

The \textbf{gerung} can function as:

\begin{enumerate}[label=(\alph*)]
	\item The subject of the sentence. \\
		e.g. \textit{ \textbf{Smoking} causes lung cancer.}
	\item The complement of the verb to be. \\
		e.g. \textit{The hardest thing about learning Russian is \textbf{memorizing} the verbs of movement. }
\end{enumerate}

The \textbf{gerund} can be used:
\begin{enumerate}[label=(\alph*)]
	\item After prepositions or as part of certain expressions. (there's no point in, in spite of, etc.)\\
		e.g. \textit{Can your brother count to ten \textbf{without looking} at his fingers? \\
					 \textbf{There's no point in going} back to his place now. }
	\item After phrasal verbs. They are compòsed of a verb + preposition/adverb. \\
		e.g. \textit{I \textbf{ended up buying} a new computer. \
		Rachel \textbf{gave up drinking} sugar drinks.}
\end{enumerate}

\section{Present Participle}
Most commonly we use the \textbf{present participle -ing} as an element in all continuous verb forms (the present continuous, the past continuous, etc).

The auxiliary verb indicates the tense, while the present participle remains unchanging.

\begin{center}
	\textit{ \textbf{I was playing computer games all night} } (past continuous)
\end{center}

\subsection{How to form the present participle}

\begin{itemize}
	\item Verb ending in -e + -ing \\
		e.g. \textit{like-liking, write-writing}
	\item Verb ending with vowel + cons $\Rightarrow$ double cons + -ing \\
		e.g. \textit{sit-sitting, swim, swimming}
	\item Verb ending in -ie $\Rightarrow$ -y + -ing \\
		e.g. \textit{lie-lying, die-dying}
\end{itemize}

\subsection{Uses}
The present participle is used not only form verb tenses. It can be used:
\begin{enumerate}[label=(\alph*)]
	\item After verbs of movement and position. \\
		e.g. \textit{ She went \textbf{shopping. \\
			} They came \textbf{running} towards me. }
	\item After verbs of perception in the pattern verb + object + present participle to indicate the action being perceived. \\
		e.g. \textit{We saw him \textbf{mowing} the lawn. \
		Liz heard someone \textbf{singing}.}
	\item After verbs of movement, action, or position to indicate parallel activity. \\
		e.g. \textit{He sat \textbf{looking at} the pedestrians. \\
		July walks \textbf{reading her newspaper.} }
	\item As an adjective. \\
		e.g. \textit{Have you heard of that \textbf{amazing} movie? \
		The family was trapped inside the \textbf{burning} barn.}
	\item To explain the cause or reason. The present participle is used instead of a phrase starting with \textbf{as, since, because.} \\
		e.g. \textit{ \textbf{Feeling} hungry, I made myself a sandwich.} (= I made myself a sandwich \textbf{because} I was hungry). \\
		\textit{ \textbf{Knowing} that his roommate was comming, James cleaned the living room.} (= James cleaned the living room \textbf{as} he knew that his roommate was comming.)
\end{enumerate}


\section{Continuous Tense}
The \textbf{continuous tense} shows an action that is, was, or will be in progress at a certain time. The \textbf{continuous tense} is formed with the verb \textbf{to be} + -ing form of the verb (present participle).

\subsection{Past Continuous}
We use the \textbf{past continuous} when we describe a situation, or several situations in progress, happening at the same time in the past.

This is often contrasted with a sudden event in the past simple.

\begin{center}
	\textit{ \textbf{I was working} on my computer and by brother \textbf{was reading} a book when we heard a loud bang on the door.}
\end{center}

\subsubsection{How to form the past continuous}
\inline{\textbf{Was/were}}{Positive} + Verb -ing \quad \inline{\textbf{wasn't/weren't}}{Negative} + Verb - ing

\subsubsection{Positive, negative and question form}
\begin{itemize}
	\item (+) \textit{Jim \textbf{was playing} video games all night.}
	\item (-) \textit{Jim \textbf{was not playing} video games all night. / He \textbf{wasn't playing} video games all night. }
	\item (?) \textit{ \textbf{Was} Jim \textbf{playing} video games all night?}
	\item (?) \textit{ \textbf{Why was} he \textbf{playing} video games all night?}
\end{itemize}

\subsubsection{Using time markers}
at 7 o'clock, for two hours, in January, last week, all night, etc.

\begin{center}
	\textit{Kate was trying to find a nice apartment in her area \textbf{for 5 months}. }
\end{center}

when, while = during the time that

\begin{center}
	\textit{ \textbf{While} they were waiting for the train, it started to rain. \\
	James broke his finger \textbf{when} he was playing basketball.}
\end{center}

\subsubsection{Exceptions}
Non-continuous verbs (e.g. \textit{to love, hate, know, want, etc.} are \textbf{no used} in any \underline{continuous tenses!} Use the past simple instead.

\begin{center}
	\textit{ \textst{I was having fun at the party, but Kim was wanting to go home.} \\
	I was having fun at the party, but \textbf{Kim wanted to go home}. }
\end{center}

\subsection{Present Continuous}
We use the \textbf{present continuous} when we talk about something happening at the time of speaking, or actions happening 'around now', even though not at the moment of speaking.

This tense also has some future meanings.

\begin{center}
	\textit{Hey, \textbf{what are you doing?} - \textbf{I am working on my thesis. I am graduating this semester.} }
\end{center}

\subsubsection{How to form the present continuous}
\inline{to be}{Positive} +  Verb -ing \quad \inline{to be + not}{Negative} + Verb -ing

\subsubsection{Positive, negative and question forms}
\begin{itemize}
	\item (+) He \textbf{is sleeping} on the couch in the living room.
	\item (-) He \textbf{is not sleeping} on the couch in the living room.
	\item (-) He \textbf{isn't sleeping} there.
	\item (?) Where is he? \textbf{Is he sleeping?}
\end{itemize}

\subsubsection{Using time markers}
Now, right now, at the moment, today, this week, etc.

\begin{center}
	\textit{I'm quite busy \textbf{this year} as I'm trying to start my small business.}
\end{center}

\subsubsection{Other uses}
Use the present continuous to talk about changing situtations

\begin{center}
	\textit{ \textst{The population of the world increases very fast.} \\
	The population of the world \textbf{is increasing} very fast.}
\end{center}

\subsection{Future Continuous}
We use the future continuous to say that we will be in the middle of doing something at a certain time in the future.

We often use this tense when we compare what we are doing now with what we will be doing in the future.

\begin{center}
	\textit{The movie starts at 8 and ends at 10. At 9 \textbf{I will be watching} the movie}
\end{center}

\subsubsection{How to form the future continuous}
\inline{will}{Positive} + be + Verb -ing \quad \inline{won't}{Negative} + be + Verb -ing

\subsubsection{Positive, negative and question forms}
\begin{itemize}
	\item (+) Sarah \textbf{will be flying} home at 5 o'clock tomorrow.
	\item (-) Sarah \textbf{will not be flying} home at 5 o'clock tomorrow. / She \textbf{won't be flying} home at 5 o'clock tomorrow.
	\item (?) \textbf{Will} Sarah \textbf{be flying} home at 5 o'clock tomorrow?
	\item (?) \textbf{Where will} she \textbf{be flying} at 5 o'clock tomorrow?
\end{itemize}

\subsubsection{Using time markers}
at 5 o'clock, at that time tomorrow, this evening, in 5 years' time, etc.

\begin{center}
	\textit{Where will you be living \textbf{in 3 years' time}? }
\end{center}

\subsubsection{Other uses}
Use the future continuous to say that something will definitely happen in the future.

\begin{center}
	\textit{I'\textbf{be going} to the shop later. Can I get you anything?}
\end{center}

\subsection{Comparing continuous tenses}
Compare \textbf{will be doing} with other continuous forms.

\begin{center}
	\textit{Jave has an ordinary 9/8 job. \\
	At 11 o'clock yesterday she was working.} (past continuous) \\
	\textit{At 11 o'clock today she is working.} (present continuous) \\
	\textit{At 11 o'clock tomorrow she will be working.} (future continuous)
\end{center}

\section{Past Participle}
A \textbf{past participle} refers to the form of a verb which is used in forming perfect and passive tenses (and sometimes used as an adjective).

\begin{center}
	\textit{Olivia has \textbf{lived} in Greece for 4 years.}
\end{center}

\subsection{How to form the past particple}
We usually add -(e)d to the base form of the regular verb to form the past participle

\begin{center}
	\textit{Jun has just \textbf{painted} this picture.} (present perfect, active voice). \\
	\textit{This picture was \textbf{painted} by Jun a month ago.} (past simple, passive voice)
\end{center}

There is no pattern as to forming the past participle of the irregular verbs. You should always consult a dictionary.

\subsection{Uses}
\begin{enumerate}[label=(\alph*)]
	\item In the perfect tenses (Present Perfect, Past Perfect, Future Perfect). \\
		e.g. \textit{I've \textbf{eaten} to much! I can't move.} (present perfect) \\
			\textit{James had already \textbf{left} when Pam arrived.} (past perfect) \\
			\textit{We will have \textbf{landed} by that hour.} (future perfect)
	\item In the passive voice. \\
		e.g. \textit{He was \textbf{driven} by genuine interest and curiosity.\\
		This dress was \textbf{made} by a famous Italian designer.}
	\item As an adjectice. In this case, place it before a noun. \\
		e.g. \textit{Mike has \textbf{broken} his arm.} $\Rightarrow$ \textit{He has a \textbf{broken} arm now.\\
		Someone has stolen Ann's purse.} $\Rightarrow$ \textit{Her purse was \textbf{stolen.} }
\end{enumerate}

\section{Perfect Tense}
The \textbf{perfect tense} or aspect is a verb form that indicates that an action or circumstance occurred earlier than the time under consideration, often focusing attention on the resulting state rather than on the ocurrence itself.

\begin{center}
	\textit{ \textbf{I have made} dinner}\\
	Although this gives information about a prior action (my making of the dinner), the focus is likely to be on the present consequences of that action (the fact that the dinner is now ready).
\end{center}

\subsection{Present Perfect}
We use the present perfect to describe past events which are connected to the present.

Although this tense can be used to describe different situations.

\begin{center}
	\textit{Sam \textbf{has lost} his keys.} (= He is looking for his keys and he still hasn't found them.)
\end{center}

\subsubsection{How to form the present participle}
\begin{center}
	\inline{have/has}{Positive} + Verb ending in -ed (\textbf{past participle}) or Simple Verb \\
	\inline{haven't/hasn't}{Negative} + Verb ending -ed (\textbf{past participle})  or Simple Verb
\end{center}

\subsubsection{Positive, negative and question forms}
\begin{itemize}
	\item (+) I \textbf{have} already \textbf{seen} that movie. / I\textbf{'ve} already \textbf{seen} that movie.
	\item (-) I \textbf{have not seen} that movie yet. / I \textbf{haven't seen} it yet.
	\item (?) \textbf{Have} I \textbf{seen} that movie?
\end{itemize}

\subsubsection{Uses}
\begin{enumerate}[label=(\alph*)]
	\item Experiences in our life up to now. \\
		e.g. \textit{I'\textbf{ve been} to Spain and Portugal. I really want to go to the UK. I \textbf{haven't been} there yet. }
	\item An event in the past that has a result in the present. \\
		e.g. \textit{Lilly \textbf{has broken} her foot. Her foot is still in a cast.}
	\item A situation that started in the past and coninues until the preset. \\
		I'\textbf{ve lived} here \textbf{for twenty years}. And I am still living here now.
	\item An event in the past that has a result in the preset. \\
		e.g. \textit{Peter \textbf{has ready} 50 pages of his book so far. There are 150 pages left.}
\end{enumerate}

\subsubsection{Using time markers}
Pay attention to the time markers:
\begin{enumerate}[label=(\alph*)]
	\item We use \textbf{every} and \textbf{never} to ask or talk about our experiences in life. \\
		e.g. \textit{Have you \textbf{ever} eaten Chinese food? - I've \textbf{never} eaten it. }
	\item We use \textbf{already} to describe an action which has happened befor; \textbf{yet} - an action which hasn't happened before.
		e.g. \textit{I haven't finished this book \textbf{yet}, and my sister has \textbf{already} begun reading another one.}
	\item We use \textbf{just} when we describe a very recent event. \\
		e.g. \textit{My mom has \textbf{just} come home from work.}
	\item \textbf{Always, often, etc.} can also be used in the present perfect. \\
		e.g. \textit{He has \textbf{always} loved Ann.}
	\item We use \textbf{for} to describe the length of a time period. We use \textbf{since} to describe the point when the time period started. \\
		e.g. \textit{Chris has worked here \textbf{for 5 months}. He has worked here \textbf{since May $5^{th}$.} }
\end{enumerate}

\subsection{Past Perfect}
We use the \textbf{past perfect} to show clearly that one past event happened befor another past event.

We use the past perect in the earlier event.

\begin{center}
	\textit{When I arrived at the party, Tom wasn't there. \textbf{He had gone home} }
\end{center}

\subsubsection{How to form the past perfect}
\inline{had}{Positive} + Verb ending in -ed (past participle) or Simple verb. \\
\inline{hadn't}{Negative} + Verb ending in -ed (part participle) or Simple verb.

\subsubsection{Positive, negative and question forms}

\begin{itemize}
	\item (?) \textbf{Had} Kate \textbf{gone} to bed when you arrived home?
	\item (+) \textbf{Yes, she had.} She \textbf{had gone} to bed when I arrived home. She\textbf{'d gone} to bed.
	\item (-) \textbf{No, she hadn't.} She \textbf{hadn't gone} to bed when I arrived home.
\end{itemize}

\subsubsection{Past perfect vs Present perfect}
The \textbf{past perfect} (\textit{I had done)}  is the past of the \textbf{present perfect} (\textit{I have done})

\begin{itemize}
	\item \textbf{Present} \\
		\textit{I\textbf{'m not} hungry. \textbf{I've just had breakfast}. \\
		Your room \textbf{is} dirty. \textbf{You haven't cleaned it for months.} }
	\item Past \\
		\textit{I \textbf{wasn't} hungry. \textbf{I'd just had breakfast.} \\
		Your room \textbf{was} dirty. \textbf{You hadn't cleaned it for months.} }
\end{itemize}


to think, know, be sure, realize, remember, suspect, understand, etc.

\begin{center}
	\textit{She \textbf{was sure} she hadn't locked the door. \\
	When I got home I \textbf{realized} I'd left my computer at Starbucks.}
\end{center}

\subsubsection{Other Uses}
Many speakers use the \textbf{past perfect} (in case of \textbf{before} or \textbf{after}) to show a strong connection between the two events.

\begin{center}
	\textit{Pam left her house before her parents arrived.} (past simple) \\
	\textit{Pam \textbf{had left} her house before her parents arrived.} (past simple + past perfect)
\end{center}


\subsection{Future Perfect}
We use the \textbf{future perfect} to look back from one point in the future to an earlier event.

The situations has not happened yet, but at a certain time in the future it will happen.

\begin{center}
	\textit{By next week \textbf{I'll have written} 20 pages for my new book. }
\end{center}

\subsubsection{How to form the future perfect}
will + have + Verb ending in -ed (past participle) or simple verb.

\subsubsection{Positive, negative and question forms}
\begin{itemize}
	\item (+) John \textbf{will have arrived} here by 5 p.m. tomorrow.
	\item (-) He \textbf{won't have arrived} here by 5 p.m. tomorrow.
	\item (?) \textbf{Will} he \textbf{have arrived} here by 5 p.m. tomorrow?
\end{itemize}

\subsubsection{Time expressions}
by + time expression

\begin{center}
	\textit{Won't they have invited us \textbf{by Friday}? \\
	James will have finished his thesis \textbf{by this time next week.} }
\end{center}

when, as soon as, before, by the time, etc.

\begin{center}
	\textit{Will you have dressed up \textbf{when I pick you up}? \\
	\textbf{By the time you read this} I will have left the city.}
\end{center}

\subsubsection{Other uses}
The \textbf{future perfect} is used only for actions that will be completed by a particular time in the future.

If the \underline{deadline is not mentioned}, use the future simple instead.

\begin{center}
	\textit{She will leave her hometown. \\
	\textst{She will have left her hometown.} \textbf{She will have left her hometown by this time next year.}  }
\end{center}



\section{Prefect Continuous Tense}
\subsection{Present Perfect Continuous}
\subsection{Past Perfect Continuous}
\subsection{Future Perfect Continuous}



\end{document}
